\documentclass[notheorem]{beamer}
\usepackage[italian]{babel}
\deftranslation{Definition}{Definizione}
\deftranslation{Theorem}{Teorema}
\deftranslation{Lemma}{Lemma}
\deftranslation{Remark}{Nota}
\usepackage[utf8]{inputenc}
\usepackage{amssymb}
\usepackage{amsthm}
\usepackage{tikz}
\usetikzlibrary{arrows, positioning}

% Ambiente per le dimostrazioni

% Ambiente per le dimostrazioni formali formali
\newenvironment{formal_proof}
    {
    \begin{center}
    \begin{tabular} {c c|c}
    }
    {\end{tabular}
    \end{center}
    }

\title{I Sistemi di Logica Modale T, S4, S5:}
\subtitle{Sintassi e Semantica}
\author{Francesco Magliocca}
%\institute{Università di Napoli Federico II}
\date{18/12/2019}

\begin{document}
\frame{\titlepage}
\begin{frame}
\frametitle{Calcolo Proposizionale}
Introduciamo un sistema formale per il Calcolo Proposizionale
che chiameremo $CP$.
\end{frame}
\begin{frame}
\frametitle{Calcolo Proposizionale: L'alfabeto}
Sia $A$ insieme numerable disgiunto da $\{\neg, \lor, (, )\}$ e contenente
almeno tre elementi che indicheremo con $p, q, r$.

Chiamiamo l'insieme $\Sigma = A \cup \{\neg, \lor, (, )\}$ \emph{alfabeto}
e gli elementi di $A$ \emph{variabili proposizionali}.
\end{frame}
\begin{frame}
\frametitle{Calcolo Proposizionale: Il linguaggio}
L'insieme delle formule ben formate (fbf) $L_{CP}$ è il più piccolo sottoinsieme di
$\Sigma^{*} = \bigcup_{n \in \mathbb{N}} \Sigma^n$ tale che:
\begin{itemize}
\item $x \in A \Rightarrow x \in L_{CP}$;
\item $\alpha \in L_{CP} \Rightarrow \neg (\alpha) \in L_{CP}$;
\item $\alpha \in L_{CP}$ e $\beta \in L_{CP} \Rightarrow (\alpha) \lor (\beta) \in L_{CP}$.
\end{itemize}
\end{frame}
\begin{frame}
\frametitle{Calcolo proposizionale: Connettivi aggiuntivi}
Se $\alpha \in L_{CP}$ e $\beta \in L_{CP}$, allora diamo le seguenti definizioni:
\begin{itemize}
\item $(\alpha) \land (\beta) := \neg(\neg(\alpha) \lor \neg(\beta))$;
\item $(\alpha) \rightarrow (\beta) := (\neg(\alpha)) \lor (\beta)$;
\item $(\alpha) \leftrightarrow (\beta) := ((\alpha) \rightarrow (\beta)) \land ((\beta) \rightarrow (\alpha))$.
\end{itemize}
\end{frame}

\begin{frame}
\frametitle{Calcolo Proposizionale: Gli assiomi}
\begin{itemize}
\item HPD: $p \rightarrow (q \rightarrow p)$
\item HPMP: $(r \rightarrow (p \rightarrow q)) \rightarrow (r \rightarrow p) \rightarrow (r \rightarrow q)$
\item $\lor$-I1: $p \rightarrow (p \lor q)$
\item $\lor$-I2: $q \rightarrow (p \lor q)$
\item $\lor$-E: $(p \rightarrow r) \rightarrow (q \rightarrow r) \rightarrow (p \lor q \rightarrow r)$
\item $\land$-I: $p \rightarrow q \rightarrow p \land q$
\item $\land$-E1: $p \land q \rightarrow p$
\item $\land$-E2: $p \land q \rightarrow q$
\item $\neg$-I: $(p \rightarrow q) \land (p \rightarrow \neg q) \rightarrow \neg p$
\item TER: $p \lor \neg p$
\end{itemize}
\end{frame}

\begin{frame}
\frametitle{Calcolo Proposizionale: Le regole di deduzione}
\begin{enumerate}
\item Modus Ponens:

$$\frac{\alpha \rightarrow \beta \ \alpha}{\beta}$$

\item Regola di sostituzione:

Se $\alpha \in L_{CP}$, $x \in A$ e $\beta \in L_{CP}$, allora
la formula $\alpha[x/\beta]$ ottenuta sostituendo uniformente $\beta$ al posto di $x$
all'interno di $\alpha$ è una conseguenza di $\alpha$.
\end{enumerate}
\end{frame}

\begin{frame}
\frametitle{Calcolo Proposizionale: Dimostrazioni}
Una dimostrazione è una sequenza finita $D$ di fbf tale che
ogni termine $D_i$ della sequenza soddisfa una delle seguenti condizioni
\begin{itemize}
\item $D_i$ è un assioma;
\item esistono $D_h, D_k$, con $h < i, k < i$, tali che $D_i$ è derivato per \emph{Modus Ponens}
da $D_h$ e $D_k$.
\item esiste $D_h$, con $h < i$, tale che $D_i$ è derivato tramite
la regola di \emph{sostituzione} a partire da $D_h$.
\end{itemize}

Se $\alpha \in L_{CP}$ ed esiste una dimostrazione $D$ il cui ultimo termine è $\alpha$,
diciamo che esiste una dimostrazione di $\alpha$ e scriviamo $\vdash \alpha$.
\end{frame}

\begin{frame}
\frametitle{Calcolo Proposizionale: Semantica}
Chiamiamo \emph{interpretazione} una funzione $I : A \to \{0, 1\} \subset \mathbb{N}$
Data un'interpretazione $I$ definiamo la funzione $V_I : L_{CP} \to \{0, 1\} \subset \mathbb{N}$
come segue:
\begin{itemize}
\item $V_I(x) = I(x)$, $\forall x \in A$
\item $V_I((\alpha) \lor (\beta)) = max\{V_I(\alpha), V_I(\beta)\}$
\item $V_I(\neg (\alpha)) = 1 - V_I(\alpha)$
\end{itemize}

Sia $\alpha \in L_{CP}$, diciamo che $\alpha$ è \emph{valida} e scriviamo $\vDash \alpha$ se
$V_I(\alpha) = 1$, per ogni interpretazione $I$.
\end{frame}

\begin{frame}
\frametitle{Calcolo Proposizionale: Semantica}
Tutti i connettivi presenti in $CP$ sono interpretati come operatori vero-funzionali.
\end{frame}

\begin{frame}
\frametitle{Calcolo Proposizionale: Semantica}
Chiamiamo \emph{interpretazione} una funzione $I : A \to \{0, 1\} \subset \mathbb{N}$
Data un'interpretazione $I$ definiamo la funzione $V_I : L_{CP} \to \{0, 1\} \subset \mathbb{N}$
come segue:
\begin{itemize}
\item $V_I(x) = I(x)$, $\forall x \in A$
\item $V_I((\alpha) \lor (\beta)) = max\{V_I(\alpha), V_I(\beta)\}$
\item $V_I(\neg (\alpha)) = 1 - V_I(\alpha)$
\end{itemize}
\end{frame}

\begin{frame}
\frametitle{Calcolo Proposizionale: Teorema di Completezza}
Per il sistema $CP$ vale il seguente importante risultato, chiamato
\emph{Teorema di Completezza}:

Per ogni $\alpha \in L_{CP}$ si ha:
$$\vdash \alpha \Leftrightarrow \vDash \alpha$$
\end{frame}

\begin{frame}
\frametitle{Logica Modale}
Estendiamo il sistema $CP$ in modo da introdurre due nuovi connettivi, $\Box$
e $\Diamond$ corrispondenti alle nozioni di \emph{necessità} e \emph{possibilità}.
\end{frame}

\begin{frame}
\frametitle{Logica Modale: Sulla vero-funzionalità di $\Box$ e $\Diamond$}
È noto che qualsiasi operatore vero-funzionale sia definibile in termini
di negazione e disgiunzione. Quindi se $\Box$ e $\Diamond$ fossero interpretabili
come operatori vero-funzionali, potremmo semplicemente aggiungerli come connettivi
derivati in $CP$ senza aggiungere altro.

\end{frame}

\begin{frame}
\frametitle{Logica Modale: Sulla vero-funzionalità di $\Box$ e $\Diamond$}
Ma così non è!

Infatti non basta sapere che ``Vincenzo è ricco'' è vera per esprimersi
circa la verità dell'affermazione ``È necessario che Vincenzo sia ricco''.

Dobbiamo allora estendere $CP$ in maniera sostanziale.
\end{frame}

\begin{frame}
\frametitle{Logica Modale: I Sistemi}

Non c'è un unico sistema per la logica modale, anzi ce ne sono molti.
Ne presentiamo tre:
\begin{enumerate}
\item Sistema T
\item Sistema S4
\item Sistema S5
\end{enumerate}
\end{frame}

\begin{frame}
\frametitle{Sistema T: L'alfabeto}
Estendiamo l'alfabeto di $CP$ aggiungendo il simbolo $\Box$, quindi
chiamiamo l'insieme $\Sigma_1 = A \cup \{\neg, \lor, \Box, (, )\}$ alfabeto

\end{frame}

\begin{frame}
\frametitle{Sistema T: Il linguaggio}
L'insieme delle formule ben formate (fbf) $L$ è il più piccolo sottoinsieme di
$\Sigma_1^{*} = \bigcup_{n \in \mathbb{N}} \Sigma_1^n$ tale che:
\begin{itemize}
\item $x \in A \Rightarrow x \in L$;
\item $\alpha \in L \Rightarrow \neg (\alpha) \in L$;
\item $\alpha \in L$ e $\beta \in L \Rightarrow (\alpha) \lor (\beta) \in L$;
\item {\color{red} $\alpha \in L \Rightarrow \Box (\alpha) \in L.$}
\end{itemize}
\end{frame}

\begin{frame}
\frametitle{Sistema T: Connettivi aggiuntivi}
Se $\alpha \in L$, allora diamo la seguente definizione:
$$\Diamond (\alpha) := \neg(\Box(\neg(\alpha)))$$
\end{frame}

\begin{frame}
\frametitle{Sistema T: Regole di deduzione}
\begin{enumerate}
\item Modus Ponens:

$$\frac{\alpha \rightarrow \beta \ \alpha}{\beta}$$

\item Regola di sostituzione:

Se $\alpha \in L$, $x \in A$ e $\beta \in L$, allora
la formula $\alpha[x/\beta]$ ottenuta sostituendo uniformente $\beta$ al posto di $x$
all'interno di $\alpha$ è una conseguenza di $\alpha$.

\item {\color{red} Regola di necessitazione:

$$\frac{\alpha}{\Box alpha}$$}
\end{enumerate}
\end{frame}

% \begin{frame}
% \frametitle{Sistema T: Sulla regola di necessitazione}
% $$\frac{\alpha}{\Box alpha}$$
% \end{frame}

\begin{frame}
\frametitle{Sistema T: Assiomi}
Gli assiomi del Sistema T sono tutti gli assiomi di $CP$ più i seguenti:
\begin{itemize}
\item T: $\Box p \rightarrow p$
\item K: $\Box(p \rightarrow q) \rightarrow (\Box p \rightarrow \Box q)$
\end{itemize}
\end{frame}

\begin{frame}
\frametitle{Sistema S4}
Il Sistema S4 è ottenuto aggiungendo al Sistema T l'assioma S4:
$$\Box p \rightarrow \Box \Box p$$
\end{frame}

\begin{frame}
\frametitle{Sistema S5}
Il Sistema S5 è ottenuto aggiungendo al Sistema T l'assioma S5:
$$\Diamond p \rightarrow \Box \Diamond p$$
\end{frame}

\begin{frame}
\frametitle{Logica Modale: Dimostrazioni}
Una dimostrazione in un sistema $S$ ($\in \{ T, S4, S5\}$)  è una sequenza finita $D$ di fbf tale che
ogni termine $D_i$ della sequenza soddisfa una delle seguenti condizioni
\begin{itemize}
\item $D_i$ è un assioma di $S$;
\item esistono $D_h, D_k$, con $h < i, k < i$, tali che $D_i$ è derivato per \emph{Modus Ponens}
da $D_h$ e $D_k$.
\item esiste $D_h$, con $h < i$, tale che $D_i$ è derivato tramite
la regola di \emph{sostituzione} a partire da $D_h$.
\item {\color{red} esiste $D_h$, con $h < i$, tale che $D_i$ è derivato tramite
la regola di \emph{necessitazione} a partire da $D_h$}.
\end{itemize}

Se $\alpha \in L$ ed esiste una dimostrazione $D$ in un sistema $S$ il cui ultimo termine è $\alpha$,
diciamo che esiste una dimostrazione di $\alpha$ e scriviamo $\vdash_S \alpha$.
\end{frame}

\begin{frame}
\frametitle{Relazioni tra i sistemi T, S4, S5}
\begin{definition}
Diremo che un sistema formale è meno forte di un altro se tutti i teoremi del primo
sono teoremi anche del secondo.
\end{definition}

\begin{block}{Nota}
È ovvio che $CP$ è meno forte del Sistema $T$, che è meno forte sia del sistema $S4$
che del sistema $S5$.
\end{block}
\end{frame}

\begin{frame}
\frametitle{Relazioni tra i sistemi T, S4, S5 II}
\begin{theorem}
$\vdash \Box p \rightarrow \Box \Box p$ in $S5$, ovvero il Sistema S5 è più forte del Sistema S4.
\end{theorem}
\end{frame}

\begin{frame}
\frametitle{Logica Modale: Semantica}

Un modello di Kripke è una terna $(W, R, I)$ costituita da:
\begin{itemize}
\item $W$ insieme non vuoto i cui elementi sono detti \emph{mondi};
\item $R$ relazione binaria su $W$ detta \emph{relazione di accessibilità};
\item $I : A \times W \to \{0, 1\} \subset \mathbb{N}$ detta \emph{interpretazione}.
\end{itemize}
\end{frame}


\begin{frame}
\frametitle{Logica Modale: Semantica II}

Dato un modello di Kripke $K = (W, R, I)$ definiamo la funzione $V_K : L \times W \to \{0, 1\}$
come segue.
Fissato $w \in W$:
\begin{itemize}
\item $V(x, w) = I(x, w), \forall x \in A$
\item $V(\neg (\alpha), w) = 1 - V(\alpha, w)$
\item $V((\alpha) \lor (\beta), w) = max{V(\alpha, w), V(\beta, w)}$
\item {\color{red} $V(\Box (\alpha), w) = min\{ V(\alpha, w') : w' \in W$ e $w R w' \}$}
\end{itemize}
\end{frame}

\begin{frame}
\frametitle{Sistema T: Semantica}
Chiamiamo T-modello un modello di Kripke in cui la relazione di \emph{accessibilità}
sia \textbf{riflessiva}.

Se $\alpha \in L$, diciamo che $\alpha$ è T-valida e
scriviamo $\vDash_T \alpha$ se per ogni T-modello $K = (W, R, I)$
si ha $\forall w \in W. V_K(\alpha, w) = 1$.
\end{frame}

\begin{frame}
\frametitle{Sistema S4: Semantica}
Chiamiamo S4-modello un modello di Kripke in cui la relazione di \emph{accessibilità}
sia \textbf{riflessiva} e \textbf{transitiva}.

Se $\alpha \in L$, diciamo che $\alpha$ è S4-valida e scriviamo
$\vDash_{S4} \alpha$ se per ogni S4-modello $K = (W, R, I)$
si ha $\forall w \in W. V_K(\alpha, w) = 1$.
\end{frame}

\begin{frame}
\frametitle{Sistema S5: Semantica}
Chiamiamo S5-modello un modello di Kripke in cui la relazione di \emph{accessibilità}
sia la relazione binaria totale su $W$.

Se $\alpha \in L$, diciamo che $\alpha$ è S5-valida e scriviamo
$\vDash_{S5} \alpha$ se per ogni S5-modello $K = (W, R, I)$
si ha $\forall w \in W. V_K(\alpha, w) = 1$.
\end{frame}

\begin{frame}
\frametitle{T, S4, S5: Teorema di adeguatezza}
Per tutti e tre i sistemi vale il teorema di adeguatezza che afferma:

Per ogni formula $\alpha \in L$:
$$\vdash \alpha \Rightarrow \vDash \alpha$$
\end{frame}

\begin{frame}
\frametitle{Collasso nel calcolo proposizionale}
In nessun sistema è dimostrabile la formula $p \rightarrow \Box p$

quindi non è un teorema la formula $p \leftrightarrow \Box p$ in nessun sistema.
\end{frame}

\begin{frame}
\frametitle{Collasso nel calcolo proposizionale}
Basta costruire un $S5$-modello come segue:
\begin{itemize}
    \item $W = \{w_1, w_2\}$;
    \item $R = W x W$;
    \item $I(1, w_1) = \begin{cases}
                      1 & \mbox{se } x = p \mbox{ e } w = w_1 \\
                      0 & \mbox{altrimenti}
    \end{cases}$
\end{itemize}
\end{frame}

\begin{frame}
\frametitle{Sistema T: Procedura di decisione}
Data $\alpha \in L$, cerchiamo un T-modello che falsifichi $\alpha$, ovvero un T-modello $K$
in cui esista un mondo $w$ tale che: $$V_K(\alpha, w) = 0$$.

Rappresentiamo graficamente questo procedimento con dei diagrammi detti T-diagrammi
\end{frame}


\begin{frame}
\frametitle{Sistema T: Procedura di decisione}
\begin{center}
    \begin{tikzpicture}[
    world/.style={rectangle,draw},
    invisible/.style={rectangle}]
        \node[world] (w_1)
            {$\underset{ }{\underset{ }{\diamond}} (p \land \diamond q)
              \underset{0}{\rightarrow}
              (\stackrel{ }{\underset{ }{\Box}} \underset{ }{\underset{ }{\diamond}} p
               \underset{ }{\rightarrow} \stackrel{ }{\underset{ }{\diamond}} \underset{ }{\underset{ }{\Box}} q)$
            };

        \node[black, left] at (w_1.west) {$w_1$};


    \end{tikzpicture}
\end{center}
\end{frame}

\begin{frame}
\frametitle{Sistema T: Procedura di decisione}
\begin{center}
    \begin{tikzpicture}[
    world/.style={rectangle,draw},
    invisible/.style={rectangle}]
        \node[world] (w_1)
            {$\underset{*}{\underset{1}{\diamond}} (p \land \diamond q)
              \underset{0}{\rightarrow}
              (\stackrel{*}{\underset{1}{\Box}} \underset{ }{\underset{ }{\diamond}} p
               \underset{0}{\rightarrow} \stackrel{*}{\underset{0}{\diamond}} \underset{ }{\underset{ }{\Box}} q)$
            };

        \node[black, left] at (w_1.west) {$w_1$};


    \end{tikzpicture}
\end{center}
\end{frame}

\begin{frame}
\frametitle{Sistema T: Procedura di decisione}
\begin{center}
    \begin{tikzpicture}[
    world/.style={rectangle,draw},
    invisible/.style={rectangle}]
        \node[world] (w_1)
            {$\underset{*}{\underset{1}{\diamond}} (p \land \diamond q)
              \underset{0}{\rightarrow}
              (\stackrel{*}{\underset{1}{\Box}} \underset{*}{\underset{1}{\diamond}} p
               \underset{0}{\rightarrow} \stackrel{*}{\underset{0}{\diamond}} \underset{*}{\underset{0}{\Box}} q)$
            };

        \node[black, left] at (w_1.west) {$w_1$};


    \end{tikzpicture}
\end{center}
\end{frame}


\begin{frame}
\frametitle{Sistema T: Procedura di decisione}
   \scalebox{0.75}{
    \begin{tikzpicture}[
    world/.style={rectangle,draw},
    invisible/.style={rectangle}]
        \node[world] (w_1)
            {$\underset{*}{\underset{1}{\diamond}} (p \land \diamond q)
              \underset{0}{\rightarrow}
              (\stackrel{*}{\underset{1}{\Box}} \underset{*}{\underset{1}{\diamond}} p
               \underset{0}{\rightarrow} \stackrel{*}{\underset{0}{\diamond}} \underset{*}{\underset{0}{\Box}} q)$
            };

        \node[black, left] at (w_1.west) {$w_1$};

        \node[world] (w_2) [below left=of w_1]
              {$\begin{array}{c|c|c}
                  \underset{1}{p} \underset{1}{\land} \underset{*}{\underset{1}{\diamond}} q &
                  \underset{1}{\diamond} \underset{1}{p} &
                  \underset{*}{\underset{0}{\Box}} q
              \end{array}$};

        \node[black, left] at (w_2.west) {$w_2$};

        \node[world] (w_3) [below=of w_1]
              {$\begin{array}{c|c|c}
                  \underset{1}{p} &
                  \underset{1}{\diamond}\underset{1}{p} &
                  \underset{*}{\underset{0}{\Box}} q
              \end{array}$};

        \node[black, right] at (w_3.east) {$w_3$};


        \node[world] (w_4) [below right=of w_1]
            {$\begin{array}{c|c|c}
                \underset{0}{q} &
                \underset{*}{\underset{1}{\diamond}} p &
                \underset{0}{\Box}\underset{0}{q}
            \end{array}$};

        \node[black, left] at (w_4.west) {$w_4$};



        \draw[->] (w_1.south west) -- (w_2);
        \draw[->] (w_1.south) -- (w_3);
        \draw[->] (w_1.south east) -- (w_4);

    \end{tikzpicture}
    }
\end{frame}
\begin{frame}
\frametitle{Sistema T: Procedura di decisione}
   \scalebox{0.75}{
    \begin{tikzpicture}[
    world/.style={rectangle,draw},
    invisible/.style={rectangle}]
        \node[world] (w_1)
            {$\underset{*}{\underset{1}{\diamond}} (p \land \diamond q)
              \underset{0}{\rightarrow}
              (\stackrel{*}{\underset{1}{\Box}} \underset{*}{\underset{1}{\diamond}} p
               \underset{0}{\rightarrow} \stackrel{*}{\underset{0}{\diamond}} \underset{*}{\underset{0}{\Box}} q)$
            };

        \node[black, left] at (w_1.west) {$w_1$};

        \node[world] (w_2) [below left=of w_1]
              {$\begin{array}{c|c|c}
                  \underset{1}{p} \underset{1}{\land} \underset{*}{\underset{1}{\diamond}} q &
                  \underset{1}{\diamond} \underset{1}{p} &
                  \underset{*}{\underset{0}{\Box}} q
              \end{array}$};

        \node[black, left] at (w_2.west) {$w_2$};

        \node[world] (w_3) [below=of w_1]
              {$\begin{array}{c|c|c}
                  \underset{1}{p} &
                  \underset{1}{\diamond}\underset{1}{p} &
                  \underset{*}{\underset{0}{\Box}} q
              \end{array}$};

        \node[black, right] at (w_3.east) {$w_3$};


        \node[world] (w_4) [below right=of w_1]
            {$\begin{array}{c|c|c}
                \underset{0}{q} &
                \underset{*}{\underset{1}{\diamond}} p &
                \underset{0}{\Box}\underset{0}{q}
            \end{array}$};

        \node[black, left] at (w_4.west) {$w_4$};


        \node[world] (w_5) [below left=of w_2]
              {$\underset{1}{q}$};

        \node[black, right] at (w_5.east) {$w_5$};

        \node[world] (w_6) [below=of w_2]
              {$\underset{0}{q}$};

        \node[black, right] at (w_6.east) {$w_6$};

        \node[world] (w_7) [below=of w_3]
             {$\underset{0}{q}$};

        \node[black, left] at (w_7.west) {$w_7$};

        \node[world] (w_8) [below=of w_4]
             {$\underset{1}{p}$};

        \node[black, left] at (w_8.west) {$w_8$};

        \draw[->] (w_1.south west) -- (w_2);
        \draw[->] (w_1.south) -- (w_3);
        \draw[->] (w_1.south east) -- (w_4);

        \draw[->] (w_2) -- (w_5);
        \draw[->] (w_2) -- (w_6);

        \draw[->] (w_3) -- (w_7);

        \draw[->] (w_4) -- (w_8);



    \end{tikzpicture}
    }
\end{frame}


\begin{frame}
\frametitle{Sistema T: Procedura di decisione II}
\begin{center}
    \begin{tikzpicture}[
    world/.style={rectangle,draw},
    invisible/.style={rectangle}]
        \node[world] (w_1)
            { $\Box p \underset{ }{\underset{0}{\leftrightarrow}} \neg \diamond \neg p$
            };

        \node[black, left] at (w_1.west) {$w_1$};

    \end{tikzpicture}
\end{center}
\end{frame}

\begin{frame}
\frametitle{Sistema T: Procedura di decisione II}
    \begin{tikzpicture}[
    world/.style={rectangle,draw},
    invisible/.style={rectangle}]
        \node[world] (w_1)
            { $\Box p \underset{\dagger}{\underset{0}{\leftrightarrow}} \neg \diamond \neg p$
            };

        \node[black, left] at (w_1.west) {$w_1$};

        \node[world] (w_1') [below left=of w_1]
            { $\stackrel{*}{\underset{1}{\Box} \underset{1}{p}} \underset{0}{\leftrightarrow} \underset{0}{\neg} \underset{*}{\underset{1}{\diamond}} \neg p$
            };

        \node[black, left] at (w_1'.west) {$w_1(1)$};

        \node[world] (w_2) [below=of w_1']
            { $\begin{array}{c|c}
                \underline{\underset{1}{\neg}\underset{0}{p}} &
                \underset{1}{p}
            \end{array}$
            };

        \node[black, right] at (w_2.east) {$w_2$};

        \node[world] (w_1'') [below right=of w_1]
            { $\underset{*}{\underset{0}{\Box}} p \underset{0}{\leftrightarrow} \underset{1}{\neg} \stackrel{*}{\underset{0}{\diamond}} \underset{0}{\neg} \underset{1}{p}$
            };

        \node[black, left] at (w_1''.west) {$w_1(2)$};

        \node[world] (w_3) [below=of w_1'']
            { $\begin{array}{c|c}
                \underline{\underset{0}{p}} &
                \underset{0}{\neg}\underset{1}{p}
            \end{array}$
            };

        \node[black, right] at (w_3.east) {$w_3$};

        \draw[->] (w_1') -- (w_2);
        \draw[->] (w_1'') -- (w_3);

    \end{tikzpicture}
\end{frame}

\begin{frame}
\frametitle{Sistema T: Procedura di decisione III}
Data una formula $\alpha \in L$, una volta seguita tutta la procedura
diciamo che abbiamo ottenuto un sistema completo di T-diagrammi per $\alpha$.
\end{frame}


\begin{frame}
\frametitle{Sistema T: Teorema di completezza}
Data una formula $\alpha \in L$ costruiamo il suo sistema completo di T-diagrammi.
Ad ogni rettangolo $w_i$ associamo la formula:
$$w_i' := \gamma \lor \neg \beta_1 \lor \ldots \lor \neg \beta_n$$
\end{frame}

\begin{frame}
\frametitle{Sistema T: Teorema di completezza}
\begin{block}{Lemma 1}
Dato un sistema completo di T-diagrammi e sia $w_i$ un suo rettangolo,
se $\beta$ è una sottoformula di $w_i'$ e a $\beta$ è assegnato univocamente il valore di verità falso
nel rettangolo $w_i$,

allora $\vdash \beta \rightarrow w_i'$ in T

Se invece a $\beta$ è associato il valore di verità vero,


allora $\vdash \neg \beta \rightarrow w_i'$ in T
\end{block}
\end{frame}

\begin{frame}
\frametitle{Sistema T: Teorema di completezza}
\begin{block}{Lemma 2}
Dato un sistema completo di T-diagrammi e sia $w_i$ un suo rettangolo in cui
non ci sono operatori contrassegnati con $\dagger$,
se in $w_i$ c'è un'inconsistenza, allora $\vdash w_i'$ in T.
\end{block}
\end{frame}

\begin{frame}
\frametitle{Sistema T: Teorema di completezza}
\begin{block}{Lemma 3}
Dato un sistema completo di T-diagrammi e sia $w_i$ un suo rettangolo in cui
non ci sono operatori contrassegnati con $\dagger$,
se in $w_i$ c'è un'inconsistenza, allora $\vdash w_i'$ in T.
\end{block}
\end{frame}

\begin{frame}
\frametitle{Sistema T: Teorema di completezza}

\begin{block}{Lemma 4}
Se nel rettangolo $w_i$ c'è un operatore contrassegnato con $\dagger$
e $w_i(1), w_i(2), w_i(3)$ sono i suoi rettangoli alternativi:
$$\vdash w_i(1)', \vdash w_i(2)', \vdash w_i(3)' \rightarrow \vdash w_i'$$
\end{block}
\end{frame}


\begin{frame}
\frametitle{Sistema T: Teorema di completezza}
    \begin{tikzpicture}[
    world/.style={rectangle,draw},
    invisible/.style={rectangle}]
        \node[world] (w_1)
            { $\Box p \underset{\dagger}{\underset{0}{\leftrightarrow}} \neg \diamond \neg p$
            };

        \node[black, left] at (w_1.west) {$w_1$};

        \node[world] (w_1') [below left=of w_1]
            { $\stackrel{*}{\underset{1}{\Box} \underset{1}{p}} \underset{0}{\leftrightarrow} \underset{0}{\neg} \underset{*}{\underset{1}{\diamond}} \neg p$
            };

        \node[black, left] at (w_1'.west) {$w_1(1)$};

        \node[world] (w_2) [below=of w_1']
            { $\begin{array}{c|c}
                \underline{\underset{1}{\neg}\underset{0}{p}} &
                \underset{1}{p}
            \end{array}$
            };

        \node[black, right] at (w_2.east) {$w_2$};

        \node[world] (w_1'') [below right=of w_1]
            { $\underset{*}{\underset{0}{\Box}} p \underset{0}{\leftrightarrow} \underset{1}{\neg} \stackrel{*}{\underset{0}{\diamond}} \underset{0}{\neg} \underset{1}{p}$
            };

        \node[black, left] at (w_1''.west) {$w_1(2)$};

        \node[world] (w_3) [below=of w_1'']
            { $\begin{array}{c|c}
                \underline{\underset{0}{p}} &
                \underset{0}{\neg}\underset{1}{p}
            \end{array}$
            };

        \node[black, right] at (w_3.east) {$w_3$};

        \draw[->] (w_1') -- (w_2);
        \draw[->] (w_1'') -- (w_3);

    \end{tikzpicture}
\end{frame}

\begin{frame}
\frametitle{Sistema T: Teorema di completezza}
Se $\alpha \in L$ e $\vDash_T \alpha$, nel sistema
completo di T-diagrammi di $\alpha$ c'è sicuramente un'inconsistenza.
Sfruttando i lemmi che abbiamo enunciato poc'anzi si riesce a mostrare
che $w_1'$ è un teorema, ma per costruzione $w_1'$ è $\alpha$.
Quindi otteniamo:
$$\vdash_T \alpha \Leftrightarrow \vDash_T \alpha$$
\end{frame}


\begin{frame}
Grazie per l'attenzione.
\end{frame}

\end{document}
