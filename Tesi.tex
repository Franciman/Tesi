\documentclass[a4paper, 12pt]{article}
\usepackage{latexsym}
\newtheorem{theorem}{Teorema}
\newtheorem{lemma}{Lemma}

% Ambiente per le dimostrazioni
\newenvironment{proof}
    {\textit{Dim.}
    }
    {\begin{flushright}$\bullet$\end{flushright}
    }

\begin{document}
La logica modale è un'estensione della logica classica.
Studieremo la validità di argomenti che comprendono le modalità.
Le modalità sono modi in cui si può affermare che una proposizione è vera in certe circostanze.
Ad esempio, ``è possibile che $\alpha$'' significa che $\alpha$ non è detto sia vera, ma nulla vieta che non lo sia.
Quindi estendiamo il numero di proposizioni su cui possiamo dare un giudizio circa la loro validità.
Di conseguenza partiremo da un sistema formale per il calcolo proposizionale, che chiameremo $CP$
e definiamo una sua estensione\footnote{Definire cos'è un'estensione, à la Tortora} $CM$ \footnote{Calcolo Modale? Un po' bruttino}

\section{Definizione del linguaggio di $CM$}
Usiamo come base quella di $CP$. Una sua definizione la si può trovare in [Tortora].
Aggiungiamo un nuovo operatore monadico\footnote{decidere se è più bello monadico o unario}
che indicheremo con il simbolo $\Box$.

\section{Definizione dell'apparato deduttivo per $CM$}
Aggiungiamo anche una nuova regola di deduzione, chiamata necessitazione:

$\alpha \vdash \Box \alpha$

L'operatore che vogliamo aggiungere, ha una natura non verofunzionale. Vale a dirsi,
che la verità o falsità della proposizione $\Box \alpha$ non dipende solo
dal valore di verità di $\alpha$.

Per quanto riguarda gli assiomi, la situazione si complica. Infatti sono stati sviluppati moltissimi sistemi di logica modale,
ognuno che rispecchiasse una certa visione delle proprietà della modalità che si considera.
Tutti ereditano da $CP$ i suoi assiomi, e ne aggiungono di nuovi.
Noi ci occuperemo, in particolare di tre tipi di sistemi formali:
\begin{itemize}
\item Sistema $T$
\item Sistema $S4$
\item Sistema $S5$
\end{itemize}

Tali sistemi sono in relazione tra di loro, infatti, il Sistema $T$ è contenuto in $S4$ che è contenuto in $S5$.
Tutti i sistemi hanno lo stesso linguaggio e le stesse regole di deduzione. Ci\`o che cambia sono gli assiomi.

Partiamo con l'analizzare gli assiomi del sistema pi\`u debole, il sistema $T$.

\section{Il Sistema $T$}
Il Sistema $T$ aggiunge i seguenti assiomi:
\begin{itemize}
\item $\Box \alpha \rightarrow \alpha$ (Assioma di necessit\`a)
\item $\Box (\alpha \rightarrow \beta) \rightarrow (\Box \alpha \rightarrow \Box \beta)$
\end{itemize}

\section{Il Sistema $S4$}
Il Sistema $S4$ aggiunge al Sistema $T$ il seguente assioma, detto assioma $S4$:

$\Box \alpha \rightarrow \Box \Box \alpha$

\section{Il Sistema $S5$}
Il Sistema $S5$ aggiunge al Sistema $T$ il seguente assioma, detto assioma $S5$:

$\Diamond \alpha \rightarrow \Box \Diamond \alpha$

\begin{lemma}
$\vdash \Diamond \Box x \rightarrow \Box x$ in $S5$
\end{lemma}
\begin{proof}

\begin{tabular} { c c|c }
(1) & $\Diamond \neg x \rightarrow \Box \Diamond \neg x$ & sostituzione di $\neg x$ in assioma $S5$ \\
(2) & $\neg \Box \Diamond \neg x \rightarrow \neg \Diamond \neg x$ & contrapposta di \\
(3) & $\Diamond \neg \Diamond \neg x \rightarrow \neg \Diamond \neg x$ & definizione di $\Diamond$ \\
(4) & $\Diamond \Box x \rightarrow \Box x$ & definizione di $\Diamond$ e $\Box$
\end{tabular}
\end{proof}

\begin{theorem}
$T + S5 \rightarrow T + S4$, ovvero Il sistema $S5$ contiene il sistema $S4$
\end{theorem}
\begin{proof}
Lasciato come esercizio al lettore
\end{proof}

\section{Modalit\`a e funzioni modali}
DA SCRIVERE

\end{document}
