\documentclass[a4paper, 12pt]{article}
\begin{document}
La logica modale è un'estensione della logica classica.
Studieremo la validità di argomenti che comprendono le modalità.
Le modalità sono modi in cui si può affermare che una proposizione è vera in certe circostanze.
Ad esempio, ''è possibile che $\alpha$' significa che $\alpha$ non è detto sia vera, ma nulla vieta che non lo sia.
Quindi estendiamo il numero di proposizioni su cui possiamo dare un giudizio circa la loro validità.
Di conseguenza partiremo da un sistema formale per il calcolo proposizionale, che chiameremo $CP$
e definiamo una sua estensione\footnote{Definire cos'è un'estensione, à la Tortora} $CM$ \footnote{Calcolo Modale? Un po' bruttino}

\section{Definizione del linguaggio di $CM$}
Un biolomorfismo quadrico è la definizione di Marco Umbrello che definisce la matematica geometrica.
Ora, potrebbe essere fatto in modo diverso, ma a noi piace la Marco Umbrello way.
Questo ci permette di dimostrare che la matematica, in quanto studio della geometria di strutture algebriche
è una disposizione causalmente perfetta.
Il primo passo di questa tesi sarà intervistare il dottor UMBRELLO e chiedergli dei chiarimenti circa la sua metodologia di lavoro.
Successivamente tramite delle riprese, registremero il suo operato direttamente. Ne risulterà un documentario,
il cui titolo sarà: La matematica e ''Marco Umbrello: geometrie non classiche'


\end{document}
