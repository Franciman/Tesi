\documentclass[a4paper, 12pt]{article}
\usepackage[utf8]{inputenc}
\usepackage{latexsym}
\newtheorem{axiom}{Assioma}
\newtheorem{theorem}{Teorema}
\newtheorem{lemma}{Lemma}
\newtheorem{definition}{Definizione}

% Ambiente per le dimostrazioni
\newenvironment{proof}
    {\textit{Dim.}
    }
    {\begin{flushright}$\bullet$\end{flushright}
    }

% Ambiente per le dimostrazioni formali formali
\newenvironment{formal_proof}
    {\begin{proof}

    % \begin{tabular} {c c|c}
    }
    {%\end{tabular}

    \end{proof}
    }

\begin{document}

\section{Logica proposizionale}
In questo paragrafo introdurremo brevemente un sistema formale per la logica proposizionale classica.
Il sistema che introduciamo sarà usato come base per la costruzione di ulteriori sistemi formali.
Esso sarà molto minimale, infatti conterrà pochi connettivi di base
e tutti gli altri saranno definiti successivamente a partire da questi.

Il primo passaggio è quello di definire un alfabeto. Usiamo una piccola scorciatoia,
infatti supponiamo che ci sia dato a priori un insieme numerabile A i cui elementi saranno detti atomi.
Nella pratica questo insieme conterrà i simboli da usare per le variabili che appaiono nelle formule.

\begin{definition}
L'alfabeto $\Sigma$ è dato dall'insieme $A \cup \{\neg, \vee, (, ) \}$
\end{definition}

Passiamo ora alla definizione dell'insieme delle formule ben formate.

\begin{definition}
L'insieme delle formule ben formate (fbf) $L$ è il più piccolo sottoinsieme dell'insieme tutte le successioni
di elementi di $\Sigma$ determinato dalla seguente definizione induttiva:
\begin{itemize}
\item Se $p \in A$, allora $p \in L$;
\item Se $\alpha \in L$, allora $\neg (\alpha) \in L$;
\item Se $\alpha \in L$ e $\beta \in L$, allora $(\alpha) \vee (\beta) \in L$;
\item $L$ non contiene altri elementi.
\end{itemize}
\end{definition}

Possiamo ricavare, ora, i connettivi che mancano come segue:
\begin{definition}
Se $\alpha, \beta \in L$,
$$(\alpha) \rightarrow (\beta) := (\neg (\alpha)) \vee (\beta)$$
$$(\alpha) \wedge (\beta) := \neg ((\neg (\alpha)) \vee (\neg (\beta)))$$
\end{definition}

Per semplificare la scrittura e rendere più comprensibili le formule che scriveremo
nel seguito, spesso ometteremo le parentesi e per interpretare correttamente
le formule scritte si useranno le seguenti regole di precedenza tra operatori
e le regole di associatività.

\begin{definition}
Gli operatori binari $\wedge,\vee$ sono associativi a sinistra,
mentre $\rightarrow$ è associativo a destra.
Per determinare l'ordine di precedenza tra due operatori si fa uso della
seguente tabella, usando il seguente criterio:
``Gli operatori che si trovano più in alto hanno precedenza maggiore rispetto
agli operatori che si trovano più in basso''
\begin{center}
\begin{tabular} {c c}
    $\neg$ \\
    $\wedge$ \\
    $\vee$ \\
    $\rightarrow$ & $\leftrightarrow$
\end{tabular}
\end{center}
\end{definition}

Di conseguenza la formula $\alpha \rightarrow \beta \rightarrow \neg \gamma$
dovrà essere interpretata come:
$(\alpha) \rightarrow ((\beta) \rightarrow (\neg (\gamma)))$.

Passiamo infine alla determinazione di un sottoinsieme di formule ben formate
che fungeranno da assiomi e delle regole di deduzione
per poter così definire l'apparato deduttivo del calcolo proposizionale.

\begin{definition}
Gli assiomi del calcolo proposizionale sono le seguenti formule ben formate:
\begin{itemize}
\item (HPD) $\alpha \rightarrow (\beta \rightarrow \alpha)$
\item (HPMP) $(\gamma \rightarrow (\alpha \rightarrow \beta)) \rightarrow (\gamma \rightarrow \alpha) \rightarrow (\gamma \rightarrow \beta)$
\item ($\vee$-I1) $\alpha \rightarrow (\alpha \vee \beta)$
\item ($\vee$-I2) $\beta \rightarrow (\alpha \vee \beta)$
\item ($\vee$-E) $(\alpha \rightarrow \gamma) \rightarrow (\beta \rightarrow \gamma) \rightarrow (\alpha \vee \beta \rightarrow \gamma)$
\item ($\wedge$-I) $\alpha \rightarrow \beta \rightarrow \alpha \wedge \beta$
\item ($\wedge$-E1) $\alpha \wedge \beta \rightarrow \alpha$
\item ($\wedge$-E2) $\alpha \wedge \beta \rightarrow \beta$
\item ($\neg$-I) $(\alpha \rightarrow \beta) \wedge (\alpha \rightarrow \neg \beta) \rightarrow \neg \alpha$
\item (TER) $\alpha \vee \neg \alpha$
\end{itemize}
\end{definition}

La logica modale è un'estensione della logica classica.
Studieremo la validità di argomenti che comprendono le modalità.
Le modalità sono modi in cui si può affermare che una proposizione è vera in certe circostanze.
Ad esempio, ``è possibile che $\alpha$'' significa che $\alpha$ non è detto sia vera, ma nulla vieta che non lo sia.
Quindi estendiamo il numero di proposizioni su cui possiamo dare un giudizio circa la loro validità.
Di conseguenza partiremo da un sistema formale per il calcolo proposizionale, che chiameremo $CP$
e definiamo una sua estensione\footnote{Definire cos'è un'estensione, à la Tortora} $CM$ \footnote{Calcolo Modale? Un po' bruttino}

\section{Definizione del linguaggio di $CM$}
Usiamo come base quella di $CP$. Una sua definizione la si può trovare in [Tortora].
Aggiungiamo un nuovo operatore monadico\footnote{decidere se è più bello monadico o unario}
che indicheremo con il simbolo $\Box$.

\section{Definizione dell'apparato deduttivo per $CM$}
Aggiungiamo anche una nuova regola di deduzione, chiamata necessitazione:

$$x \vdash \Box x$$

L'operatore che vogliamo aggiungere, ha una natura non verofunzionale. Vale a dirsi,
che la verità o falsità della proposizione $\Box \alpha$ non dipende solo
dal valore di verità di $\alpha$.

Per quanto riguarda gli assiomi, la situazione si complica. Infatti sono stati sviluppati moltissimi sistemi di logica modale,
ognuno che rispecchiasse una certa visione delle proprietà della modalità che si considera.
Tutti ereditano da $CP$ i suoi assiomi, e ne aggiungono di nuovi.
Noi ci occuperemo, in particolare di tre tipi di sistemi formali:
\begin{itemize}
\item Sistema $T$
\item Sistema $S4$
\item Sistema $S5$
\end{itemize}

Tali sistemi sono in relazione tra di loro, infatti, il Sistema $T$ è contenuto in $S4$ che è contenuto in $S5$.
Tutti i sistemi hanno lo stesso linguaggio e le stesse regole di deduzione. Ciò che cambia sono gli assiomi.

Partiamo con l'analizzare gli assiomi del sistema più debole, il sistema $T$.

\section{Il Sistema $T$}
Il Sistema $T$ aggiunge i seguenti assiomi:
\begin{itemize}
\item \textbf{Assioma 1} $\Box \alpha \rightarrow \alpha$ (Assioma di necessità)
\item \textbf{Assioma 2} $\Box (\alpha \rightarrow \beta) \rightarrow (\Box \alpha \rightarrow \Box \beta)$
\end{itemize}

\section{Il Sistema $S4$}
Il Sistema $S4$ aggiunge al Sistema $T$ il seguente assioma, detto assioma $S4$:

\textbf{Assioma 3} $\Box \alpha \rightarrow \Box \Box \alpha$

\section{Il Sistema $S5$}
Il Sistema $S5$ aggiunge al Sistema $T$ il seguente assioma, detto assioma $S5$:

\textbf{Assioma 4} $\diamond \alpha \rightarrow \Box \diamond \alpha$

\begin{flushleft}
\textbf{Lemma 1}
$\vdash \diamond \Box x \rightarrow \Box x$ in $S5$

\textit{Dim.}

(1) $\diamond \neg x \rightarrow \Box \diamond \neg x$ ; sostituzione di $\neg x$ in assioma $S5$ \\
(2) $\neg \Box \diamond \neg x \rightarrow \neg \diamond \neg x$ ; contrapposta di (1) \\
(3) $\diamond \neg \diamond \neg x \rightarrow \neg \diamond \neg x$ ; definizione di $\diamond$ \\
(4) $\diamond \Box x \rightarrow \Box x$ ; definizione di $\diamond$ e $\Box$

\begin{flushright}
$\bullet$
\end{flushright}
\end{flushleft}


\begin{flushleft}
\textbf{Teorema 1}
$T + S5 \rightarrow T + S4$, ovvero Il sistema $S5$ contiene il sistema $S4$

\textit{Dim.}

(1) $\Box x \rightarrow \diamond \Box x$ ; Necessitazione per possibilità \\
(2) $\Box x \rightarrow \Box \diamond \Box x$ ; Sillogismo di (1) e $S5$ \\
(3) $\diamond \Box x \rightarrow \Box x$ ; Lemma 1 \\
(4) $\Box (\diamond \Box x \rightarrow \Box x)$ ; Necessitazione di (3) \\
(5) $\Box \diamond \Box x \rightarrow \Box \Box x)$ ; Assioma 2 \\
(6) $\Box x \rightarrow \Box \Box x$ ; Sillogismo di (1) e (5)

\begin{flushright}$\bullet$\end{flushright}
\end{flushleft}

\section{Un teorema di deduzione}
In questo paragrafo proviamo a derivare un teorema di deduzione per il Sistema $S4$.
Non possiamo aspettarci che il teorema di deduzione per il calcolo proposizionale $CP$ valga,
infatti la regola di deduzione di necessitazione ci porterebbe ad asserire che:
$$ x \vdash \Box x \Rightarrow \vdash x \rightarrow \Box x $$
E ciò vorrebbe dire che le modalità non aggiungono nulla al sistema del calcolo proposizionale.

Se guardiamo attentamente la regola di necessitazione e la scriviamo in questo modo:
$$x \vdash \Box x$$
possiamo notare che l'ipotesi $x$ è sufficiente per dimostrare $\Box x$ che afferma molto più
che semplicemente $x$.
Notiamo anche che la regola di necessitazione ci permette di provare
$$x \vdash \Box \Box x$$

Quindi, far semplicemente passare la x da sinistra a destra, significherebbe affermare che
da ipotesi più deboli si possono comunque dedurre le stesse conclusioni. Allora possiamo provare,
nello spostare la x da sinistra a destra di $\vdash$, ad aggiungere un $\Box$.
Questo può bastare in sistemi come $S4$ o più potenti, perché da $\Box x$ possiamo dedurre
$\Box \Box x$, $\Box \Box \Box x$, ecc...
Mentre nel sistema T questo non è vero e quindi dovremo risolvere diversamente.

Recuperiamo quindi un teorema di deduzione, facendo la seguente modifica:
$$\alpha \vdash \beta \Rightarrow \vdash \Box \alpha \rightarrow \beta$$
Ora il precedente problema è facilmente risolvibile, infatti, la deduzione $x \vdash \Box x$
diventa il teorema $\vdash \Box x \rightarrow \Box x$ banalmente valido.
Otteniamo però anche il seguente teorema, $\vdash \Box x \rightarrow \Box \Box x$ che è l'assioma $S4$
Ecco perché abbiamo supposto di trovarci nel sistema $S4$.

Diamo una dimostrazione di

\begin{flushleft}
\textbf{Teorema}
$\alpha \vdash \beta \Leftrightarrow \vdash \Box \alpha \rightarrow \beta$ in $S4$

\textit{Dim.}
Lasciata al lettore, è una banale modifica della dimostrazione nel caso del calcolo proposizionale.
L'unica parte interessante è quella riguardante i teoremi ottenuti con la regola di necessitazione.
In quel caso dobbiamo far uso dell'assioma $S4$ ed è tutto banale.
\end{flushleft}



\section{Modalità e funzioni modali}
Rendiamo più formale il concetto di modalità.
\textbf{Definizione}
Una modalità è una successione di 0 o più di uno dei seguenti simboli: $\neg,\diamond,\Box$.

Viene da domandarci, ora, quante modalità ci sono in ognuno dei sistemi che abbiamo considerato.
Si può dimostrare che nel Sistema $T$ ce ne sono infinite,
mentre in $S4$ e $S5$ ce ne sono solo un numero finito.

Parleremo più approfonditamente di tali questioni solo dopo aver introdotto una semantica
per il nostro linguaggio.

\begin{flushleft}
\textbf{Definizione 1}

Una funzione modale è una formula contenente almeno una funzione modale
\end{flushleft}

In $T$ ovviamente c'è un numero infinito di funzioni modali esprimibili.
In $S5$ ce n'è un numero finito, mentre sorprendentemente,
in $S4$ ce ne sono infinite, nonostante le modalità esprimibili siano in numero finito.

\section{Un sistema interessante}
Dedichiamo questo paragrafo un sistema di logica modale particolare, chiamato Sistema $B$.
Esso è intimamente collegato alla logica intuizionista, come si evince dal nome,
infatti la B è la iniziale di Brouwer, logico padre dell'intuizionismo.

\section{Una semantica per i sistemi modali}
Seguendo la tradizione, proviamo a dare una semantica per i sistemi formali sopra definiti.
La costruzione che presenteremo è dovuta a Kripke e Hintikka ed altri.

\end{document}
