\documentclass[a4paper, 12pt]{article}
\usepackage[utf8]{inputenc}
\usepackage{amssymb}
\newtheorem{axiom}{Assioma}
\newtheorem{theorem}{Teorema}
\newtheorem{lemma}{Lemma}
\newtheorem{definition}{Definizione}

\title{I sistemi di logica modale T, S4, S5: Sintassi e Semantica}
\date{}

\linespread{1.5}

% Ambiente per le dimostrazioni
\newenvironment{proof}
    {\textit{Dim.}
    }
    {\begin{flushright}$\bullet$\end{flushright}
    }

% Ambiente per le dimostrazioni formali formali
\newenvironment{formal_proof}
    {\begin{proof}

    % \begin{tabular} {c c|c}
    }
    {%\end{tabular}

    \end{proof}
    }

\begin{document}
\maketitle
\section{Nozioni Preliminari}
Premettiamo allo studio di sistemi di logica modale la definizione
di sintassi e semantica di un sistema formale per il calcolo proposizionale,
che da ora in poi denoteremo con $CP$, da usare come base per la definizione
dei sistemi formali successivi.
È possibile presentare un tale sistema in vari modi, tutti pressoché equivalenti tra di loro.
Noi seguiremo seguire la strada di [Hughes-Cresswell], definendo un linguaggio con un numero ristretto
di connettivi primitivi e introducendo successivamente i connettivi mancanti definendoli a partire
dai connettivi primitivi.
Questa scelta ci permette di usare tutti i connettivi che siamo soliti adoperare,
rendendo quindi la scrittura delle formule più chiara, ma allo stesso tempo ci permette anche di
considerare solo i connettivi primitivi nello studio delle proprietà del sistema,
semplificandoci il lavoro.

\subsection{Il Linguaggio}
Per definire il linguaggio di $CP$ avremo bisogno di un insieme di oggetti di base,
chiamati variabili proposizionali. Senza soffermarci ulteriormente sulla natura di questi oggetti,
possiamo tranquillamente assumere che esista un insieme $A$ numberabile
disgiunto da $\{\neg, \lor, (, )\}$.

\begin{definition}
Chiamiamo variabile proposizionale ogni elemento di $A$.
\end{definition}

Scegliamo come connettivi primitivi $\neg$ e $\lor$,
a partire da essi sarà possibile definire tutti gli altri.
L'alfabeto di $CP$ è dunque dato dall'insieme $\Sigma = A \cup \{\neg, \lor, (, )\}$.

Chiamiamo $\Sigma^{*} = \bigcup_{n \in A} \Sigma^n$ l'insieme delle espressioni
e definiamo l'insieme delle formule ben formate (fbf) $L_{CP} \subseteq \Sigma^{*}$ con la seguente definizione
per induzione:
\begin{itemize}
\item $x \in A \Rightarrow x \in L_{CP}$;
\item $\alpha \in L_{CP} \Rightarrow \neg (\alpha) \in L_{CP}$;
\item $\alpha \in L_{CP}$ e $\beta \in L_{CP} \Rightarrow (\alpha) \lor (\beta) \in L_{CP}$;
\item Non ci sono altri elementi in $L_{CP}$.
\end{itemize}

Una volta definito il linguaggio, definiamo i connettivi che mancano a partire da $\neg$ e $\lor$
come segue:
\begin{definition}
Se $\alpha \in L_{CP}$ e $\beta \in L_{CP}$, allora:
\begin{itemize}
\item $(\alpha) \land (\beta) := \neg(\neg(\alpha) \lor \neg(\beta))$
\item $(\alpha) \rightarrow (\beta) := (\neg(\alpha)) \lor (\beta)$
\item $(\alpha) \leftrightarrow (\beta) := ((\alpha) \rightarrow (\beta)) \land ((\beta) \rightarrow (\alpha))$
\end{itemize}
\end{definition}

Al fine di rendere le formule che scriveremo più comprensibili, seguendo la prassi,
ometteremo spesso le parentesi usando le convenzioni descritte in [Tortora; pag 46].

\subsection{L'apparato deduttivo}
Per la definizione dell'apparato deduttivo di $CP$ seguiamo quanto fatto in [Tortora].

Vale la pena far notare che questa non è l'unica possibilità, ma ci sono tante altre scelte sia
per quanto riguarda gli assiomi che le regole di deduzione, che sono altrettanto valide
e che portano allo stesso risultato. La motivazione dietro la nostra scelta
risiede semplicemente nel fatto che questo approccio è quello da noi ritenuto più familiare.

\begin{definition}
Gli assiomi di $CP$ sono ottenuti dai seguenti schemi
sostituendo in modo uniforme formule ben formate al posto di $\alpha, \beta, \gamma$:
\begin{itemize}
\item (HPD) $\alpha \rightarrow (\beta \rightarrow \alpha)$
\item (HPMP) $(\gamma \rightarrow (\alpha \rightarrow \beta)) \rightarrow (\gamma \rightarrow \alpha) \rightarrow (\gamma \rightarrow \beta)$
\item ($\lor$-I1) $\alpha \rightarrow (\alpha \lor \beta)$
\item ($\lor$-I2) $\beta \rightarrow (\alpha \lor \beta)$
\item ($\lor$-E) $(\alpha \rightarrow \gamma) \rightarrow (\beta \rightarrow \gamma) \rightarrow (\alpha \lor \beta \rightarrow \gamma)$
\item ($\land$-I) $\alpha \rightarrow \beta \rightarrow \alpha \land \beta$
\item ($\land$-E1) $\alpha \land \beta \rightarrow \alpha$
\item ($\land$-E2) $\alpha \land \beta \rightarrow \beta$
\item ($\neg$-I) $(\alpha \rightarrow \beta) \land (\alpha \rightarrow \neg \beta) \rightarrow \neg \alpha$
\item (TER) $\alpha \lor \neg \alpha$
\end{itemize}
\end{definition}

L'unica regola di deduzione di $CP$ è il \textit{Modus Ponens}:
$$\frac{\alpha \rightarrow \beta \ \ \alpha}{\beta}$$

Notiamo che gli assiomi che abbiamo scelto per $CP$ non sono in numero finito,
ma sono infiniti. L'uso di schemi di assiomi porta a notevoli semplificazioni
nella pratica; infatti ci permette di dimostrare in una sola volta famiglie di teoremi,
senza dover ripetere volta per volta un procedimento che sarebbe essenzialmente sempre uguale.

Diamo ora le seguenti definizioni che introducono i concetti di dimostrazione e di
deduzione, strettamente collegati tra loro.

\begin{definition}
Una dimostrazione è una sequenza finita $D$ di fbf tale che
ogni termine $D_i$ della sequenza soddisfa almeno una delle seguenti condizioni
\begin{itemize}
\item $D_i$ è un assioma;
\item esistono $D_h, D_k$, con $h < i, k < i$, tali che $D_i$ è derivato per \textit{Modus Ponens}
da $D_h$ e $D_k$.
\end{itemize}
\end{definition}

Se $\alpha \in L_{CP}$ ed esiste una dimostrazione $D$ il cui ultimo termine è $\alpha$,
diciamo che esiste una dimostrazione di $\alpha$ e scriviamo $\vdash \alpha$.


\begin{definition}
Se $\Gamma \subseteq L_{CP}$, una deduzione a partire dalle ipotesi $\Gamma$ è una sequenza finita $D$
di fbf tale che ogni termine $D_i$ della sequenza soddisfa almeno una delle seguenti condizioni
\begin{itemize}
\item $D_i \in \Gamma$;
\item $D_i$ è un assioma;
\item esistono $D_h, D_k$, con $h < i, k < i$, tali che $D_i$ è derivato per \textit{Modus Ponens}
da $D_h$ e $D_k$.
\end{itemize}
\end{definition}

Se $\alpha \in L_{CP}$, $\Gamma \subseteq L_{CP}$ ed esiste una deduzione $D$ a partire dalle ipotesi $\Gamma$
il cui ultimo termine è $\alpha$, diciamo che esiste una deduzione di $\alpha$
a partire dalle ipotesi $\Gamma$ e scriviamo $\Gamma \vdash \alpha$.

Nel seguito, scriveremo $\Gamma, \alpha \vdash \beta$
per indicare $\Gamma \cup \{\alpha\} \vdash \beta$.

Vale la pena ricordare il seguente teorema:
\begin{flushleft}
\textbf{Teorema di deduzione}
$$\Gamma, \alpha \vdash \beta \Leftrightarrow \Gamma \vdash \alpha \rightarrow \beta$$
\textit{Dim.} [Tortora]
\begin{flushright}
$\bullet$
\end{flushright}
\end{flushleft}


Terminiamo questo paragrafo mostrando un esempio di dimostrazione di una famiglia
di teoremi, che ci tornerà utile anche nel seguito.

\begin{flushleft}
\textbf{Teorema(Sillogismo)}
Se $\alpha \in L_{CP}$ e $\beta \in L_{CP}$
$$\vdash (\alpha \rightarrow \beta) \rightarrow
        (\beta \rightarrow \gamma) \rightarrow (\alpha \rightarrow \gamma)$$

\textit{Dim.}
Semplifichiamo il lavoro dimostrando
$$(\alpha \rightarrow \beta),(\beta \rightarrow \gamma),\alpha \vdash \gamma$$
in luogo di
$$\vdash (\alpha \rightarrow \beta) \rightarrow
        (\beta \rightarrow \gamma) \rightarrow (\alpha \rightarrow \gamma)$$
e sfruttiamo poi il teorema di deduzione sopra menzionato.

Una possibile deduzione per $\gamma$ a partire dalle ipotesi
$\{(\alpha \rightarrow \beta),(\beta \rightarrow \gamma),\alpha\}$ è la seguente.

(1) $\alpha \rightarrow \beta$ ; ipotesi \\
(2) $\alpha$ ; ipotesi \\
(3) $\beta$ ; Modus Ponens di (1) e (2) \\
(4) $\beta \rightarrow \gamma$ ; ipotesi \\
(5) $\gamma$ ; Modus Ponens di (3) e (4)

Possiamo dunque affermare:
$$(\alpha \rightarrow \beta),(\beta \rightarrow \gamma),\alpha \vdash \gamma$$

Infine, applicando tre volte il teorema di deduzione otteniamo:
$$\vdash (\alpha \rightarrow \beta) \rightarrow (\beta \rightarrow \gamma) \rightarrow (\alpha \rightarrow \gamma)$$

\begin{flushright}
$\bullet$
\end{flushright}
\end{flushleft}

\subsection{Semantica di $CP$}
Ci proponiamo, ora, di fornire una semantica per $CP$.
La semantica che intendiamo fornire per $CP$ è quella standard, in particolare
seguiamo l'esposizione di [Tortora] per presentarla.
L'idea è quella di associare ad ogni formula
un valore di verità, interpretando le variabili proposizionali come proposizioni
che descrivono fatti della \textit{realtà}, ad esempio ``Michele è altisonante'',
il simbolo $\neg$ come il connettivo logico ``non'' dell'italiano
e il simbolo $\lor$ come disgiunzione inclusiva, ossia quel connettivo
logico che in italiano si rappresenta spesso come ``o..., o..., oppure entrambi''.
Per fare ciò dovremmo definire cosa si intende per proposizione,
ma possiamo omettere questo passaggio, a dire il vero un po' problematico,
osservando che, associando ad ogni variabile proposizionale un valore di verità,
siamo in grado di associare un valore di verità ad ogni formula ben formata.
Infatti, il valore di verità di ``non $\alpha$'' dipende solo dal valore di verità
di $\alpha$ e allo stesso modo il valore di verità di
``o $\alpha$, o $\beta$, oppure entrambi``
dipende solo dal valore di verità di $\alpha$ e $\beta$.

Connettivi che hanno questa peculiarità vengono detti operatori vero-funzionali.

Passiamo, ora alla formalizzazione di queste idee e diamo le seguenti definizioni:

\begin{definition}
Chiamiamo interpretazione una funzione $I: A \to \{0, 1\}$
\end{definition}

Data un'interpretazione $I$ definiamo per ricorsione la funzione di valutazione $V_I$ che associa ad ogni formula
del linguaggio di $CP$ un valore dell'insieme $\{0, 1\}$ come segue:

Se $\alpha, \beta \in L_{CP}$:
\begin{itemize}
\item $V_I(p) = I(p)$, $\forall p \in A$
\item $V_I(\alpha \lor \beta) = max\{V_I(\alpha), V_I(\beta)\}$
\item $V_I(\neg \alpha) = 1 - V_I(\alpha)$
\end{itemize}


Quindi un'interpretazione ci dà, in un certo senso, una lista di tutte le proposizioni
vere e di tutte le proposizioni false.
Usiamo, poi, questi valori assegnati dall'interpretazione per determinare il valore di verità delle formule
in base alla semantica che abbiamo assegnato a ciascun connettivo (che sono codificate nelle regole
che abbiamo usato per definire la funzione $V_I$).

\begin{definition}
Sia $\alpha \in L_{CP}$, diremo che $\alpha$ è valida e scriveremo $\vDash \alpha$ se
per ogni interpretazione $I$, $V_I(\alpha) = 1$.
\end{definition}

È noto, e una discussione approfondita di questo argomento si può trovare in [Mendelson],
che è possibile definire qualsiasi tipo di operatore vero-funzionale usando
solo gli operatori di negazione e disgiunzione inclusiva.
Quindi, analogamente a quanto fatto per i connettivi $\land, \rightarrow, \leftrightarrow$,
possiamo pensare di aggiungere qualsiasi operatore vero-funzionale ci venga in mente
a $CP$ definendolo in termini di $\neg$ e $\lor$.
Quindi anche se dalla presentazione non sembrerebbe, in realtà in $CP$ sono presenti
tutti gli operatori vero-funzionali possibili.

\subsection{Alcune proprietà importanti di $CP$}

Terminiamo questo capitolo richiamando alcune proprietà di $CP$ fondamentali, omettendone
le dimostrazioni. Un trattamento più approfondito di questi argomenti si può trovare in
[Tortora] o [Mendelson].

Esiste una procedura di decisione che data una qualsiasi formula ben formata di $CP$
ci permette sempre di determinare in un numero finito di passi se questa è valida o no.

Inoltre vale il seguente teorema che collega l'aspetto sintattico all'aspetto semantico
di $CP$:

\begin{flushleft}
\textbf{Teorema di completezza}
$$\vdash \alpha \Leftrightarrow \vDash \alpha$$
\end{flushleft}

Un corollario immediato del teorema di completezza è che anche l'insieme dei teoremi
di $CP$ è un insieme decidibile, infatti secondo il teorema di completezza
l'insieme dei teoremi di $CP$ e l'insieme delle formule ben formate di $CP$ valide
coincidono, quindi usando la procedura di decisione per le formule valide
possiamo anche determinare le formule che sono teoremi.

\section{Logica modale}
Ora che abbiamo definito un sistema formale per la logica proposizionale,
lo estenderemo in vari modi.

Il nostro obiettivo è avere un sistema formale nel quale possiamo definire e studiare
le proprietà di due nuovi operatori unari che vorremmo avessero un comportamento che rispecchi
i concetti di ``necessità'' e ``possibilità''.

Prima di cominciare facciamo delle osservazioni importanti.
Ricordiamo che un operatore verofunzionale è una funzione il cui risultato
è univocamente determinato dal valore di verità dei parametri da cui dipende.
Inoltre, si può dimostrare che nel calcolo proposizionale si possono esprimere
tutti gli operatori verofunzionali possibili in termini di $\neg e \vee$.
Quindi, se gli operatori che vogliamo aggiungere fossero verofunzionali, potremmo
continuare il nostro studio usando $CP$ e non servirebbe altro.
Notiamo, però, che gli operatori che vogliamo aggiungere non possono essere verofunzionali,
infatti il valore di verità della frase ``è necessario che p'' non dipende \textit{solo}
dal valore di verità che si associ, infatti, il fatto che p sia vera, non ci permette
di concludere che p sia necessariamente vera. Quindi per avere una speranza di formalizzare
i concetti di necessità e possibilità siamo costretti ad estendere il sistema $CP$
in maniera sostanziale.

A tal proposito, non è esattamente chiaro come caratterizzare formalmente
le nozioni di ``necessità'' e ``possibilità'', e anzi vedremo che queste potranno essere
intese in modi diversi.
Questa ricchezza di significati si rifletterà in una varietà di sistemi formali più o meno potenti.
Noi ci focalizzeremo in particolare su quattro sistemi formali:
\begin{itemize}
\item Sistema T
\item Sistema S4
\item Sistema S5
\end{itemize}

Questi sistemi differiscono tra di loro soltanto per gli assiomi scelti in ognuno di essi.
Quindi iniziamo definendo la parte comune di tutti i sistemi formali, ovvero il linguaggio
e le regole di deduzione.

L'alfabeto dei sistemi è ottenuto dall'alfabeto del sistema $CP$ aggiungendo il simbolo $\Box$,
che vorremo far corrispondere alla nozione di ``necessità''.

Per quanto riguarda le formule ben formate, la definizione induttiva che genera l'insieme $L$
delle formule ben formate è ottenuta dalle regole usate per definire $L$ in $CP$ più
la seguente regola aggiuntiva:
\begin{itemize}
\item Se $\alpha \in L$, allora $\Box (\alpha) \in L$
\end{itemize}

Diamo la seguente definizione, il cui intento è esprimere l'altro operatore
che ci interessa, quello corrispondente alla nozione di ``possibilità'', in termini di $\Box$.

\begin{definition}
Se $\alpha \in L$:
$$\diamond (\alpha) := \neg \Box \neg (\alpha)$$
\end{definition}

Infine, per quanto riguarda le regole di deduzione, alle due di $CP$ aggiungiamo
una nuova regola, detta \textbf{Regola di necessitazione}:
$$\vdash \alpha \Rightarrow \vdash \Box \alpha$$

Una volta definita la parte comune a tutti i sistemi a cui siamo interessati,
passiamo ora ad esaminare per ogni sistema gli assiomi che lo caratterizzano.

\section{Sistema T}
Il sistema T ha come assiomi tutti gli assiomi di $CP$ più:
\begin{itemize}
\item (K) $\Box (\alpha \rightarrow \beta) \rightarrow (\Box \alpha \rightarrow \Box \beta)$
\item (T) $\Box \alpha \rightarrow \alpha$
\end{itemize}

\section{Sistema S4}
Il sistema S4 ha come assiomi tutti gli assiomi del Sistema T più il seguente assioma:
$$(S4) \Box \alpha \rightarrow \Box \Box \alpha$$

\section{Sistema S5}
Il sistema S5 ha come assiomi tutti gli assiomi del Sistema T più:
$$(S5) \diamond \alpha \rightarrow \Box \diamond \alpha$$


Ora che abbiamo definito il linguaggio e l'apparato deduttivo di tutti i sistemi
che ci interessano, facciamo delle osservazioni su di essi.
\begin{definition}
Diremo che un sistema formale è meno forte di un altro se tutte i teoremi del primo
sono teoremi anche del secondo.
\end{definition}

Ovviamente, per come abbiamo costruito i nostri sistemi formali, il Sistema T è meno forte
sia di S4 che di S5.
Ora, però mostreremo che anche S4 è meno forte di S5. A tal fine premettiamo alcuni risultati
preliminari.

\begin{flushleft}
\textbf{Teorema}
$\vdash \alpha \rightarrow \diamond \alpha$ in $T$

\textit{Dim.}

(1) $\Box \neg \alpha \rightarrow \neg \alpha$ ; $T[\neg \alpha/\alpha]$ \\
(2) $\alpha \rightarrow \neg \Box \neg \alpha$ ; Contrapposta di (1) \\
(3) $\alpha \rightarrow \diamond \alpha$ ; Definizione di $\diamond$

\begin{flushright}
$\bullet$
\end{flushright}
\end{flushleft}

\begin{flushleft}
\textbf{Lemma 1}
$\vdash \diamond \Box x \rightarrow \Box x$ in $S5$

\textit{Dim.}

(1) $\diamond \neg x \rightarrow \Box \diamond \neg x$ ; $S5[\neg x/\alpha]$ \\
(2) $\neg \Box \diamond \neg x \rightarrow \neg \diamond \neg x$ ; contrapposta di (1) \\
(3) $\diamond \neg \diamond \neg x \rightarrow \neg \diamond \neg x$ ; definizione di $\diamond$ \\
(4) $\diamond \Box x \rightarrow \Box x$ ; definizione di $\diamond$ e $\Box$

\begin{flushright}
$\bullet$
\end{flushright}
\end{flushleft}

\begin{flushleft}
\textbf{Teorema 1}
$\vdash \Box x \rightarrow \Box \Box x$ in $S5$, ovvero il Sistema S5 è più forte del Sistema S4.

\textit{Dim.}

(1) $\Box x \rightarrow \diamond \Box x$ ; Teorema \\
(2) $(\Box x \rightarrow \diamond \Box x) \rightarrow (\diamond \Box x \rightarrow \Box\diamond\Box x) \rightarrow (\Box x \rightarrow \Box \diamond \Box x)$ ; Teorema \\
(3) $\diamond x \rightarrow \Box \diamond x$ ; Assioma S5 \\
(4) $\Box x \rightarrow \Box \diamond \Box x$ ; Due volte Modus Ponens usando (1) (2) e (3) \\
(5) $\diamond \Box x \rightarrow \Box x$ ; Lemma 1 \\
(4) $\Box (\diamond \Box x \rightarrow \Box x)$ ; Necessitazione (5) \\
(5) $\Box(\diamond\Box x \rightarrow \Box x) \rightarrow \Box \diamond \Box x \rightarrow \Box \Box x$ ; Assioma K \\
(6) $\Box\diamond\Box x \rightarrow \Box\Box x$ ; Modus Ponens (4) e (5) \\
(7) $(\Box x \rightarrow \diamond \Box x) \rightarrow (\Box\diamond\Box x \rightarrow \Box\Box x) \rightarrow (\Box x \rightarrow \Box\Box x)$ ; Teorema \\
(8) $\Box x \rightarrow \Box \Box x$ ; Due volte Modus Ponens usando (1) (5) e (6)

\begin{flushright}$\bullet$\end{flushright}
\end{flushleft}

\section{Un teorema di deduzione (DA AGGIUSTARE)}
In questo paragrafo proviamo a derivare un teorema di deduzione per il Sistema $S4$.
Non possiamo aspettarci che il teorema di deduzione per il calcolo proposizionale $CP$ valga,
infatti la regola di deduzione di necessitazione ci porterebbe ad asserire che:
$$ x \vdash \Box x \Rightarrow \vdash x \rightarrow \Box x $$
E ciò vorrebbe dire che le modalità non aggiungono nulla al sistema del calcolo proposizionale.

Se guardiamo attentamente la regola di necessitazione e la scriviamo in questo modo:
$$x \vdash \Box x$$
possiamo notare che l'ipotesi $x$ è sufficiente per dimostrare $\Box x$ che afferma molto più
che semplicemente $x$.
Notiamo anche che la regola di necessitazione ci permette di provare
$$x \vdash \Box \Box x$$

Quindi, far semplicemente passare la x da sinistra a destra, significherebbe affermare che
da ipotesi più deboli si possono comunque dedurre le stesse conclusioni. Allora possiamo provare,
nello spostare la x da sinistra a destra di $\vdash$, ad aggiungere un $\Box$.
Questo può bastare in sistemi come $S4$ o più potenti, perché da $\Box x$ possiamo dedurre
$\Box \Box x$, $\Box \Box \Box x$, ecc...
Mentre nel sistema T questo non è vero e quindi dovremo risolvere diversamente.

Recuperiamo quindi un teorema di deduzione, facendo la seguente modifica:
$$\alpha \vdash \beta \Rightarrow \vdash \Box \alpha \rightarrow \beta$$
Ora il precedente problema è facilmente risolvibile, infatti, la deduzione $x \vdash \Box x$
diventa il teorema $\vdash \Box x \rightarrow \Box x$ banalmente valido.
Otteniamo però anche il seguente teorema, $\vdash \Box x \rightarrow \Box \Box x$ che è l'assioma $S4$
Ecco perché abbiamo supposto di trovarci nel sistema $S4$.

Diamo una dimostrazione di

\begin{flushleft}
\textbf{Teorema}
$\alpha \vdash \beta \Leftrightarrow \vdash \Box \alpha \rightarrow \beta$ in $S4$

\textit{Dim.}
Lasciata al lettore, è una banale modifica della dimostrazione nel caso del calcolo proposizionale.
L'unica parte interessante è quella riguardante i teoremi ottenuti con la regola di necessitazione.
In quel caso dobbiamo far uso dell'assioma $S4$ ed è tutto banale.
\end{flushleft}



\section{Modalità e funzioni modali}
Rendiamo più formale il concetto di modalità.
\textbf{Definizione}
Una modalità è una successione di 0 o più di uno dei seguenti simboli: $\neg,\diamond,\Box$.

Viene da domandarci, ora, quante sono le modalità distinte che ci sono in ognuno dei sistemi che abbiamo considerato.
Si può dimostrare che nel Sistema $T$ ce ne sono infinite,
mentre in $S4$ e $S5$ ce ne sono solo un numero finito.

Parleremo più approfonditamente di tali questioni solo dopo aver introdotto una semantica
per il nostro linguaggio.

\begin{flushleft}
\textbf{Definizione 1}

Una funzione modale è una formula contenente almeno una funzione modale
\end{flushleft}

In $T$ ovviamente c'è un numero infinito di funzioni modali esprimibili.
In $S5$ ce n'è un numero finito, mentre sorprendentemente,
in $S4$ ce ne sono infinite, nonostante le modalità esprimibili siano in numero finito.

\section{Una semantica per i sistemi di logica modale}
Dopo aver definito il linguaggio comune a tutti i sistemi di logica modale introdotti,
lo abbiamo usato unicamente come base di un calcolo che usa le regole di deduzione per costruire
dimostrazioni. Come ben sappiamo un linguaggio può anche essere usato per parlare di qualcosa,
di oggetti esterni al sistema formale.
Fare ciò significa associare ad ogni formula ben formata del linguaggio un oggetto
che rappresenti il suo significato.
Quando seguiamo questo procedimento diciamo che abbiamo fornito una semantica per il linguaggio.

Definire semantiche per i sistemi formali ci aiuta sia a confrontare sistemi formali
con strutture maggiormente conosciute e ad assicurarci
che alcune costruzioni corrispondano alla nostra intuizione,
sia a determinare alcune proprietà del sistema formale. E nel nostro caso ci aiuterà anche
a comprendere meglio quali siano le nozioni di modali rappresentate in ciascuno dei sistemi
che abbiamo definito e come differiscono tra di loro.


Come abbiamo fatto per l'apparato deduttivo, prima di dare una semantica ai nostri sistemi,
partiamo ricordando come è fatta la semantica classica per $CP$.

Per i nostri sistemi di logica modale possiamo pensare di fare qualcosa di simile,
però non possiamo aspettarci che $V_I$ per formule del tipo $\Box \alpha$ si basi solo
sul valore di $V_I(\alpha)$.

La semantica che useremo è in gran parte dovuta a Saul A. Kripke (vale la pena notare
che pubblicò un teorema di completezza per i sistemi di logica modale all'età di 17 anni)
e si basa su un concetto già introdotto da Leibniz, quello dei mondi possibili.
Quindi non supponiamo più che esista un'unica realtà, ma che esistono tanti mondi possibili,
e in più, e questa è una delle idee chiave della semantica di Kripke, da alcuni mondi
si può accedere ad altri mondi, cioè sapere come sono fatte altre realtà alternative.
Il mondo in cui si può accedere agli altri mondi sarà la base per distinguire
tra i vari concetti di necessità e tra i vari sistemi formali che abbiamo definito prima.

\section{Semantica per il Sistema T}
Definiamo un T-modello come una tripla $(W, R, I)$, in cui:
\begin{itemize}
\item $W$ è un insieme non vuoto i cui elementi saranno chiamati mondi;
\item $R$ è una relazione binaria riflessiva su $W$, detta relazione di accessibilità;
\item $I$ è una funzione dall'insieme $L \times W$ all'insieme $\{0, 1\}$;
\end{itemize}

L'insieme $W$ è l'insieme di tutti i mondi.
$I$ ci fornisce per ogni mondo la lista di proposizione vere
e quella delle proposizioni false, analogamente al suo corrispettivo nella semantica per $CP$.
$R$, infine, è la \textit{relazione di accessibilità},
essa ci permette di determinare a quali mondi si può accedere da un determinato mondo.
Quindi se $w_1, w_2 \in W e w_1 R w_2$, allora una persona in $w_1$ potrà accedere
al mondo $w_2$ e sapere quali proposizioni sono vere in $w_2$.
Il fatto che $R$ sia riflessiva ci dice che l'unica garanzia che abbiamo
è che ogni persona può accedere al proprio mondo, e questa è una garanzia molto scarsa.

Definiamo come prima una funzione di valutazione $V : L \times W \to \{0, 1\}$ che dipenderà da un dato T-modello
$(w, W, R, I)$.

Se $\alpha, \beta \in L$:
\begin{itemize}
\item $V(p, w) = I(p, w)$, $\forall p \in A$
\item $V(\neg \alpha, w) = 1 - V(\alpha, w)$
\item $V(\alpha \vee \beta, w) = max\{V(\alpha, w), V(\beta)\}$
\item $V(\Box \alpha, w) = min\{ V(\alpha, w') : w R w' \}$
\end{itemize}

Le prime tre regole sono pressoché identiche a quelle specificate per la semantica di $CP$,
tranne per il fatto che nella prima regola usiamo la funzione $I$ valutandola per il mondo w
a cui siamo interessati.

La novità è la quarta regola, quella riguardante la semantica di $\Box$.
Questa regola afferma che nel mondo $w$, $\alpha$ è necessariamente vera se e solo se
è $\alpha$ vera in tutti i mondi accessibili da $w$.

Diciamo che una formula $\alpha$ è T-valida e scriviamo $\vDash \alpha$
se per ogni T-modello $(w, W, R, I)$ $\forall w \in W. V(\alpha, w) = 1$.

\section{Mostrare teorema di adeguatezza}


\section{Un sistema interessante(DA AGGIUSTARE)}
Dedichiamo questo paragrafo un sistema di logica modale particolare, chiamato Sistema $B$.
Esso è intimamente collegato alla logica intuizionista, come si evince dal nome,
infatti la B è la iniziale di Brouwer, logico padre dell'intuizionismo.

\end{document}
