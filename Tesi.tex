\documentclass[a4paper, titlepage, 12pt]{report}
\usepackage[nottoc, notlot, notlof]{tocbibind}
\usepackage[italian]{babel}
\usepackage[T1]{fontenc}
\usepackage[utf8]{inputenc}
\usepackage{lmodern}
\usepackage{amssymb}
\usepackage{amsmath}
\newtheorem{axiom}{Assioma}[chapter]
\newtheorem{theorem}{Teorema}[chapter]
\newtheorem{lemma}{Lemma}[chapter]
\newtheorem{corollario}{Corollario}[chapter]
\newtheorem{definition}{Definizione}[chapter]

\title{I sistemi di logica modale T, S4, S5: Sintassi e Semantica}
\date{}

\linespread{1.5}

% Ambiente per le dimostrazioni
\newenvironment{proof}
    {\textit{Dim.}
    }
    {\begin{flushright}$\bullet$\end{flushright}
    }

% Ambiente per le dimostrazioni formali formali
\newenvironment{formal_proof}
    {
    \begin{center}
    \begin{tabular} {c c|c}
    }
    {\end{tabular}
    \end{center}
    }

\newenvironment{dedication}
         {\vspace{6ex}\begin{quotation}\begin{center}\begin{em}}
         {\par\end{em}\end{center}\end{quotation}}

\begin{document}
\begin{dedication}
Alla mia famiglia e ai miei amici
\end{dedication}
\tableofcontents
\chapter*{Introduzione}
\addcontentsline{toc}{chapter}{Introduzione}
Lo scopo di questa tesi è di applicare i metodi della logica matematica
alla studio della logica modale che si occupa dei
ragionamenti che fanno uso delle nozioni
di \emph{necessità} e \emph{possibilità}.
Nel primo capitolo introduciamo un sistema formale per il calcolo
proposizionale da usare come base per gli sviluppi successivi,
presentandone apparato deduttivo, semantica e ricordandone
alcune delle proprietà più importanti come il teorema di completezza.
Nel secondo capitolo presentiamo tre dei più comuni sistemi formali
per la logica modale: T, S4 e S5 come estensione del calcolo proposizionale
definito nel primo capitolo,
Nel terzo capitolo ci occupiamo di introdurre una semantica nello stile
di Kripke insieme ad un teorema di adeguatezza per ciascuno dei sistemi e
con questi strumenti procediamo poi ad un'analisi approfondita di alcune proprietà
dei sistemi T, S4, S5, mostrando le differenze e i punti in comune tra essi
e chiarendo in che senso ciascun sistema formalizza i concetti di
\emph{necessità} e \emph{possibilità}.
Infine, nel quarto capitolo mostriamo per ciascun sistema una procedura di decisione
per le formule valide e terminiamo il lavoro iniziato con la dimostrazione
del teorema di adeguatezza nel terzo capitolo, dimostrando il teorema
di completezza per ciascun sistema.

\chapter{Logica Proposizionale Classica}
Premettiamo allo studio di sistemi di logica modale la definizione
di sintassi e semantica di un sistema formale per il calcolo proposizionale,
che da ora in poi denoteremo con $CP$, da usare come base per la definizione
dei sistemi formali successivi.
È possibile presentare un sistema di questo tipo in vari modi, tutti pressoché equivalenti tra di loro
e la strada che intendiamo seguire è stata preferita alle altre solo
perché riteniamo che sia quella che serve gli sviluppi successivi nel modo più semplice.

\section{Il Linguaggio}
Nel definire il linguaggio di $CP$ seguiamo l'approccio di \cite{IntroModale}
definendo un linguaggio con un numero ristretto
di connettivi primitivi e introducendo successivamente i connettivi mancanti definendoli a partire
dai connettivi primitivi.
Questa scelta ci permette di usare tutti i connettivi che siamo soliti adoperare,
rendendo quindi la scrittura delle formule più chiara, ma allo stesso tempo ci permette di
considerare solo un numero esiguo di connettivi nello studio delle proprietà del sistema,
semplificandoci il lavoro.

Supponiamo di avere un insieme $A$ non vuoto e numerabile disgiunto da $\{\neg, \lor, (, )\}$
e che contenga almeno tre elementi che denotiamo con $p, q, r$
(ad esempio possiamo usare l'insieme dei numeri naturali $\mathbb{N}$).

\begin{definition}
Chiamiamo variabile proposizionale ogni elemento di $A$.
\end{definition}

Come connettivi primitivi scegliamo $\neg$ e $\lor$, a partire da essi è possibile definire tutti gli altri connettivi.
L'alfabeto di $CP$ è dunque dato dall'insieme $\Sigma = A \cup \{\neg, \lor, (, )\}$.

Chiamiamo $\Sigma^{*} = \bigcup_{n \in A} \Sigma^n$ l'insieme delle espressioni
e definiamo l'insieme delle formule ben formate (fbf) $L_{CP} \subseteq \Sigma^{*}$ come segue:
\begin{definition}
$L_{CP}$ è il più piccolo sottoinsieme di $\Sigma^{*}$ tale che:
\begin{itemize}
\item $x \in A \Rightarrow x \in L_{CP}$;
\item $\alpha \in L_{CP} \Rightarrow \neg (\alpha) \in L_{CP}$;
\item $\alpha \in L_{CP}$ e $\beta \in L_{CP} \Rightarrow (\alpha) \lor (\beta) \in L_{CP}$;
\end{itemize}
\end{definition}

Una volta definito il linguaggio, definiamo i connettivi che mancano a partire da $\neg$ e $\lor$
come segue:
\begin{definition}
Se $\alpha \in L_{CP}$ e $\beta \in L_{CP}$, allora:
\begin{itemize}
\item $(\alpha) \land (\beta) := \neg(\neg(\alpha) \lor \neg(\beta))$
\item $(\alpha) \rightarrow (\beta) := (\neg(\alpha)) \lor (\beta)$
\item $(\alpha) \leftrightarrow (\beta) := ((\alpha) \rightarrow (\beta)) \land ((\beta) \rightarrow (\alpha))$
\end{itemize}
\end{definition}

Al fine di rendere le formule che scriveremo più comprensibili, seguendo la prassi,
ometteremo spesso le parentesi usando le convenzioni descritte in \cite{Tortora} a p. 64.

\section{L'apparato deduttivo}
Anche per l'apparato deduttivo ci basiamo su \cite{IntroModale}, eccezion fatta che
per la lista di assiomi,
questi ultimi sono presi dal sistema di calcolo proposizionale definito in \cite{Kleene}.

Vale la pena far notare ancora una volta che questo modo di procedere non è l'unico
possibile, ma ci sono tante altre scelte sia
per quanto riguarda gli assiomi che le regole di deduzione, che sono altrettanto valide
e che portano allo stesso risultato. La motivazione dietro la nostra scelta
risiede semplicemente nel fatto che questo approccio è quello da noi ritenuto più familiare.

\begin{definition}
Gli assiomi di $CP$ sono i seguenti:
\begin{itemize}
\item (HPD) $p \rightarrow (q \rightarrow p)$
\item (HPMP) $(r \rightarrow (p \rightarrow q)) \rightarrow (r \rightarrow p) \rightarrow (r \rightarrow q)$
\item ($\lor$-I1) $p \rightarrow (p \lor q)$
\item ($\lor$-I2) $q \rightarrow (p \lor q)$
\item ($\lor$-E) $(p \rightarrow r) \rightarrow (q \rightarrow r) \rightarrow (p \lor q \rightarrow r)$
\item ($\land$-I) $p \rightarrow q \rightarrow p \land q$
\item ($\land$-E1) $p \land q \rightarrow p$
\item ($\land$-E2) $p \land q \rightarrow q$
\item ($\neg$-I) $(p \rightarrow q) \land (p \rightarrow \neg q) \rightarrow \neg p$
\item (TER) $p \lor \neg p$
\end{itemize}
\end{definition}

Usiamo due regole di deduzione:

La prima regola di deduzione di $CP$ è il \emph{Modus Ponens}, che è l'insieme
di tutte le terne ordinate $<\alpha \rightarrow \beta$, $\alpha$, $\beta>$,
al variare di $\alpha$ e $\beta$ in $L_{CP}$.

Le fbf $\alpha \rightarrow \beta$ e $\alpha$ vengono dette premesse e la formula $\beta$
viene detta conseguenza nell'applicazione del \emph{Modus Ponens}.
Si dice anche che $\beta$ si deduce da $\alpha \rightarrow \beta$ e da $\alpha$
per \emph{Modus Ponens}.
Inoltre una tale terna si rappresenta usualmente nella forma grafica qui di seguito indicata:

$$\frac{\alpha \rightarrow \beta \ \ \alpha}{\beta}$$


Per enunciare la seconda dobbiamo prima dare la seguente definizione:
\begin{definition}
Se $\alpha \in L_{CP}, x \in A,$ e $\beta \in L_{CP}$, allora la formula
ottenuta sostituendo uniformemente $\beta$ al posto di $x$ in $\alpha$
e che indichiamo con $\alpha[x/\beta]_{CP}$ è definita come segue:
\begin{itemize}
\item Se $\alpha = x$, allora $\alpha[x/\beta]_{CP} = \beta$,
\item Se $\alpha \in A \setminus \{x\}$, allora $\alpha[x/\beta]_{CP} = \alpha$,
\item Se $\alpha = \neg (\gamma)$, allora $\alpha[x/\beta]_{CP} = \neg (\gamma[x/\beta]_{CP})$,
\item Se $\alpha = (\gamma) \lor (\delta)$,
      allora $\alpha[x/\beta]_{CP} = (\gamma[x/\beta]_{CP}) \lor (\delta[x/\gamma]_{CP})$
\end{itemize}
\end{definition}

Quindi definiamo la regola di \emph{sostituzione} come l'insieme
di tutte le coppie ordinate $<\alpha, \alpha[x/\beta]_{CP}>$, al variare di
$\alpha$ e $\beta$ in $L_{CP}$ e di $x$ in $A$.
La fbf $\alpha$ si dice premessa e la formula $\alpha[x/\beta]_{CP}$ viene detta conseguenza
nell'applicazione della regola di \emph{sostituzione}.
Si dice anche che $\alpha[x/\beta]_{CP}$ si deduce da $\alpha$ per \emph{sostituzione}.

Al contrario del \emph{Modus Ponens},
la regola di \emph{sostituzione} non è indispensabile. Abbiamo deciso di introdurla
perché ci permette di dimostrare vari teoremi essenzialmente in un solo colpo,
evitando di dover eseguire tante dimostrazioni che sarebbero essenzialmente sempre uguali.
Un altro metodo che ci consente di ottenere lo stesso risultato
è quello di usare un insieme infinito
di assiomi, facendo uso di schemi di assiomi, ma riteniamo che utilizzare un insieme
finito di formule di $L_{CP}$ congiuntamente alla regola di \emph{sostituzione}
ci permetta di estendere più agevolmente $CP$.

Diamo ora le seguenti definizioni che introducono i concetti di dimostrazione e di
deduzione, strettamente collegati tra loro.

\begin{definition}
Una dimostrazione è una sequenza finita $D$ di fbf tale che
ogni termine $D_i$ della sequenza soddisfa una delle seguenti condizioni
\begin{itemize}
\item $D_i$ è un assioma;
\item esistono $D_h, D_k$, con $h < i, k < i$, tali che $D_i$ è derivato per \emph{Modus Ponens}
da $D_h$ e $D_k$.
\item esiste $D_h$, con $h < i$, tale che $D_i$ è derivato tramite
la regola di \emph{sostituzione} a partire da $D_h$.
\end{itemize}
\end{definition}

Se $\alpha \in L_{CP}$ ed esiste una dimostrazione $D$ il cui ultimo termine è $\alpha$,
diciamo che esiste una dimostrazione di $\alpha$ e scriviamo $\vdash \alpha$.


\begin{definition}
Se $\Gamma \subseteq L_{CP}$, una deduzione a partire dalle ipotesi $\Gamma$ è una sequenza finita $D$
di fbf tale che ogni termine $D_i$ della sequenza soddisfa una delle seguenti condizioni
\begin{itemize}
\item $D_i \in \Gamma$;
\item $D_i$ è un assioma;
\item esistono $D_h, D_k$, con $h < i, k < i$, tali che $D_i$ è derivato per \emph{Modus Ponens}
da $D_h$ e $D_k$.
\item esiste $D_h$, con $h < i$, tale che $D_i$ è ottenuto tramite
la regola di \emph{sostituzione} a partire da $D_h$.
\end{itemize}
\end{definition}

Se $\alpha \in L_{CP}$, $\Gamma \subseteq L_{CP}$ ed esiste una deduzione $D$ a partire dalle ipotesi $\Gamma$
il cui ultimo termine è $\alpha$, diciamo che esiste una deduzione di $\alpha$
a partire dalle ipotesi $\Gamma$ e scriviamo $\Gamma \vdash \alpha$.

Nel seguito, scriveremo $\Gamma, \alpha \vdash \beta$
per indicare $\Gamma \cup \{\alpha\} \vdash \beta$.

Vale la pena ricordare il seguente teorema:
\begin{theorem}[Teorema di Deduzione]
Se $\alpha, \beta \in L_{CP},$
$$\Gamma, \alpha \vdash \beta \Leftrightarrow \Gamma \vdash \alpha \rightarrow \beta$$
\end{theorem}
\begin{proof}
Una dimostrazione può essere trovata in \cite{Mendelson}, e può essere facilmente
adattata al sistema $CP$.
\end{proof}

Terminiamo questo paragrafo mostrando un esempio di dimostrazione di una famiglia
di teoremi, che ci tornerà utile anche nel seguito.

\begin{theorem}[Sillogismo] \label{sillogismo}
$$\vdash (p \rightarrow q) \rightarrow
        (q \rightarrow r) \rightarrow (p \rightarrow r)$$
\end{theorem}

\begin{proof}
Semplifichiamo il lavoro dimostrando
$$(p \rightarrow q),(q \rightarrow r),p \vdash r$$
in luogo di
$$\vdash (p \rightarrow q) \rightarrow
        (q \rightarrow r) \rightarrow (p \rightarrow r)$$
e sfruttiamo poi il Teorema di Deduzione sopra menzionato.

Una possibile deduzione per $r$ a partire dalle ipotesi
$\{(p \rightarrow q),(q \rightarrow r),p\}$ è la seguente.
\begin{formal_proof}
(1) & $p \rightarrow q$ & ipotesi \\
(2) & $p$ & ipotesi \\
(3) & $q$ & Modus Ponens da (1) e (2) \\
(4) & $q \rightarrow r$ & ipotesi \\
(5) & $r$ & Modus Ponens da (3) e (4)
\end{formal_proof}

Possiamo dunque affermare:
$$(p \rightarrow q),(q \rightarrow r),p \vdash r$$

Infine, applicando tre volte il Teorema di Deduzione otteniamo:
$$\vdash (p \rightarrow q) \rightarrow (q \rightarrow r) \rightarrow (p \rightarrow r)$$
\end{proof}

\section{Semantica di $CP$} \label{SemanticaCP}
Ci proponiamo ora di fornire una semantica per $CP$.
La semantica che intendiamo fornire per $CP$ è quella standard, in particolare
seguiamo l'esposizione di \cite{Tortora} per presentarla.

\begin{definition}
Chiamiamo valori di verità gli elementi dell'insieme $\{0, 1\}$
\end{definition}

\begin{definition}
Chiamiamo interpretazione una funzione $I: A \to \{0, 1\}$
\end{definition}

Data un'interpretazione $I$ definiamo per ricorsione la funzione di valutazione $V_I$ che associa ad ogni formula
del linguaggio di $CP$ un valore dell'insieme $\{0, 1\}$ come segue:

Se $\alpha, \beta \in L_{CP}$:
\begin{itemize}
\item $V_I(p) = I(p)$, $\forall p \in A$
\item $V_I(\alpha \lor \beta) = max\{V_I(\alpha), V_I(\beta)\}$
\item $V_I(\neg \alpha) = 1 - V_I(\alpha)$
\end{itemize}


Quindi un'interpretazione ci dà in un certo senso una lista di tutte le proposizioni
vere e di tutte le proposizioni false.
Usiamo poi i valori assegnati dall'interpretazione per determinare il valore di verità delle formule
in base alla semantica che abbiamo assegnato a ciascun connettivo (che sono codificate nelle regole
che abbiamo usato per definire la funzione $V_I$).

Osserviamo che ciascuno dei connettivi $\neg$ e $\lor$
è interpretato come una funzione il cui risultato
dipende unicamente dal valore di verità dei suoi argomenti.
Connettivi che possono essere interpretati in questo modo
vengono detti \emph{operatori vero-funzionali}.

\begin{definition}
Sia $\alpha \in L_{CP}$, diremo che $\alpha$ è valida e scriveremo $\vDash \alpha$ se
$V_I(\alpha) = 1$, per ogni interpretazione $I$.
\end{definition}

È noto, e una discussione approfondita di questo argomento si può trovare in \cite{Mendelson}
pp. 27--29,
che è possibile definire qualsiasi tipo di operatore vero-funzionale usando
solo gli operatori di negazione e disgiunzione inclusiva.
Quindi, analogamente a quanto fatto per i connettivi $\land, \rightarrow, \leftrightarrow$,
possiamo pensare di aggiungere qualsiasi operatore vero-funzionale ci venga in mente
a $CP$ definendolo in termini di $\neg$ e $\lor$.
Quindi anche se dalla presentazione non sembrerebbe, in realtà in $CP$ sono presenti
tutti gli operatori vero-funzionali possibili.

\section{Alcune proprietà importanti di $CP$}

Terminiamo questo capitolo richiamando alcune proprietà di $CP$ fondamentali, omettendone
le dimostrazioni. Un trattamento più approfondito di questi argomenti si può trovare in
\cite{Tortora} o \cite{Mendelson}.

Esiste una procedura di decisione che, data una qualsiasi formula ben formata di $CP$,
ci permette sempre di determinare in un numero finito di passi se questa è valida o meno.

Inoltre vale il seguente teorema che collega l'aspetto sintattico all'aspetto semantico
di $CP$:

\begin{theorem}[Teorema di Completezza]
Se $\alpha \in L_{CP}$,
$$\vdash \alpha \Leftrightarrow \vDash \alpha$$
\end{theorem}

Un corollario immediato del Teorema di Completezza è che anche l'insieme dei teoremi
di $CP$ è un insieme decidibile, infatti secondo il teorema di completezza
l'insieme dei teoremi di $CP$ e l'insieme delle formule ben formate di $CP$ valide
coincidono, quindi usando la procedura di decisione per le formule valide
possiamo anche determinare le formule che sono teoremi.

\chapter{I Sistemi di Logica Modale}

Ora che abbiamo definito un sistema formale per la logica proposizionale,
lo estenderemo in vari modi.

Il nostro obiettivo è avere un sistema formale nel quale possiamo definire e studiare
le proprietà di due nuovi operatori unari che vorremmo avessero un comportamento che rispecchi
i concetti di \emph{necessità} e \emph{possibilità}.

Prima di cominciare facciamo delle osservazioni importanti.
Ricordiamo che un operatore vero-funzionale è un connettivo
interpretabile come una funzione il cui risultato
è univocamente determinato dal valore di verità dei suoi argomenti.
Quindi, se gli operatori che vogliamo aggiungere fossero vero-funzionali,
per quanto detto nel paragrafo \ref{SemanticaCP} potremmo
continuare il nostro studio usando $CP$ e non servirebbe altro.
Notiamo, però, che i connettivi che vogliamo aggiungere non possono essere vero-funzionali,
infatti il valore di verità della frase ``è necessario che p'' non dipende \emph{solo}
dal valore di verità che si associa a p. Infatti, il fatto che p sia vera, non ci permette
di concludere che p sia necessariamente vera. Quindi per avere una speranza di formalizzare
i concetti di necessità e possibilità siamo costretti ad estendere il sistema $CP$
in maniera sostanziale.

A tal proposito, non è esattamente chiaro come caratterizzare formalmente
le nozioni di \emph{necessità} e \emph{possibilità},
e anzi vedremo che queste potranno essere intese in modi diversi.
Questa ricchezza di significati si riflette in una varietà di sistemi formali più o meno potenti.
Noi ci focalizzeremo in particolare su tre dei sistemi formali definiti in \cite{IntroModale}:
\begin{itemize}
\item Sistema T
\item Sistema S4
\item Sistema S5
\end{itemize}

Questi sistemi differiscono tra di loro soltanto per gli assiomi scelti in ognuno di essi.
Quindi iniziamo definendo la parte comune di tutti i sistemi formali, ovvero il linguaggio
e le regole di deduzione.

L'alfabeto dei sistemi è ottenuto dall'alfabeto del sistema $CP$ aggiungendo il simbolo $\Box$,
che corrisponde alla nozione di \emph{necessità}.

Per quanto riguarda le formule ben formate, la definizione induttiva che genera l'insieme $L$
delle formule ben formate è ottenuta dalle regole usate per definire $L$ in $CP$ più
la seguente regola aggiuntiva:
\begin{itemize}
\item Se $\alpha \in L$, allora $\Box (\alpha) \in L$
\end{itemize}

La formula $\Box (\alpha)$ si legge: ``È necessario (che) $\alpha$''.

Diamo la seguente definizione, il cui intento è esprimere l'altro operatore
che ci interessa, quello corrispondente alla nozione di \emph{possibilità},
in termini di $\Box$.

\begin{definition}
Se $\alpha \in L$:
$$\diamond (\alpha) := \neg \Box \neg (\alpha)$$
\end{definition}

Infine, per quanto riguarda le regole di deduzione,
usiamo il \emph{Modus Ponens} e la regola di \emph{sostituzione}
come per $CP$. Per la regola di sostituzione bisogna però fare attenzione,
infatti la nozione di sostituzione uniforme di formule al posto di variabili
proposizionali va aggiornata e quindi per le formule di $L$ definiamo:

\begin{definition}
Se $\alpha \in L, x \in A,$ e $\beta \in L$, allora la formula
ottenuta sostituendo uniformemente $\beta$ al posto di $x$ in $\alpha$
e che indichiamo con $\alpha[x/\beta]$ è definita come segue:
\begin{itemize}
\item Se $\alpha = x$, allora $\alpha[x/\beta] = \beta$,
\item Se $\alpha \in A \setminus \{x\}$, allora $\alpha[x/\beta] = \alpha$,
\item Se $\alpha = \neg (\gamma)$, allora $\alpha[x/\beta] = \neg (\gamma[x/\beta])$,
\item Se $\alpha = \Box (\gamma)$, allora $\alpha[x/\beta] = \Box (\gamma[x/\beta])$,
\item Se $\alpha = (\gamma) \lor (\delta)$,
      allora $\alpha[x/\beta] = (\gamma[x/\beta]) \lor (\delta[x/\gamma])$
\end{itemize}
\end{definition}

Osserviamo che la nuova definizione è un'estensione della definizione data per le
formule di $CP$ e quindi ogni sostituzione uniforme in $CP$ è anche una sostituzione
uniforme nei nuovi sistemi che stiamo definendo.

Quindi definiamo la regola di \emph{sostituzione} come l'insieme
di tutte le coppie ordinate $<\alpha, \alpha[x/\beta]>$, al variare di
$\alpha$ e $\beta$ in $L$ e di $x$ in $A$.
La fbf $\alpha$ si dice premessa e la formula $\alpha[x/\beta]$ viene detta conseguenza
nell'applicazione della regola di \emph{sostituzione}.
Si dice anche che $\alpha[x/\beta]$ si deduce da $\alpha$ per \emph{sostituzione}.


A queste due regole ne aggiungiamo una nuova, detta regola di \emph{necessitazione},
essa è l'insieme
di tutte le coppie ordinate $<\alpha, \Box \alpha>$,
al variare di $\alpha$ in $L$.

La fbf $\alpha$ viene detta premessa e la formula $\Box \alpha$
viene detta conseguenza nell'applicazione della regola di \emph{necessitazione}.
Si dice anche che $\Box \alpha$ si deduce da $\alpha$
per \emph{necessitazione}.
Inoltre una tale coppia si rappresenta usualmente nella forma grafica qui di seguito indicata:

$$\frac{\alpha}{\Box \alpha}$$

Vale la pena soffermarsi brevemente sulla regola di necessitazione, 
infatti interpretando $\Box$ come ``è necessario che'' la regola di necessitazione
sembra affermare che da $\alpha$ si possa dedurre che $\alpha$ sia necessaria e quindi
che da ipotesi deboli si possa arrivare ad una conclusione più forte.
Invece mettiamo in evidenza che ciò che effettivamente la regola di necessitazione
afferma è che se $\alpha$ è un teorema, allora da essa possiamo dedurre $\Box \alpha$,
quindi non è assolutamente possibile applicare la regola di necessitazione
a qualsiasi formula, ma solo alle formule che sono teoremi, e in un certo senso
ciò che questa regola afferma è che dire che $\alpha$ è un teorema
è sufficiente per dedurre che $\alpha$
sia necessaria, conclusione che ora ci sembra più sensata.


Una volta definita la parte comune a tutti i sistemi a cui siamo interessati,
passiamo ora ad esaminare per ogni sistema gli assiomi che lo caratterizzano.

Il sistema T ha come assiomi tutti gli assiomi di $CP$ più:
\begin{itemize}
\item (K) $\Box (p \rightarrow q) \rightarrow (\Box p \rightarrow \Box q)$
\item (T) $\Box p \rightarrow p$
\end{itemize}

Il sistema S4 ha come assiomi tutti gli assiomi del Sistema T più il seguente assioma:
$$(S4) \Box p \rightarrow \Box \Box p$$

Il sistema S5 ha come assiomi tutti gli assiomi del Sistema T più:
$$(S5) \diamond p \rightarrow \Box \diamond p$$


Ora che abbiamo definito il linguaggio e l'apparato deduttivo di tutti i sistemi
che ci interessano, facciamo delle osservazioni su di essi.
\begin{definition}
Diremo che un sistema formale è meno forte di un altro se tutti i teoremi del primo
sono teoremi anche del secondo.
\end{definition}

Ovviamente, per come abbiamo costruito i nostri sistemi formali, il Sistema T è meno forte
sia di S4 che di S5.
Inoltre il sistema $CP$ è meno forte sia di T che di S4 che di S5.
Ora, però mostreremo che anche S4 è meno forte di S5. A tal fine premettiamo alcuni risultati
preliminari.

\begin{theorem} \label{DualeT}
$\vdash p \rightarrow \diamond p$ in $T$
\end{theorem}

\begin{proof}
\begin{formal_proof}
(1) & $\Box \neg p \rightarrow \neg p$ & $T[p/\neg p]$ \\
(2) & $p \rightarrow \neg \Box \neg p$ & Contrapposta di (1) \\
(3) & $p \rightarrow \diamond p$ & Definizione di $\diamond$
\end{formal_proof}
\end{proof}

\begin{lemma} \label{Diamond-Box-Box}
$\vdash \diamond \Box p \rightarrow \Box p$ in $S5$
\end{lemma}

\begin{proof}
\begin{formal_proof}
(1) & $\diamond \neg p \rightarrow \Box \diamond \neg p$ & $S5[p/\neg p]$ \\
(2) & $\neg \Box \diamond \neg p \rightarrow \neg \diamond \neg p$ & contrapposta di (1) \\
(3) & $\diamond \neg \diamond \neg p \rightarrow \neg \diamond \neg p$ & definizione di $\diamond$ \\
(4) & $\diamond \Box p \rightarrow \Box p$ & definizione di $\diamond$ e $\Box$
\end{formal_proof}
\end{proof}

\begin{theorem}
$\vdash \Box p \rightarrow \Box \Box p$ in $S5$, ovvero il Sistema S5 è più forte del Sistema S4.
\end{theorem}

\begin{proof}
\begin{formal_proof}

(1) & $\Box p \rightarrow \diamond \Box p$ & Teorema \ref{DualeT}[p/$\Box p$] \\
(2) & $(\Box p \rightarrow \diamond \Box p) \rightarrow (\diamond \Box p \rightarrow \Box\diamond\Box p) \rightarrow (\Box p \rightarrow \Box \diamond \Box p)$ & Sostituzione di Teorema \ref{sillogismo} \\
(3) & $\diamond \Box p \rightarrow \Box \diamond \Box p$ & S5[p/$\Box p$] \\
(4) & $(\diamond \Box p \rightarrow \Box \diamond \Box p) \rightarrow \Box p \rightarrow \Box \diamond \Box p$ & Modus Ponens da (1) e (2) \\
(5) & $\Box p \rightarrow \Box \diamond \Box p$ & Modus Ponens da (3) e (4) \\
(6) & $\diamond \Box p \rightarrow \Box p$ & Lemma \ref{Diamond-Box-Box} \\
(7) & $\Box (\diamond \Box p \rightarrow \Box p)$ & Necessitazione (6) \\
(8) & $\Box(\diamond\Box p \rightarrow \Box p) \rightarrow \Box \diamond \Box p \rightarrow \Box \Box p$ & K[$p$/$\diamond\Box p$][$q$/$\Box p$] \\
(9) & $\Box\diamond\Box p \rightarrow \Box\Box p$ & Modus Ponens da (7) e (8) \\
(10) & $(\Box p \rightarrow \Box \diamond \Box p) \rightarrow (\Box\diamond\Box p \rightarrow \Box\Box p) \rightarrow (\Box p \rightarrow \Box\Box p)$ & Sostituzione di Teorema \ref{sillogismo} \\
(11) & $(\Box\diamond\Box p \rightarrow \Box\Box p) \rightarrow (\Box p \rightarrow \Box\Box p)$ & Modus Ponens da (5) e (10)\\
(12) & $\Box p \rightarrow \Box \Box p$ & Modus Ponens da (9) e (11)

\end{formal_proof}
\end{proof}

\chapter{Una semantica per i sistemi di logica modale}
Dopo aver definito il linguaggio comune a tutti i sistemi di logica modale introdotti,
lo abbiamo usato finora come base di un calcolo che usa le regole di deduzione per costruire
dimostrazioni. Come ben sappiamo un linguaggio può anche essere usato per parlare di qualcosa,
di oggetti esterni al sistema formale.
Fare ciò significa associare ad ogni formula ben formata del linguaggio un'affermazione
che rappresenti il suo significato.
Quando seguiamo questo procedimento diciamo che abbiamo fornito una semantica per il linguaggio.

Definire semantiche per i sistemi formali ci aiuta sia a confrontare sistemi formali
con strutture maggiormente conosciute e ad assicurarci
che alcune costruzioni corrispondano alla nostra intuizione,
sia a determinare alcune proprietà del sistema formale. E nel nostro caso ci aiuterà anche
a comprendere meglio quali siano le nozioni modali rappresentate in ciascuno dei sistemi
che abbiamo definito e come differiscono tra di loro.

Per i nostri sistemi di logica modale possiamo pensare di fare qualcosa di simile
a quanto fatto per $CP$, però poiché $\Box$, come abbiamo visto, non può essere
un operatore vero-funzionale, non possiamo aspettarci di poter definire la
funzione di valutazione per formule
del tipo $\Box \alpha$ basandoci unicamente sul valore di verità di $\alpha$.

La semantica che useremo è in gran parte dovuta a Saul A. Kripke (vale la pena notare
che pubblicò un teorema di completezza per i sistemi di logica modale all'età di 17 anni)
e si basa su un concetto già introdotto da Leibniz, quello dei mondi possibili.
Quindi non supponiamo più che esista un'unica realtà, ma che esistono tanti mondi possibili,
e in più, e questa è una delle idee chiave della semantica di Kripke, da alcuni mondi
si può accedere ad altri mondi, cioè sapere come sono fatte altre realtà alternative.
Il modo con cui si può accedere agli altri mondi sarà la base per distinguire
tra i vari concetti di necessità e tra i vari sistemi formali che abbiamo definito prima.

\section{Semantica per il Sistema T}
Definiamo un T-modello come una tripla $(W, R, I)$, in cui:
\begin{itemize}
\item $W$ è un insieme non vuoto i cui elementi saranno chiamati mondi;
\item $R$ è una relazione binaria riflessiva su $W$, detta relazione di accessibilità;
\item $I$ è una funzione dall'insieme $L \times W$ all'insieme $\{0, 1\}$;
\end{itemize}

L'insieme $W$ è l'insieme di tutti i mondi.
$I$ ci fornisce, per ogni mondo, la lista delle proposizioni vere
e quella delle proposizioni false, analogamente al suo corrispettivo nella semantica per $CP$.
$R$, infine, è la \textit{relazione di accessibilità},
essa ci permette di determinare a quali mondi si può accedere da un determinato mondo.
Quindi se $w_1, w_2 \in W e w_1 R w_2$, allora una persona in $w_1$ potrà accedere
al mondo $w_2$ e sapere quali proposizioni sono vere in $w_2$.
Il fatto che $R$ sia riflessiva ci dice che l'unica garanzia che abbiamo
è che ogni persona può accedere al proprio mondo, e questa è una garanzia molto scarsa.

Definiamo come prima una funzione di valutazione $V : L \times W \to \{0, 1\}$ che dipenderà da un dato T-modello
$(w, W, R, I)$.

Se $\alpha, \beta \in L$:
\begin{itemize}
\item $V(p, w) = I(p, w)$, $\forall p \in A$
\item $V(\neg \alpha, w) = 1 - V(\alpha, w)$
\item $V(\alpha \vee \beta, w) = max\{V(\alpha, w), V(\beta)\}$
\item $V(\Box \alpha, w) = min\{ V(\alpha, w') : w R w' \}$
\end{itemize}

Le prime tre regole sono pressoché identiche a quelle specificate per la semantica di $CP$,
tranne per il fatto che nella prima regola usiamo la funzione $I$ valutandola per il mondo w
a cui siamo interessati.

La novità è la quarta regola, quella riguardante la semantica di $\Box$.
Questa regola afferma che nel mondo $w$, $\alpha$ è necessariamente vera se e solo se
$\alpha$ è vera in tutti i mondi accessibili da $w$.

Diciamo che una formula $\alpha$ è T-valida e scriviamo $\vDash \alpha$
se per ogni T-modello $(w, W, R, I)$ $\forall w \in W. V(\alpha, w) = 1$.

Ci occupiamo adesso di mostrare che tutti i teoremi del sistema T sono formule valide
nella semantica che abbiamo fornito. Un risultato di questo tipo nel quale
ci si assicura che ogni teorema di un sistema formale è anche una formula
valida in una sua semantica viene chiamato
\emph{teorema di adeguatezza} e può essere visto come una conferma
importante che la semantica scelta sia una buona scelta per l'interpretazione
del sistema formale.
Procediamo dunque all'enuciazione e alla dimostrazione del:

\begin{theorem}[Teorema di adeguatezza per T]
$\forall \alpha \in L. \vdash \alpha \Rightarrow \vDash \alpha$ in $T$
\end{theorem}

\begin{proof}

Premettiamo alla dimostrazione vera e propria le seguenti osservazioni:

Se $(W, R, I)$ è un T-modello, $w \in W$ e $\alpha, \beta \in L$, allora:
$V(\alpha \rightarrow \beta, w) = V(\neg \alpha \lor \beta, w) = 0$ se e solo se
$V(\neg \alpha, w) = 0$ e $V(\beta, w) = 0$, ovvero, se e solo se $V(\alpha, w) = 1$ e $V(\beta, w) = 0$.

Se poi $V(\alpha \rightarrow \beta, w) = 1$ e $V(\alpha, w) = 1$, allora, se fosse
$V(\beta, w) = 0$, per quanto detto prima avremmo $V(\alpha \rightarrow \beta, w) = 1$, ma ciò
è assurdo, quindi deve essere $V(\beta, w) = 1$.

Occupiamoci ora di dimostrare il teorema di adeguatezza, la strategia che seguiremo è la seguente:
mostriamo che tutti gli assiomi sono T-validi e che le regole di deduzione conservano la T-validità.
Di conseguenza per le condizioni che abbiamo dato sulle formule ben formate che formano una dimostrazione,
deduciamo che tutte le formule ben formate che appaiono in una dimostrazione sono T-valide
ed in particolare l'ultima formula della sequenza finita è T-valida.
Questo sarà sufficiente per giungere alla tesi.

Iniziamo col mostrare che tutti gli assiomi sono T-validi:

Mostriamo che $\vDash \Box \alpha \rightarrow \alpha$:

Se $(W, R, I)$ è un T-modello, per quanto visto in precedenza abbiamo che:
$V(\Box \alpha \rightarrow \alpha, w) = 0 \Leftrightarrow V(\Box \alpha, w) = 1$ e $V(\alpha, w) = 1$.
Siccome $R$ è una relazione riflessiva, dalla definizione di $V$ segue che:
$V(\Box \alpha, w) <= V(\alpha, w)$ e quindi se $V(\Box \alpha, w) = 1$, si ha anche che
$V(\alpha, w) = 1$ e quindi otteniamo che
$\forall w \in W. V(\Box \alpha \rightarrow \alpha, w) = 1$ e per l'arbitrarietà
del T-modello otteniamo l'asserto.

Mostriamo che $\vDash \Box (\alpha \rightarrow \beta) \rightarrow (\Box \alpha \rightarrow \Box \beta)$:

Se $(W, R, I)$ è un T-modello, come prima abbiamo che:
$V(\Box (\alpha \rightarrow \beta) \rightarrow (\Box \alpha \rightarrow \Box \beta), w) = 0
\Leftrightarrow V(\Box (\alpha \rightarrow \beta), w) = 1$ e $V(\Box \alpha \rightarrow \Box \beta, w) = 0$.

Se $V(\Box(\alpha \rightarrow \beta), w) = 1$, allora: $\forall w' \in W. wRw' \Rightarrow V(\alpha \rightarrow \beta, w') = 1$
Se fosse $V(\Box \alpha, w) = 1$ e $V(\Box \beta, w) = 0$, allora si avrebbe la seguente
situazione: $\forall w' \in W. wRw' \Rightarrow V(\alpha, w') = 1$. Quindi avremmo che:
$\forall w' \in W. wRw' \Rightarrow V(\alpha \rightarrow \beta, w') = 1$ e $V(\alpha, w') = 1$,
dunque per l'osservazione precedente si avrebbe: $\forall w' \in W. wRw' \Rightarrow V(\beta, w') = 1$
e perciò $V(\Box \beta, w) = 1$, il che è assurdo.

Quindi se $V(\Box(\alpha \rightarrow \beta), w) = 1$, deve essere $V(\Box \alpha \rightarrow \Box \beta, w) = 1$,
e quindi per l'arbitrarietà di $w$ e del T-modello otteniamo l'asserto.

Ci resta da verificare la T-validità degli assiomi ereditati da $CP$.
Per il teorema di completezza per $CP$, tutti gli assiomi sono formule valide per $CP$,
e poiché per la definizione di $V$ per i T-modelli coincide con la definizione di
$V_I$ nella semantica di $CP$ per quanto riguarda le formule contenenti solo
variabili proposizionali e i connettivi comuni a $T$ e $CP$, è facile
verificare che gli assiomi derivati da $CP$ sono anche T-validi.

Quindi, poiché, una volta forniti i valori di verità per $\alpha$ e $\beta$,
le regole per determinare i valori di verità delle formule $\neg \alpha$ e $\alpha \lor \beta$
sono le stesse sia per $CP$ che per il sistema T
($\neg$ e $\lor$ sono interpretati allo stesso modo sia in T che in $CP$) e poiché
gli schemi di assiomi ereditati da $CP$ contengono solo $\neg$ e $\lor$ come connettivi
(gli altri connettivi di $CP$ sono definiti in termini di questi due),
possiamo dedurre che tutte le possibili sostituzioni uniformi di formule di $L$
negli schemi di assiomi derivati da $CP$ forniscono formule ancora valide.

Ora dimostriamo che il \emph{Modus Ponens} conserva la validità:

Se $\alpha \rightarrow \beta$ e $\alpha$ sono T-valide, allora fissato un T-modello $(W, R, I)$,
$\forall w \in W. V(\alpha \rightarrow \beta, w) = 1$ e $V(\alpha, w) = 1$.
Allora per l'osservazione che abbiamo fatto prima otteniamo che:
$\forall w \in W. V(\beta, w) = 1$. Per l'arbitrarietà del T-modello otteniamo l'asserto.

Poi dimostriamo che la regola di \emph{sostituzione} conserva la validità:
Se $\alpha$ è T-valida, $x$ è una variabile proposizionale occorrente in $\alpha$
e $\beta \in L$, dato un T-modello $(W, R, I)$ definiamo
la funzione $I' : A \times W \to \{0, 1\}$ in questo modo:
\begin{itemize}
    \item $I'(x, w) = V(\beta, w), \forall w \in W$
    \item $I'(y, w) = I(y, w), \forall w \in W, \forall y \in A \setminus \{x\}$
\end{itemize}
Poiché $\alpha$ è T-valida, dato il T-modello $(W, R, I')$, vale
$V(\alpha, w) = 1, \forall w \in W$.
Mostriamo, allora, per induzione sulla lunghezza di $\alpha$ che
vale $V(\alpha[x/\beta], w) = V'(\alpha, w), \forall w \in W$:
\begin{itemize}
\item Se $\alpha = x \in A$, allora $V'(x, w) = I'(x, w) = I(\beta, w)$
      e quindi l'asserto;
\item Se $\alpha = y \in A \setminus \{x\}$, allora l'asserto è banale;
\item Se $\alpha = \neg \gamma$, allora per ipotesi induttiva,
      $V'(\gamma, w) = V(\gamma[x/\beta], w), \forall w \in W$ e quindi l'asserto;
\item Se $\alpha = \gamma \lor \delta$, allora per ipotesi induttiva,
      $V'(\gamma, w) = V(\gamma[x/\beta], w), \forall w \in W$,
      $V'(\delta, w) = V(\delta[x/\beta], w), \forall w \in W$
      e quindi l'asserto;

\item Se $\alpha = \Box \gamma$, allora per ipotesi induttiva,
      $V'(\gamma, w) = V(\gamma[\gamma[x/\beta], w), \forall w \in W$,
      quindi l'asserto.

\end{itemize}

Ora, poiché $\alpha$ è T-valida, allora $V(\alpha[x/\beta], w) = V'(\alpha, w) = 1, \forall w \in W$,
e per l'arbitrarietà del T-modello, otteniamo che $\alpha[x/\beta]$ è una formula T-valida.

Infine dimostriamo che la regola di necessitazione conserva la validità:
Fissato un T-modello $(W, R, I)$ e un mondo $w \in W$,
se $\alpha$ è T-valida, allora $\forall w' \in W.V(\alpha, w') = 1$,
quindi in particolare si ha: $\forall w' \in W. wRw' \Rightarrow V(\alpha, w') = 1$.
Quindi $V(\Box \alpha, w) = 1$ e per l'arbitrarietà di $w$ e del T-modello otteniamo l'asserto.

\end{proof}

\section{Semantica per i sistemi S4 e S5}
Con leggere modifiche sulle condizioni da richiedere
per la relazione di accessibilità dei T-modello possiamo costruire delle semantiche
anche per i sistemi S4 e S5.

\begin{definition}
Un S4-modello è un T-modello $(W, R, I)$ in cui $R$ oltre ad essere riflessiva è anche transitiva.
\end{definition}

\begin{definition}
Un S5-modello è un T-modello $(W, R, I)$ in cui $R$ è la relazione di equivalenza totale
(e in particolare è una relazione di equivalenza),
vale a dire che si ha: $\forall w_1, w_2 \in W. w_1 R w_2$.
\end{definition}

Anche per S4 e S5 proviamo un teorema di adeguatezza

\begin{theorem}[Teorema di adeguatezza per S4]
$\vdash \alpha \Rightarrow \vDash \alpha$ in $S4$
\end{theorem}

\begin{proof}
Siccome un S4-modello è anche un T-modello e poiché ogni teorema del Sistema T
è anche un teorema del Sistema S4, possiamo sfruttare il teorema di adeguatezza per T
e dimostrare qui solo che l'assioma $S4$ è una formula valida:

Se $(W, R, I)$ è un S4-modello e $w \in W$,
se $V(\Box \alpha, w) = 1$, allora $\forall w_1 \in W. wRw_1 \Rightarrow V(\alpha, w_1) = 1$,
quindi poiché $R$ è riflessiva e transitiva
se $w_1, w_2 \in W$ e $wRw_1, w_1Rw_2$, risulta $wRw_2$, quindi per ipotesi,
$\forall w_2 \in W. w_1Rw_2 \Rightarrow V(\alpha, w_2) = 1$, quindi $V(\Box \alpha, w_1) = 1$
e perciò $\forall w_1 \in W. wRw_1 \Rightarrow V(\Box \alpha, w_1) = 1$,
dunque $V(\Box \Box \alpha, w) = 1$.
Per l'arbitrarietà di $w$ e dell'S4-modello abbiamo la tesi.
\end{proof}


\begin{theorem}[Teorema di adeguatezza per S5]
$\vdash \alpha \Rightarrow \vDash \alpha$ in $S5$
\end{theorem}

\begin{proof}
Siccome un S5-modello è anche un T-modello e poiché ogni teorema del Sistema T
è anche un teorema del Sistema S5, possiamo sfruttare il teorema di adeguatezza per T
e dimostrare qui solo che l'assioma $S5$ è una formula valida:

Se $(W, R, I)$ è un S5-modello e $w \in W$,

se $V(\diamond \alpha, w) = 1$, allora $\exists w_1 \in W. wRw_1$ e $V(\alpha, w_1) = 1$,

Se $w_2 \in W$ e $wRw_2$, allora siccome $R$ è simmetrica, $w_2Rw$ e poiché
è transitiva $w_2Rw_1$, ma $V(\alpha, w_1) = 1$, quindi
$\exists w_1 \in W. w_2Rw_1$ e $V(\alpha, w_1) = 1$, cioè $V(\diamond \alpha, w_2) = 1$.

Ciò ci permette di affermare che $\forall w_2 \in W. wRw_2 \Rightarrow V(\diamond \alpha, w_2) = 1$
e quindi $V(\Box \diamond \alpha, w) = 1$.
Per l'arbitrarietà di $w$ e dell'S5-modello otteniamo la tesi.
\end{proof}

\section{Proprietà dei sistemi T, S4, S5}
I teoremi di adeguatezza che abbiamo dimostrato ci dicono che ogni teorema
è anche una formula valida per i vari sistemi.
Questo significa che il concetto di semantico di validità che abbiamo poco fa definita
si concilia bene con quello di dimostrabilità. Ciò è una buona assicurazione circa
la bontà delle semantiche.

Il teorema di adeguatezza ci permette di dimostrare varie importanti proprietà tramite
strumenti semantici e che sarebbero difficili da dimostrare usando solo l'apparato deduttivo.

\begin{theorem}
S4 e T sono due sistemi distinti, ovvero l'assioma S4 non è una tesi del sistema T.
\end{theorem}

\begin{proof}
In virtù del teorema di adeguatezza se troviamo un T-modello nel quale l'assioma S4
sia falso, potremo dedurre che S4 non può essere un teorema del sistema T.

Consideriamo 3 mondi $W = \{w_1, w_2, w_3\}$, e definiamo la relazione di accessibilità
in questo modo:
\begin{itemize}
    \item $wRw, \forall w \in W$;
    \item $w_1Rw_2$;
    \item $w_2Rw_3$;
    \item Non ci sono altre relazioni tra mondi.
\end{itemize}

Siccome $A$ è non vuoto, esiste $p \in A$ e definiamo la seguente interpretazione $I:$
\begin{itemize}
    \item Se $x \in A \setminus \{p\}, \forall w \in W. I(x, w) = 0$;
    \item $I(p, w_1) = I(p, w_2) = 1$;
    \item $I(p, w_3) = 0$.
\end{itemize}

È facile vedere che $(W, R, I)$ è un T-modello, infatti $W$ è non vuoto e $R$ è una relazione binaria riflessiva.

Inoltre $V(p, w_1) = V(p, w_2) = 1$ e $V(p, w_3) = 0$, quindi per come è definita
la relazione di accessibilità, $V(\Box p, w_1) = 1$, mentre
$V(\Box p, w_2) = 0$. Quindi $\Box p$ non è vera in tutti i mondi accessibili da $w_1$
e quindi $V(\Box \Box p, w_1) = 0$ e perciò: $V(\Box p \rightarrow \Box\Box p, w_1) = 0$.
Quindi $\Box p \rightarrow \Box\Box p$ non è una formula valida per il sistema T e quindi
non è dimostrabile.
Nel sistema S4 invece è un'istanza dell'assioma (S4) e quindi è banalmente dimostrabile.
Possiamo perciò concludere che i sistemi S4 e T non hanno gli stessi teoremi e quindi non
sono lo stesso sistema formale.
\end{proof}

\begin{corollario}
S5 e T sono due sistemi distinti
\end{corollario}

\begin{proof}
Abbiamo dimostrato in precedenza che l'assioma (S4) è una tesi di S5 e quindi
per il teorema precedente, S5 ha più tesi di T.
\end{proof}

\begin{theorem}
S5 e S4 sono due sistemi distinti, ovvero l'assioma S5 non è una tesi del sistema S4.
\end{theorem}

\begin{proof}

In virtù del teorema di adeguatezza se troviamo un S4-modello nel quale l'assioma S5
sia falso, potremo dedurre che S5 non può essere un teorema del sistema S4.

Consideriamo 2 mondi $W = \{w_1, w_2\}$, e definiamo la relazione di accessibilità
in questo modo:
\begin{itemize}
    \item $wRw, \forall w \in W$;
    \item $w_1Rw_2$;
    \item Non ci sono altre relazioni tra mondi.
\end{itemize}

Siccome $A$ è non vuoto, esiste $p \in A$ e definiamo la seguente interpretazione $I:$
\begin{itemize}
    \item Se $x \in A \setminus \{p\}, \forall w \in W. I(x, w) = 0$;
    \item $I(p, w_1) = 1$;
    \item $I(p, w_2) = 0$.
\end{itemize}

È facile vedere che $(W, R, I)$ è un S4-modello, infatti $W$ è non vuoto e
$R$ è una relazione binaria riflessiva e transitiva.

Inoltre $V(p, w_1) = 1$ e $V(p, w_2) = 0$, quindi per come è definita
la relazione di accessibilità, $V(\diamond p, w_1) = 1$, mentre
$V(\diamond p, w_2) = 0$. Quindi $\diamond p$ non è vera in tutti i mondi accessibili da $w_1$
e quindi $V(\Box \diamond p, w_1) = 0$ e perciò: $V(\diamond p \rightarrow \Box\diamond p, w_1) = 0$.
Quindi $\diamond p \rightarrow \Box\diamond p$ non è una formula valida per il sistema S4 e quindi
non è dimostrabile.
Nel sistema S5 invece è un'istanza dell'assioma (S5) e quindi è banalmente dimostrabile.
Possiamo perciò concludere che i sistemi S5 e S4 non hanno gli stessi teoremi e quindi non
sono lo stesso sistema formale.


\end{proof}

In definitiva abbiamo dimostrato che i sistemi T, S4 e S5 sono tre sistemi distinti
e questo significa che presentano tre nozioni diverse di ``necessità'' e ``possibilità''.

Forniamo, infine quest'ultima dimostrazione:

\begin{theorem}
Esiste $\alpha \in L$, tale che $\alpha \rightarrow \Box \alpha$ non è una tesi di T (e quindi nemmeno di S4 e S5)
\end{theorem}

\begin{proof}

Costruiamo un T-modello in cui $\alpha \rightarrow \Box \alpha$ è falsa.

Consideriamo 2 mondi $W = \{w_1, w_2\}$, e definiamo la relazione di accessibilità
in questo modo:
\begin{itemize}
    \item $wRw, \forall w \in W$;
    \item $w_1Rw_2$;
    \item Non ci sono altre relazioni tra mondi.
\end{itemize}

Siccome $A$ è non vuoto, esiste $p \in A$ e definiamo la seguente interpretazione $I:$
\begin{itemize}
    \item Se $x \in A \setminus \{p\}, \forall w \in W. I(x, w) = 0$;
    \item $I(p, w_1) = 1$;
    \item $I(p, w_2) = 0$.
\end{itemize}


È facile vedere che $(W, R, I)$ è un T-modello, infatti $W$ è non vuoto e
$R$ è una relazione binaria riflessiva.

Inoltre $V(p, w_1) = 1$ e $V(p, w_2) = 0$, quindi per come è definita
la relazione di accessibilità, $V(\Box p, w_1) = 1$, perciò: $V(p \rightarrow \Box p, w_1) = 0$.
Quindi $\diamond p \rightarrow \Box\diamond p$ non è una formula valida per il sistema T e quindi
non è dimostrabile.
Abbiamo ottenuto così la tesi.

\end{proof}

Nei sistemi di logica modale in cui per ogni
$\alpha \in L$ vale $\alpha \leftrightarrow \Box \alpha$ si dice che avviene il collasso
delle modalità. Questi sistemi sono solo una versione barocca, con simboli inutili,
di un sistema formale per il calcolo proposizionale:
i connettivi $\Box$ e $\diamond$ sono inutili, infatti la formula
precedente ci dice che affermare $\Box \alpha$ è esattamente lo stesso che affermare $\alpha$
e quindi non aggiungono niente di nuovo.

Il teorema che abbiamo appena dimostrato ci assicura che nei sistemi definiti non avviene
il collasso delle modalità e che quindi i nuovi connettivi introdotti non sono solo inutili
orpelli che non aggiungono niente a $CP$.

\chapter{Procedure di decisione e teorema di completezza}

\section{Procedura di decisione per T}
Ci occupiamo ora di definira una procedura di decisione per l'insieme delle formule T-valide.

Per verificare la T-validità di una formula $\alpha \in L$ avremmo bisogno di verificare che essa sia vera in tutti
i mondi di tutti i possibili T-modelli, ciò che invece ci proponiamo di fare è
provare a dimostrare che la formula non è T-valida e quindi costruire un T-modello
falsificante per $\alpha$, ossia un T-modello in cui c'è un mondo in cui $\alpha$ sia falsa.
Se ciò risulta impossibile avremo dimostrato che la $\alpha$ è T-valida.

Per rendere il lavoro più semplice, invece che costruire un T-modello
tramite la determinazione diretta della terna $(W, R, I)$, procediamo alla costruzione
progressiva di un diagramma che rappresenti i vari mondi, le relazioni di accessibilità
stabilite tra essi e i valori di verità associati alle formule in ciascun mondo.
I mondi vengono rappresentati da rettangoli (che avranno lo stesso nome dei mondi
da loro rappresentati) dentro i quali
vengono scrivere delle formule a cui vengono associati
dei valori di verità a seconda di come si vuole vengano interpretate nel mondo
corrispondente a quel rettangolo. La relazione di accessibilità è rappresentata
tramite frecce che collegano i rettangoli in modo che se $w_1$ è accessibile da $w_2$,
allora c'è una frecca che parte dal rettangolo di $w_2$ ed è diretta verso il rettangolo
di $w_1$.


Allora procediamo come segue:
Data la formula $\alpha \in L$, iniziamo introducendo un rettangolo $w_1$ e
al suo interno scriviamo $\alpha$ e le assegniamo il valore $0$ scrivendo uno zero
sotto al suo operatore principale.

Dato un qualsiasi rettangolo $w_i$, questo conterrà un numero finito di formule
$\beta_1, ..., \beta_n$
e a ciascuna di essa è associato un valore di verità, allora proviamo
a determinare un assegnamento di valori di verità per le variabili proposizionali
che occorrono nelle formule del rettangolo $w_i$
in modo che tutte le $\beta_1, ..., \beta_n$
abbiano contemporanemanete il valore di verità che le era stato associato.
Per farlo sfruttiamo la definizione di $V$ e la struttura delle varie formule.

Rappresentiamo graficamente questi passaggi scrivendo sotto all'operatore principale
di ogni sottoformula $\gamma_{i,j}$ di ogni formula $\beta_i$
il valore di verità associato a $\gamma_{i,j}$.

Diamo un esempio di questo procedimento provando a trovare un assegnamento
di valori di verità che falsifichi $p \rightarrow q$:

Siccome vogliamo che la formula sia falsa, iniziamo scrivendo $0$ sotto l'operatore principale $\rightarrow$:
$$p \underset{0}{\rightarrow} q$$

Ora, dalla definizione di $\rightarrow$ e dalla definizione di $V$
deduciamo che affinché la formula sia falsa, c'è bisogno
che a $p$ sia assegnato $1$ e a $q$ $0$, e quindi scriviamo:

$$\underset{1}{p} \underset{1}{\rightarrow} \underset{0}{q}$$

Se nel fare ciò si arriva a dover associare alla stessa sottoformula
due valori di verità diversi, allora arrestiamo il processo e sottolineiamo
la sottoformula incriminata.

\textbf{Fornire un altro esempio}

Durante questo procedimento possono presentari vari casi particolari:
Può capitare che nel rettangolo $w_i$ l'operatore principale di una sottoformula $\beta$ sia tale che le regole di $V$
non ci permettono di determinare univocamente dei valori di verità per le sue sottoformule.
Ad esempio ciò succede se vogliamo che $\gamma \lor \delta$ sia vera, in questo caso abbiamo tre possibili
assegnamenti:
\begin{itemize}
    \item $\gamma$ vera  e $\delta$ falsa
    \item $\gamma$ vera  e $\delta$ vera
    \item $\gamma$ falsa e $\delta$ vera
\end{itemize}

In eventualità come questa facciamo tante copie del diagramma (che chiameremo versioni alternative)
quante sono i possibili assegnamenti di verità e di volta in volta, sostituiamo al rettangolo $w_i$ contenente le ambiguità,
un rettangolo $w_i(n)$ detto rettangolo alternativo di $w_i$ in cui alle sottoformule
con valore ambiguo viene associato ciascuno dei possibili assegnamenti.
E infine poniamo un $\dagger$ sotto all'operatore principale di $\beta$ nel rettangolo $w_i$.
Per ciascun diagramma continuiamo questo procedimento separatamente.

Quindi nel caso precedente avremmo tre rettangoli: $w_i(1), w_i(2), w_i(3)$.
Nel primo alle sottoformule $\gamma$ e $\delta$ saranno assegnati rispettivamente i valori $1$ e $0$,
nel secondo i valori $1$ e $1$ e nel terzo i valori $0$ e $1$.

Se invece ci ritroviamo a dover assegnare ad una formula del tipo $\Box \beta$
il valore di verità $0$ nel rettangolo $w_i$, allora disegniamo un asterisco sotto al simbolo $\Box$
e se in $w_i$ a $\beta$ non è stato assegnato il valore di verità $0$,
introduciamo un nuovo rettangolo $w_j$ accessibile da $w_i$ (cioè disegniamo una freccia diretta verso $w_j$ che congiunge $w_i$ a $w_j$)
e in $w_j$ scriviamo la formula $\beta$ e le associamo il valore $0$, così che sia giustificato
l'assegnamento del valore $0$ alla formula $\Box \beta$.
Il rettangolo $w_i$ viene detto \emph{predecessore diretto} del rettangolo $w_j$.

Infine se dobbiamo assegnare ad una formula del tipo $\Box \beta$
il valore di verità $1$ nel rettangolo $w_i$, allora disegniamo un asterisco sopra al simbolo $\Box$
e in ogni rettangolo $w_j$ (diverso da $w_i$)  accessibile da $w_i$ scriviamo la formula $\beta$ e le associamo il valore $1$,
così che risulti giustificato l'assegnamento del valore $1$ alla formula $\Box \beta$,
mentre in $w_i$ associamo a $\beta$ il valore $1$.

Ogni volta che introduciamo un nuovo rettangolo ripetiamo per esso la procedura.

Un diagramma costruito secondo le regole appena descritte è detto T-diagramma.

Inoltre un insieme di T-diagrammi a cui non si può applicare più nessuna regola è detto
un sistema completo di T-diagrammi.

Se in un rettangolo è necessario assegnare ad una sottoformula due valori di verità diversi,
allora diciamo che il rettangolo è esplicitamente inconsistente.
Se un rettangolo esplicitamente inconsistente $w_j$ è accessibile da un rettangolo $w_i$,
allora $w_i$ è esplicitamente inconsistente.

Se un rettangolo ha dei rettangoli alternativi e questi sono tutti esplicitamente inconsistenti,
allora anch'esso è esplicitamente inconsistente.

Per il modo in cui viene costruito un sistema completo di T-diagrammi, è facile dedurre che
$\alpha \in L$ è T-valida se e solo se nel suo sistema completo di T-diagrammi, il rettangolo
$w_1$ è esplicitamente inconsistente.

Non ci resta che dimostrare che la procedura che abbiamo definito è effettivamente
una procedura di decisione, vogliamo cioè dimostrare che per ogni formula di $L$
essa ci permette di determinare in modo univoco e in un numero finito di passi
se questa è T-valida o meno.

Il fatto che il procedimento funzioni e che se il procedimento termina la risposta è univoca
sono ovvi, dobbiamo solo mostrare che per costruire un sistema completo di T-diagrammi
sono necessari solo un numero finito di passi:

Iniziamo osservando che in un rettangolo $w_i$
la determinazione dei valori di verità delle sottoformule della formula $\alpha$,
nel caso in cui non ci siano ambiguità comporta solo un numero finito di passaggi.
Se c'è qualche ambiguità bisogna usare la regola per introdurre i rettangoli alternativi.
Il numero di possibili combinazioni di valori di verità è finito e quindi
i rettangoli alternativi introdotti sono in numero finito.

Notiamo ora che se è necessario introdurre un nuovo mondo $w_j$ accessibile da $w_i$,
allora in questo mondo le formule avranno un numero strettamente inferiore di operatori
modali rispetto al numero di operatori modali nelle formule di $w_i$, questo perché
le formule scritte in $w_j$ si tutte ottengono rimuovendo un operatore da una sottoformula
della formula scritta in $w_i$.
Dal momento che in $w_1$ c'è solo un numero finito di operatori
modali, allora il numero di mondi da dover introdurre sarà necessariamente finito.

Abbiamo perciò che il numero di rettangoli in un sistema completo di T-diagrammi è finito
e che per ciascuno di essi le operazioni da compiere sono finite. In definitiva
la procedura che abbiamo definito termina in un numero finito di passi.

\textbf{Mettere foto esempi}


\section{Procedura di decisione per S4}
Cerchiamo di adattare la procedura appena definita per decidere le formule S4-valide.
Siccome un S4-modello ha una relazione di accessibilità riflessiva e transitiva, allora bisogna stare
attenti che qualora si introducano mondi nuovi le relazioni di accessibilità siano
correttamente registrate in modo che se $w_3$ è accessibile da $w_2$ e $w_2$ è accessibile
da $w_1$, allora $w_3$ risulti accessibile da $w_1$ e quindi bisogna aggiungere il giusto
numero di di frecce per far si che la transitività della relazione di accessibilità sia rispettata
Questa situazione ci porta delle complicazione, per illustrare bene il problema
mostriamo cosa succede con la seguente formula:

$$MLp \rightarrow LMp$$
\textbf{fare il grafico}

Il problema che riscontriamo è che essendo la relazione di accessibilità transitiva,
quando in $w_i$ si ha che $\Box \alpha$ è vera, dobbiamo registrare in ogni altro mondo
accessibile da $w_i$ che $\alpha$ è vera, e in particolare se c'è un mondo $w_j$
accessibile da $w_i$ e un mondo $w_k$ accessibile da $w_j$, allora $w_k$ è accessibile
da $w_i$ e quindi sia in $w_j$ che in $w_k$ dobbiamo registrare che $\alpha$ deve essere vera,
ma ciò significa che ora nei mondi accessibili da un mondo $w_m$ non è più detto
che il numero di operatori modali presenti nelle formule sia diminuito e quindi
può capitare come nell'esempio precedente di avere un numero infinito di mondi.

Introduciamo la seguente definizione:
\begin{definition}
diciamo che i mondi $w_1, ..., w_n$ formano una catena se:
$w_{i+1}$ è accessibile da $w_i$ per ogni $i = 1, ..., n-1$
\end{definition}

Se in un rettangolo $w_{i+1}$ della catena ci sono delle condizioni introdotte
a partire dal mondo $w_i$, allora quelle condizioni devono valere
in ogni altro elemento successivo della catena.

Per ovviare al problema che abbiamo riscontrato, osserviamo che
in ogni rettangolo sono presenti solo
sottoformule ben formate della formula $\alpha$ presente in $w_1$,
siccome tutte le sottoformule possibili di $\alpha$ sono in numero finito,
allora ad un certo punto, quando dovremo aggiungere un nuovo rettangolo $w_i$
al termine di una catena, ci sarà sicuramente un rettangolo più in alto nella catena
che contenga la formula e tutte le condizioni che vogliamo aggiungere $w_i$.
Inoltre notiamo che anche se questo rettangolo contiene più condizioni di $w_i$,
ciò non è un problema, perché essendo $w_i$ accessibile da questo rettangolo,
quelle condizioni valgono anche in $w_i$, facendo parte della stessa catena.
capiterà sicuramente che esista un altro rettangolo già introdotto
con la stessa formula.

I T-diagrammi in cui la relazione tra i rettangoli è transitiva e si usa la regola
appena definita vengono detti S4-diagrammi. Se poi questi particolari T-diagrammi
sono un sistema completo di T-diagrammi, allora si dirà che sono un sistema
completo di S4-diagrammi.


Con questa modifica possiamo dimostrare che il numero di rettangoli in un sistema
completo di S4-diagrammi è finito:

Una catena di rettangoli è finita, per quanto detto prima.
Ogni catena inizia da $w_1$, siccome si introducono rettangoli
solo in corrispondenza di operatori modali, e ogni rettangolo contiene solo un numero finito
di operatori modali, allora il numero di catene necessarie è finito
così come il numero di rettangoli.

\section{Procedura di decisione per S5}
Un S5-diagramma è un S4-diagramma in cui, però, ogni rettangolo è accessibile
da ogni altro rettangolo. Con un argomento molto simile a quello usato
per dimostrare che la costruzione di un sistema completo di S4-diagrammi termina dopo
un numero finito di passi, si può dimostrare lo stesso risultato per gli S5-diagrammi.
Quindi anche per S5 c'è una procedura di decisione per le formule S5-valide.


\section{Teorema di completezza per T}
Data una formula $\alpha$, una volta costruito il suo sistema completo di T-diagrammi,
associamo ad ogni rettango $w_i$ del sistema una formula $w_i'$ definita in questo modo:

nel rettangolo $w_i$ c'è una formula $\beta$ che deve essere falsa ed eventualmente delle condizioni,
cioè un numero finito di formule $\gamma_1, ...\gamma_n$ che devono essere vere, ereditate
dal suo predecessore diretto, allora definiamo $w_i'$ come la formula
$\beta \lor \neg \gamma_1 \lor ... \lor \neg \gamma_n$, con $n \geq 0$.

\begin{flushleft}
\textbf{Lemma 1}
Dato un sistema completo di T-diagrammi e sia $w_i$ un suo rettangolo,
se $\beta$ è una sottoformula di $w_i'$ e a $\beta$ è assegnato univocamente il valore di verità falso
nel rettangolo $w_i$,

allora $\vdash \beta \rightarrow w_i'$ in T

Se invece a $\beta$ è associato il valore di verità vero,


allora $\vdash \neg \beta \rightarrow w_i'$ in T

\textit{Dim.}

Supponiamo che a $\beta$ sia associato il valore di verità falso, l'altro caso è analogo,
allora se $\delta_1, ..., \delta_n$ sono tutte le sottoformule ben formate di $w_i'$ è
facile vedere che
$$(\Box \delta_1 \rightarrow \delta_1) \rightarrow ... \rightarrow (\Box \delta_n \rightarrow \delta_n) \rightarrow (\beta \rightarrow w_i')$$
è ottenuta per sostituzione da una tautologia di $CP$, infatti l'unico modo per rendere
falsa questa formula è assegnare a $\delta_k$ $1$ ogni volta che associamo a $\delta_k$ $1$,
per $k = 1 ... n$, assegnare $1$ a $\beta$ e $0$ a $w_i'$.
Ma sappiamo, per quanto dedotto dal T-diagramma, e per il fatto,
che se $w_i'$ ha valore di verità $1$,
allora $\beta$ deve avere valore di verità $0$.

Quindi otteniamo l'asserto.

\end{flushleft}


\begin{flushleft}
\textbf{Lemma}

Dato un sistema completo di T-diagrammi e sia $w_i$ un suo rettangolo in cui
non ci sono operatori contrassegnati con $\dagger$,
se in $w_i$ c'è un'inconsistenza, allora $\vdash w_i'$ in T.

\textit{Dim.}

Siccome $w_i$ è esplicitamente inconsistente, esiste una sottoformula $\beta$ di $w_i'$
tale che ad essa è associata sia il valore di verità $0$ che il valore di verità $1$,
allora per il lemma precedente si ha:
$$\vdash \beta \rightarrow w_i'$$
e
$$\vdash \neg \beta \rightarrow w_i'$$

È facile provare che
$(\neg \beta \rightarrow \alpha) \rightarrow (\beta \rightarrow \alpha) \rightarrow \alpha$
è una tautologia in $CP$ e quindi è un teorema in $CP$ e dunque è un teorema nel sistema T.

Allora applicando due volte il modus ponens e sfruttando anche i teoremi che abbiamo mostrato poco sopra,
otteniamo che $\vdash w_i'$.

\end{flushleft}


\begin{flushleft}
\textbf{Lemma}

Dato un sistema completo di T-diagrammi e un rettangolo $w_i$, se $\vdash w_i'$
e $w_j$ è il suo predecessore diretto e in $w_j$ non ci sono connettivi contrassegnati
con $\dagger$,
allora $\vdash w_j'$

\textit{Dim.}

Se $w_i$ e $w_j$ sono lo stesso rettangolo, non c'è niente da dimostrare.
Mentre se $w_i$ e $w_j$ sono due rettangoli diversi, siccome $w_i$ è costruito
a partire da $w_j$ secondo le regole della procedura, allora in $w_i$ c'è
una formula $\gamma$ a cui è assegnato il valore $0$ e tale che in $w_j$ c'è
$\Box \gamma$ a cui è assegnato il valore $0$ e poi ci sono delle formule
$\beta_1, ..., \beta_n$ a cui è assegnato il valore $1$ e tali che in $w_j$
a $\Box \beta_1, ..., \Box \beta_n$ è assegnato il valore $1$ in $w_i$.
Per questo motivo $w_j'$ è la formula: $\neg \beta_1 \lor ... \neg \beta_n \lor \gamma$.

Quindi per ipotesi abbiamo:
$$\vdash \neg \beta_1 \lor ... \neg \beta_n \lor \gamma$$
per la regola di necessitazione, otteniamo:
$$\vdash \Box(\neg \beta_1 \lor ... \neg \beta_n \lor \gamma)$$
Applicando la definizione di $\rightarrow$:
$$\vdash \Box(\beta_1 \rightarrow ... \beta_n \rightarrow \gamma)$$
Per applicazioni ripetute dell'assioma $K$:
$$\vdash (\Box \beta_1 \rightarrow ... \Box \beta_n \rightarrow \Box \gamma)$$
E quindi per definizione di $\rightarrow$:
$$\vdash (\neg \Box \beta_1 \lor ... \neg \Box \beta_n \lor \Box \gamma)$$

Siccome in $w_j'$ abbiamo assegnato a $\Box \beta_m$ il valore di verità $1$, per $m = 1...n$,
per il lemma mostrato prima abbiamo che:
$$\vdash \neg \Box \beta_1 \rightarrow w_j'$$
$$...$$
$$\vdash \neg \Box \beta_n \rightarrow w_j'$$

Allora, usando un ragionamento valido in $CP$, otteniamo:
$$\vdash (\neg \Box \beta_1 \lor ... \neg \Box \beta_n \lor \Box \gamma) \rightarrow w_j'$$

Dunque per Modus Ponens si ricava:
$$\vdash w_j'$$
cioè la tesi.

\end{flushleft}

Consideriamo ora il caso in cui un rettangolo contenga delle sottoformule
a cui non è possibile dare un valore di verità univoco e per cui bisogna
usare la regola dei rettangoli alternativi. Se nelle formule rimuoviamo
tutte le occorrenze di connettivi non primitivi, sostituendoli con la loro definizione
in termini di $\neg$ e $\lor$, allora otteniamo che l'unico connettivo
per cui si può presentare una situazione di ambiguità e $\lor$ e per le sue sottoformule
sono possibili tre assegnamenti diversi. Inoltre ricordiamo che un rettangolo a cui è stata applicata
la regola dei rettangoli alternativi è esplicitamente inconsistente se tutti i
rettangoli alternativi sono inconsistenti. Quindi dato un rettangolo
a cui è applicata la regola dei rettangoli alternativi non perdiamo di generalità
se assumiamo che i rettangoli alternativi siano tre. Allora dimostriamo il seguente lemma:

\begin{flushleft}
\textbf{Lemma}
$$\vdash w_i(1)', \vdash w_i(2)', \vdash w_i(3)' \rightarrow \vdash w_i'$$

\textit{Dim.}

Siccome in $w_i$ c'è un'ambiguità sappiamo che c'è una sottoformula in $w_i'$
del tipo $\beta \lor \gamma$ a cui è stato assegnato $1$, quindi vale:
$$\vdash \neg (\beta \lor \gamma) \rightarrow w_i'$$

In $w_i(1)$ a $\beta$ è associato $1$, a $\gamma$ $0$, quindi $w_i(1)'$ è $w_i' \lor \neg \beta \lor \gamma$.

In $w_i(2)$ a $\beta$ è associato $0$, a $\gamma$ $1$, quindi $w_i(2)'$ è $w_i' \lor \beta \lor \neg \gamma$.

In $w_i(3)$ a $\beta$ è associato $1$, a $\gamma$ $1$, quindi $w_i(3)'$ è $w_i' \lor \neg \beta \lor \neg \gamma$.

Perciò, per ipotesi, abbiamo rispettivamente:

$$\vdash w_i' \lor \neg \beta \lor \gamma$$
$$\vdash w_i' \lor \beta \lor \neg \gamma$$
$$\vdash w_i' \lor \neg \beta \lor \neg \gamma$$

Ora usiamo la tautologia di $CP$:
$$\vdash (\neg (\delta \lor \epsilon) \rightarrow \rho) \rightarrow
         (\rho \lor \neg \delta \lor \neg \epsilon) \rightarrow (\rho \lor \delta \lor \neg \epsilon)
           \rightarrow (\rho \lor \neg \delta \lor \epsilon) \rightarrow \rho$$

Sostituendo al posto di $\rho$ la formula $w_i'$, al posto di $\delta$ la formula $\beta$
e al posto di $\epsilon$ la formula $\gamma$.

Dunque abbiamo, sfruttando anche la definizione di $w_i(1)'$, $w_i(3)'$, $w_i(3)'$:
$$\vdash (\neg (\beta \lor \epsilon) \rightarrow w_i') \rightarrow
         (w_i(1)' \rightarrow w_i(2)' \rightarrow w_i(3)' \rightarrow w_i')$$

Allora applicando quattro volte il Modus Ponens, otteniamo $\vdash w_i'$
\end{flushleft}

Questi lemmi che abbiamo dimostrato ci permettono di concludere facilmente
che se in un sistema completo di T-diagrammi per la formula c'è un rettangolo esplicitamente inconsistente,
allora $\vdash w_1'$.

Quindi se $\vDash \alpha$, allora il sistema completo di T-diagrammi per $\alpha$ è inconsistente
e quindi $\vdash w_1'$, ma $w_1' = \alpha$, quindi $\vdash \alpha$.

\section{Teorema di completezza per S4}
Mostriamo ora il teorema di completezza per il sistema S4.
Data una formula $\alpha$, una volta costruito il suo sistema completo di 4-diagrammi,
associamo ad ogni rettango $w_i$ del sistema una formula $w_i''$ definita in questo modo:

nel rettangolo $w_i$ c'è una formula $\beta$ che deve essere falsa ed eventualmente delle condizioni,
cioè un numero finito di formule $\gamma_1, ...\gamma_n$ che devono essere vere, ereditate
dal suo predecessore diretto, allora definiamo $w_i''$ come la formula
$\beta \lor \neg \Box \gamma_1 \lor ... \lor \neg \Box \gamma_n$, con $n \geq 0$.

\begin{flushleft}
\textbf{Lemma}
Dato un sistema completo di S4-diagrammi e sia $w_i$ un suo rettangolo,

allora $\vdash w_i' \rightarrow w_i''$ in T

\textit{Dim.}

La dimostrazione è semplice e si basa sul fatto che vale:
$\vdash \neg \alpha \rightarrow \neg \Box \alpha$ in T

\end{flushleft}

Da questo lemma e dal lemma precedente per T ricaviamo che se un rettangolo $w_i$
di un sistema completo di S4-diagrammi è esplicitamente inconsitente
allora $\vdash w_i''$

Mostriamo ora:

\begin{flushleft}
\textbf{Lemma}

Dato un sistema completo di S4-diagrammi e un rettangolo $w_i$, se $\vdash w_i''$
e $w_j$ è il suo predecessore diretto e in $w_j$ non ci sono connettivi contrassegnati
con $\dagger$,
allora $\vdash w_j''$

\textit{Dim.}

Se $w_i$ e $w_j$ sono lo stesso rettangolo, non c'è niente da dimostrare.
Mentre se $w_i$ e $w_j$ sono due rettangoli diversi, siccome $w_i$ è costruito
a partire da $w_j$ secondo le regole della procedura, allora in $w_i$ c'è
una formula $\gamma$ a cui è assegnato il valore $0$ e tale che in $w_j$ c'è
$\Box \gamma$ a cui è assegnato il valore $0$ e poi ci sono delle formule
$\beta_1, ..., \beta_k$, con $k \geq 0$ a cui è assegnato il valore $1$ e tali che in $w_j$
a $\Box \beta_1, ..., \Box \beta_n$ è assegnato il valore $1$ in $w_i$,
e infine ci sono delle formule $\beta_{k+1}, ..., \beta_{n}$, con $n \geq 0$,  ereditate da qualche
altro rettangolo nella catena a cui appartengono sia $w_j$ che $w_i$.

Quindi per il lemma visto nel paragrafo precedente, abbiamo:
$$\vdash (\neg \Box \beta_1 \lor ... \neg \Box \beta_k \lor \Box \gamma) \rightarrow w_j'$$
Usando il lemma precedente e il teorema di sillogismo otteniamo:
$$\vdash (\neg \Box \beta_1 \lor ... \neg \Box \beta_k \lor \Box \gamma) \rightarrow w_j''$$

Se $n > 0$, siccome $\beta_{k+1}, ..., \beta_{n}$ sono ereditate sia da $w_i$ che da $w_j$,
allora $w_j''$ sarà del tipo $\delta \lor \neg \Box \beta_{k+1} \lor ... \lor \Box \beta_{n}$, per qualche $\delta \in L$,
Quindi poiché $p \rightarrow (p \lor q)$ è una tautologia di $CP$ e quindi un teorema, otteniamo:
$\vdash \neg \Box \beta_{k+1} \lor ... \lor \Box \beta_{n} \rightarrow w_j''$.

e quindi usando un'altra tautologia di $CP$: $(p \rightarrow r) \rightarrow (q \rightarrow r) \rightarrow (p \lor q) \rightarrow r$
e usando due volte il modus ponens, otteniamo:
$$\vdash (\neg \Box \beta_1 \lor ... \neg \Box \beta_n \lor \Box \gamma) \rightarrow w_j''$$

Poiché $\vdash w_i''$, si ha:
$$\vdash (\neg \Box \beta_1 \lor ... \neg \Box \beta_n \lor \gamma)$$
Per necessitazione, definizione di $\rightarrow$ e per l'assioma $K$ otteniamo:
$$\vdash (\neg \Box \Box \beta_1 \lor ... \neg \Box \Box \beta_n \lor \Box \gamma)$$
Ora sfruttando il fatto che in S4 si ha: $\Box \Box p \leftrightarrow \Box p$, otteniamo:
$$\vdash (\neg \Box \beta_1 \lor ... \neg \Box \beta_n \lor \Box \gamma)$$
Dunque poiché come abbiamo visto:
$$\vdash (\neg \Box \beta_1 \lor ... \neg \Box \beta_n \lor \Box \gamma) \rightarrow w_j''$$
Per modus ponens:
$$\vdash w_j''$$

\end{flushleft}

Infine, con qualche piccola accortezza si può giungere anche qui a dimostrare il lemma:
\begin{flushleft}
\textbf{Lemma}
$$\vdash w_i(1)', \vdash w_i(2)', \vdash w_i(3)' \rightarrow \vdash w_i'$$
\end{flushleft}

E perciò come prima otteniamo il nostro teorema di completezza.

\section{Teorema di completezza per S5}
Per il sistema S5 possiamo sfruttare quanto fatto finora per il sistema S4,
l'unico inconveniente è che $w_1''$ non è la formula che ci aspettiamo,
infatti poiché la relazione di accessibilità deve essere anche riflessiva,
allora $w_1''$ sarà del tipo: $\alpha \lor \neg \Box \beta_1 \lor ... \lor \neg \Box \beta_n$,
con $n \ge 0$.
Nel caso in cui $n > 0$, dobbiamo aggiungere dei passaggi per poter ottenere $\vdash \alpha$.
Se $n > 0$, allora nel rettangolo $w_1$ c'è una formula $\beta_1$ che è stata scritta
perché c'è un rettangolo $w_j$,
in cui c'è una sottoformula del tipo $\Box \beta_1$
a cui è stata assegnato il valore di verità $1$ e dunque a $\beta_1$ è assegnato il valore di verità $1$
La formula $w_j''$ associata a $w_j$ è del tipo $\gamma \lor \neg \Box \delta_1 \lor ... \lor \neg \Box \delta_k$, con $k \ge 0$,
e per come sono costruiti gli S5-diagrammi, $\Box \beta_1$ può essere o una sottoformula di $\gamma$
o una sottoformula di qualche $\delta_i$, con $i < n$

Se $\Box \beta_1$ è una sottoformula di $\gamma$, allora per un argomento simile al lemma $1$ abbiamo che:
$$\vdash \neg \Box \beta_1 \rightarrow \gamma$$
e quindi per necessitazione e grazie all'assioma $K$:
$$\vdash \Box \neg \Box \beta_1 \rightarrow \Box \gamma$$

Ma per l'assioma S5:
$$\vdash \neg \Box \beta_1 \rightarrow \Box \neg \Box \beta_1$$
Quindi per sillogismo:
$$\vdash \neg \Box \beta_1 \rightarrow \Box \gamma$$

Ma per come sono costruiti gli S5-diagrammi, o $w_j$ è $w_1$ e $\gamma$ è $\alpha$,
oppure $w_j$ ha un predecessore diretto
che chiamiamo $w_l$ e in $w_l$ a $\Box \gamma$ è associato il valore di verità $0$.

Se $\Box \beta_1$ è una sottoformula di $\delta_i$, con $i < n$,
allora, per un argomento simile al lemma $1$
$$ \vdash \neg \Box \beta_1 \rightarrow \neg \delta_i$$
e poiché
$$ \vdash \neg \delta_i \rightarrow \neg \Box \delta_i$$
e per sillogismo:
$$ \vdash \neg \Box \beta_1 \rightarrow \neg \Box \delta_i$$

Ma per come sono costruiti gli S5-diagrammi, c'è un rettangolo $w_l$ in cui a $\Box \delta_i$
è stato associato $1$.

Se abbiamo ottenuto $\neg \Box \beta_1 \rightarrow \alpha$, ci fermiamo,
altrimenti, usando la formula, che chiameremo $\sigma$, implicata da $\neg \Box \beta_1$
($\Box \gamma$ o $\neg \Box \delta_i$, a seconda dei casi) reiteriamo questo procedimento
e otteniamo che è un teorema di S5 che $\sigma$ uno dei disgiunti della formula $w_l''$, e quindi
per sillogismo $\neg \Box \beta_1$ implica uno dei disgiunti della formula $w_l''$
Osserviamo, però, che ad ogni iterazione di questo procedimento la lunghezza
della formula implicata da $\neg \Box \beta_1$ aumenta sempre di più, inoltre ogni formula
è sempre una delle formule scritta nell'S5-diagramma che sono in numero finito,
quindi non è possibile dover iterare questo procedimento all'infinito, ma prima o poi
bisognerà arrivare in un numero finito di passi alla formula $\alpha$ che è la formula
più lunga di tutte nell'S5-diagramma.

Abbiamo perciò dimostrato che $\vdash \neg \Box \beta_1 \rightarrow \alpha$,
è possibile fare lo stesso per tutti gli altri $\beta_i$, con $i = 2, ..., n$.
Dunque abbiamo
$$\vdash \neg \Box \beta_1 \rightarrow \alpha$$
$...$
$$\vdash \neg \Box \beta_n \rightarrow \alpha$$

e inoltre:

$$\vdash \neg \Box \beta_1 \lor ... \neg \Box \beta_n \lor \alpha$$

Dunque sfruttando svariate volte la tautologia di $CP$:
$$(p \rightarrow r) \rightarrow (p \lor q) \rightarrow (r \lor q)$$
e il modus ponens, otteniamo:
$$\vdash \alpha \lor ... \lor \alpha$$
che è facile far vedere che ci porta a poter affermare che
$$\vdash \alpha$$

\chapter*{Conclusioni}
\addcontentsline{toc}{chapter}{Conclusioni}
Abbiamo inizialmente introdotto tre sistemi di logica modale: T, S4 ed S5,
dicendo che non era chiaro se ci fosse un sistema da preferire agli altri.
Dopo aver introdotto delle semantiche di Kripke per essi, abbiamo mostrato
come queste semantiche rispecchino perfettamente l'apparato deduttivo dei sistemi formali.
Ora possiamo usare i concetti introdotti da queste semantiche per fare alcune riflessioni
circa la natura dei tre sistemi introdotti. Infatti è emerso dallo studio che abbiamo condotto
che i tre sistemi rappresentano connotazioni diverse del concetto di ``necessità''
che abbiamo in mente.
Ad esempio nel sistema S5, abbiamo visto grazie alla semantica, che è possibile
determinare che una proposizione è necessaria solo quando questa è vera in tutti i mondi
possibili. Mentre in S4 e in T, ciò non è detto, infatti in questi sistemi una proposizione
è necessaria in un mondo quando è vera in tutti i mondi accessibili da quel mondo.
Questa differenza di definizioni mette in mostra una proprietà importante della necessità,
infatti non è detto che una persona sia in grado di concepire tutti i mondi possibili,
ma la nostra capacità di concepire stati di cose diversi è sicuramente condizionata dal mondo
in cui viviamo. Proprio su quest'osservazione possiamo indagare sulla la distinzione tra T e S4,
infatti in S4 quando si concepisce un mondo alternativo è possibile concepire quel mondo
come se ci vivessimo dentro, e quindi è per noi possibile concepire tutti i mondi che sarebbero
concepibili se vivessimo in quel mondo. Al contrario in T non è così, concepire un mondo
significa solo pensare ad un mondo alternativo, senza che questo influisca sulla nostra capacità
di concepire mondi alterantivi.

In definitiva ciò che viene messo in evidenza dalle semantiche di Kripke per questi sistemi è
che il concetto di necessità è in un certo senso collegato al concetto di conoscenza, più
è possibile conoscere mondi alternativi, più sarà forti saranno le proposizioni che possiamo
affermare essere vere.

Questo fatto ci spinge a pensare alla pluralità di sistemi modali possibili
come a una fonte di ricchezza di sfumature del concetto di ``necessità'' e perciò non ha molto
senso chiedersi quale sia il sistema formale migliore, mentre ha più senso usare tali sistemi
come strumento per un'analisi più approfondita per distinguere le varie sfumature del concetto
di ``necessità'' che si usano nei ragionamenti.
Alcuni esempi di questo approccio sono forniti da Lemmon [Lemmon] e terminiamo riportandone qui alcuni:
Se vogliamo interpretare $\Box \alpha$ come ``è analiticamente vero che $\alpha$'', allora il sistema
formale S5 è quello corretto. Mentre se vogliamo interpretare $\Box \alpha$ come
è ``informalmente dimostrabile in matematica che $\alpha$'', allora il sistema corretto è S4.

\chapter*{Ringraziamenti}
\addcontentsline{toc}{chapter}{Ringraziamenti}
Ringrazio solo me stesso e nessun altro.

\begin{thebibliography}{5}
\bibitem{IntroModale} G.E. Hughes, M.J. Cresswell, \textit{Introduzione alla logica modale},
edizione a cura di C. Pizzi, [1973], Milano, Il Saggiatore, $1983^{2}$.
\bibitem{Kleene} S.C. Kleene, \textit{Mathematical Logic}, New York, J. Wiley \& Sons, 1967.
\bibitem{Mendelson} E. Mendelson, \textit{Introduction to Mathematical Logic}, [1964], Londra, Chapman \& Hall, $1997^{4}$.
\bibitem{Tortora} R. Tortora, \textit{Elementi di teoria degli insiemi}, Napoli, E.DI.S.U. - Napoli 1, 1989.
\end{thebibliography}

\end{document}
