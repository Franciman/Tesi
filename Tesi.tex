\documentclass[a4paper, 12pt]{article}
\usepackage[utf8]{inputenc}
\usepackage{amssymb}
\newtheorem{axiom}{Assioma}
\newtheorem{theorem}{Teorema}
\newtheorem{lemma}{Lemma}
\newtheorem{definition}{Definizione}

% Ambiente per le dimostrazioni
\newenvironment{proof}
    {\textit{Dim.}
    }
    {\begin{flushright}$\bullet$\end{flushright}
    }

% Ambiente per le dimostrazioni formali formali
\newenvironment{formal_proof}
    {\begin{proof}

    % \begin{tabular} {c c|c}
    }
    {%\end{tabular}

    \end{proof}
    }

\begin{document}

\section{Logica proposizionale}
In questo paragrafo introdurremo brevemente un sistema formale per la logica proposizionale classica.
Il sistema che introduciamo sarà usato come base per la costruzione di ulteriori sistemi formali.
Esso sarà molto minimale, infatti conterrà pochi connettivi di base
e tutti gli altri saranno definiti successivamente a partire da questi.

Chiameremo il sistema che stiamo definendo CP (calcolo proposizionale).

Il primo passaggio è quello di definire un alfabeto. Usiamo una piccola scorciatoia,
infatti supponiamo che ci sia dato a priori un insieme numerabile A i cui elementi saranno detti atomi.
Nella pratica questo insieme conterrà i simboli da usare per le variabili che appaiono nelle formule.
In virtù di questa interpretazione, chiameremo gli elementi di A \textit{variabili proposizionali}.

\begin{definition}
L'alfabeto $\Sigma$ è dato dall'insieme $A \cup \{\neg, \vee, (, ) \}$
\end{definition}

Passiamo ora alla definizione dell'insieme delle formule ben formate.

\begin{definition}
L'insieme delle formule ben formate (fbf) $L$ è il più piccolo sottoinsieme dell'insieme tutte le successioni
di elementi di $\Sigma$ determinato dalla seguente definizione induttiva:
\begin{itemize}
\item Se $p \in A$, allora $p \in L$;
\item Se $\alpha \in L$, allora $\neg (\alpha) \in L$;
\item Se $\alpha \in L$ e $\beta \in L$, allora $(\alpha) \vee (\beta) \in L$;
\item $L$ non contiene altri elementi.
\end{itemize}
\end{definition}

Possiamo ricavare, ora, i connettivi che mancano come segue:
\begin{definition}
Se $\alpha, \beta \in L$,
$$(\alpha) \rightarrow (\beta) := (\neg (\alpha)) \vee (\beta)$$
$$(\alpha) \wedge (\beta) := \neg ((\neg (\alpha)) \vee (\neg (\beta)))$$
\end{definition}

Per semplificare la scrittura e rendere più comprensibili le formule che scriveremo
nel seguito, spesso ometteremo le parentesi e per interpretare correttamente
le formule scritte si useranno le seguenti regole di precedenza tra operatori
e le regole di associatività.

\begin{definition}
Gli operatori binari $\wedge,\vee$ sono associativi a sinistra,
mentre $\rightarrow$ è associativo a destra.
Per determinare l'ordine di precedenza tra due operatori si fa uso della
seguente tabella, usando il seguente criterio:
``Gli operatori che si trovano più in alto hanno precedenza maggiore rispetto
agli operatori che si trovano più in basso''
\begin{center}
\begin{tabular} {c c}
    $\neg$ \\
    $\wedge$ \\
    $\vee$ \\
    $\rightarrow$ & $\leftrightarrow$
\end{tabular}
\end{center}
\end{definition}

Di conseguenza la formula $\alpha \rightarrow \beta \rightarrow \neg \gamma$
dovrà essere interpretata come:
$(\alpha) \rightarrow ((\beta) \rightarrow (\neg (\gamma)))$.

Passiamo infine alla determinazione di un sottoinsieme di formule ben formate
che fungeranno da assiomi e delle regole di deduzione
per poter così definire l'apparato deduttivo del calcolo proposizionale.

\begin{definition}
Gli assiomi del calcolo proposizionale sono ottenuti dai seguenti schemi
sostituendo in modo uniforme formule ben formate al posto di $\alpha, \beta, \gamma$:
\begin{itemize}
\item (HPD) $\alpha \rightarrow (\beta \rightarrow \alpha)$
\item (HPMP) $(\gamma \rightarrow (\alpha \rightarrow \beta)) \rightarrow (\gamma \rightarrow \alpha) \rightarrow (\gamma \rightarrow \beta)$
\item ($\vee$-I1) $\alpha \rightarrow (\alpha \vee \beta)$
\item ($\vee$-I2) $\beta \rightarrow (\alpha \vee \beta)$
\item ($\vee$-E) $(\alpha \rightarrow \gamma) \rightarrow (\beta \rightarrow \gamma) \rightarrow (\alpha \vee \beta \rightarrow \gamma)$
\item ($\wedge$-I) $\alpha \rightarrow \beta \rightarrow \alpha \wedge \beta$
\item ($\wedge$-E1) $\alpha \wedge \beta \rightarrow \alpha$
\item ($\wedge$-E2) $\alpha \wedge \beta \rightarrow \beta$
\item ($\neg$-I) $(\alpha \rightarrow \beta) \wedge (\alpha \rightarrow \neg \beta) \rightarrow \neg \alpha$
\item (TER) $\alpha \vee \neg \alpha$
\end{itemize}
\end{definition}

Ricordiamo le due seguenti definizioni:

Ricordiamo che se esiste una dimostrazione della formula $\alpha \in L$,
ovvero, se $\alpha$ è un teorema, scriviamo $\vdash \alpha$.
Descriviamo ora le due regole di deduzione del nostro sistema.

La prima regola è il classico \textbf{Modus Ponens}:
$$\vdash \alpha \rightarrow \beta, \vdash \alpha \Rightarrow \vdash \beta$$

La seconda regola non è strettamente necessaria, ma sarà particolarmente utile per riutilizzare
teoremi già dimostrati, essa è detta \textbf{Regola di sostituzione}:
Se $\vdash \alpha$, allora, sostituendo uniformenente (ovvero sempre allo stesso modo)
ogni occorrenza di una variabile proposizionale all'interno di $\alpha$
con una formula ben formata, si ottiene ancora un teorema.

Come ultima cosa dimostriamo il seguente teorema che riutilizzeremo spesso nel seguito

\begin{flushleft}
\textbf{Teorema(Sillogismo)}
$\vdash (\alpha \rightarrow \beta) \rightarrow
        (\beta \rightarrow \gamma) \rightarrow (\alpha \rightarrow \gamma)$ in $CP$

\textit{Dim.}
Ricordiamo che per $CP$ vale il teorema di deduzione, ovvero:
$$\Gamma,\alpha \vdash \beta \Leftrightarrow \Gamma \vdash \alpha \rightarrow \beta$$
Quindi semplifichiamo il lavoro dimostrando:
$$(\alpha \rightarrow \beta),(\beta \rightarrow \gamma),\alpha \vdash \gamma$$

(1) $\alpha \rightarrow \beta$ ; ipotesi \\
(2) $\alpha$ ; ipotesi \\
(3) $\beta$ ; Modus Ponens di (1) e (2) \\
(4) $\beta \rightarrow \gamma$ ; ipotesi \\
(5) $\gamma$ ; Modus Ponens di (3) e (4)

Ora applicando tre volte il teorema di deduzione otteniamo:
$$\vdash (\alpha \rightarrow \beta) \rightarrow (\beta \rightarrow \gamma) \rightarrow (\alpha \rightarrow \gamma)$$

\begin{flushright}
$\bullet$
\end{flushright}
\end{flushleft}


\section{Logica modale}
Ora che abbiamo definito un sistema formale per la logica proposizionale,
lo estenderemo in vari modi.

Il nostro obiettivo è avere un sistema formale nel quale possiamo definire e studiare
le proprietà di due nuovi operatori unari che vorremmo avessero un comportamento che rispecchi
i concetti di ``necessità'' e ``possibilità''.

Prima di cominciare facciamo delle osservazioni importanti.
Ricordiamo che un operatore verofunzionale è una funzione il cui risultato
è univocamente determinato dal valore di verità dei parametri da cui dipende.
Inoltre, si può dimostrare che nel calcolo proposizionale si possono esprimere
tutti gli operatori verofunzionali possibili in termini di $\neg e \vee$.
Quindi, se gli operatori che vogliamo aggiungere fossero verofunzionali, potremmo
continuare il nostro studio usando $CP$ e non servirebbe altro.
Notiamo, però, che gli operatori che vogliamo aggiungere non possono essere verofunzionali,
infatti il valore di verità della frase ``è necessario che p'' non dipende \textit{solo}
dal valore di verità che si associ, infatti, il fatto che p sia vera, non ci permette
di concludere che p sia necessariamente vera. Quindi per avere una speranza di formalizzare
i concetti di necessità e possibilità siamo costretti ad estendere il sistema $CP$
in maniera sostanziale.

A tal proposito, non è esattamente chiaro come caratterizzare formalmente
le nozioni di ``necessità'' e ``possibilità'', e anzi vedremo che queste potranno essere
intese in modi diversi.
Questa ricchezza di significati si rifletterà in una varietà di sistemi formali più o meno potenti.
Noi ci focalizzeremo in particolare su quattro sistemi formali:
\begin{itemize}
\item Sistema T
\item Sistema S4
\item Sistema S5
\end{itemize}

Questi sistemi differiscono tra di loro soltanto per gli assiomi scelti in ognuno di essi.
Quindi iniziamo definendo la parte comune di tutti i sistemi formali, ovvero il linguaggio
e le regole di deduzione.

L'alfabeto dei sistemi è ottenuto dall'alfabeto del sistema $CP$ aggiungendo il simbolo $\Box$,
che vorremo far corrispondere alla nozione di ``necessità''.

Per quanto riguarda le formule ben formate, la definizione induttiva che genera l'insieme $L$
delle formule ben formate è ottenuta dalle regole usate per definire $L$ in $CP$ più
la seguente regola aggiuntiva:
\begin{itemize}
\item Se $\alpha \in L$, allora $\Box (\alpha) \in L$
\end{itemize}

Diamo la seguente definizione, il cui intento è esprimere l'altro operatore
che ci interessa, quello corrispondente alla nozione di ``possibilità'', in termini di $\Box$.

\begin{definition}
Se $\alpha \in L$:
$$\diamond (\alpha) := \neg \Box \neg (\alpha)$$
\end{definition}

Infine, per quanto riguarda le regole di deduzione, alle due di $CP$ aggiungiamo
una nuova regola, detta \textbf{Regola di necessitazione}:
$$\vdash \alpha \Rightarrow \vdash \Box \alpha$$

Una volta definita la parte comune a tutti i sistemi a cui siamo interessati,
passiamo ora ad esaminare per ogni sistema gli assiomi che lo caratterizzano.

\section{Sistema T}
Il sistema T ha come assiomi tutti gli assiomi di $CP$ più:
\begin{itemize}
\item (K) $\Box (\alpha \rightarrow \beta) \rightarrow (\Box \alpha \rightarrow \Box \beta)$
\item (T) $\Box \alpha \rightarrow \alpha$
\end{itemize}

\section{Sistema S4}
Il sistema S4 ha come assiomi tutti gli assiomi del Sistema T più il seguente assioma:
$$(S4) \Box \alpha \rightarrow \Box \Box \alpha$$

\section{Sistema S5}
Il sistema S5 ha come assiomi tutti gli assiomi del Sistema T più:
$$(S5) \diamond \alpha \rightarrow \Box \diamond \alpha$$


Ora che abbiamo definito il linguaggio e l'apparato deduttivo di tutti i sistemi
che ci interessano, facciamo delle osservazioni su di essi.
\begin{definition}
Diremo che un sistema formale è meno forte di un altro se tutte i teoremi del primo
sono teoremi anche del secondo.
\end{definition}

Ovviamente, per come abbiamo costruito i nostri sistemi formali, il Sistema T è meno forte
sia di S4 che di S5.
Ora, però mostreremo che anche S4 è meno forte di S5. A tal fine premettiamo alcuni risultati
preliminari.

\begin{flushleft}
\textbf{Teorema}
$\vdash \alpha \rightarrow \diamond \alpha$ in $T$

\textit{Dim.}

(1) $\Box \neg \alpha \rightarrow \neg \alpha$ ; $T[\neg \alpha/\alpha]$ \\
(2) $\alpha \rightarrow \neg \Box \neg \alpha$ ; Contrapposta di (1) \\
(3) $\alpha \rightarrow \diamond \alpha$ ; Definizione di $\diamond$

\begin{flushright}
$\bullet$
\end{flushright}
\end{flushleft}

\begin{flushleft}
\textbf{Lemma 1}
$\vdash \diamond \Box x \rightarrow \Box x$ in $S5$

\textit{Dim.}

(1) $\diamond \neg x \rightarrow \Box \diamond \neg x$ ; $S5[\neg x/\alpha]$ \\
(2) $\neg \Box \diamond \neg x \rightarrow \neg \diamond \neg x$ ; contrapposta di (1) \\
(3) $\diamond \neg \diamond \neg x \rightarrow \neg \diamond \neg x$ ; definizione di $\diamond$ \\
(4) $\diamond \Box x \rightarrow \Box x$ ; definizione di $\diamond$ e $\Box$

\begin{flushright}
$\bullet$
\end{flushright}
\end{flushleft}

\begin{flushleft}
\textbf{Teorema 1}
$\vdash \Box x \rightarrow \Box \Box x$ in $S5$, ovvero il Sistema S5 è più forte del Sistema S4.

\textit{Dim.}

(1) $\Box x \rightarrow \diamond \Box x$ ; Teorema \\
(2) $(\Box x \rightarrow \diamond \Box x) \rightarrow (\diamond \Box x \rightarrow \Box\diamond\Box x) \rightarrow (\Box x \rightarrow \Box \diamond \Box x)$ ; Teorema \\
(3) $\diamond x \rightarrow \Box \diamond x$ ; Assioma S5 \\
(4) $\Box x \rightarrow \Box \diamond \Box x$ ; Due volte Modus Ponens usando (1) (2) e (3) \\
(5) $\diamond \Box x \rightarrow \Box x$ ; Lemma 1 \\
(4) $\Box (\diamond \Box x \rightarrow \Box x)$ ; Necessitazione (5) \\
(5) $\Box(\diamond\Box x \rightarrow \Box x) \rightarrow \Box \diamond \Box x \rightarrow \Box \Box x$ ; Assioma K \\
(6) $\Box\diamond\Box x \rightarrow \Box\Box x$ ; Modus Ponens (4) e (5) \\
(7) $(\Box x \rightarrow \diamond \Box x) \rightarrow (\Box\diamond\Box x \rightarrow \Box\Box x) \rightarrow (\Box x \rightarrow \Box\Box x)$ ; Teorema \\
(8) $\Box x \rightarrow \Box \Box x$ ; Due volte Modus Ponens usando (1) (5) e (6)

\begin{flushright}$\bullet$\end{flushright}
\end{flushleft}

\section{Un teorema di deduzione (DA AGGIUSTARE)}
In questo paragrafo proviamo a derivare un teorema di deduzione per il Sistema $S4$.
Non possiamo aspettarci che il teorema di deduzione per il calcolo proposizionale $CP$ valga,
infatti la regola di deduzione di necessitazione ci porterebbe ad asserire che:
$$ x \vdash \Box x \Rightarrow \vdash x \rightarrow \Box x $$
E ciò vorrebbe dire che le modalità non aggiungono nulla al sistema del calcolo proposizionale.

Se guardiamo attentamente la regola di necessitazione e la scriviamo in questo modo:
$$x \vdash \Box x$$
possiamo notare che l'ipotesi $x$ è sufficiente per dimostrare $\Box x$ che afferma molto più
che semplicemente $x$.
Notiamo anche che la regola di necessitazione ci permette di provare
$$x \vdash \Box \Box x$$

Quindi, far semplicemente passare la x da sinistra a destra, significherebbe affermare che
da ipotesi più deboli si possono comunque dedurre le stesse conclusioni. Allora possiamo provare,
nello spostare la x da sinistra a destra di $\vdash$, ad aggiungere un $\Box$.
Questo può bastare in sistemi come $S4$ o più potenti, perché da $\Box x$ possiamo dedurre
$\Box \Box x$, $\Box \Box \Box x$, ecc...
Mentre nel sistema T questo non è vero e quindi dovremo risolvere diversamente.

Recuperiamo quindi un teorema di deduzione, facendo la seguente modifica:
$$\alpha \vdash \beta \Rightarrow \vdash \Box \alpha \rightarrow \beta$$
Ora il precedente problema è facilmente risolvibile, infatti, la deduzione $x \vdash \Box x$
diventa il teorema $\vdash \Box x \rightarrow \Box x$ banalmente valido.
Otteniamo però anche il seguente teorema, $\vdash \Box x \rightarrow \Box \Box x$ che è l'assioma $S4$
Ecco perché abbiamo supposto di trovarci nel sistema $S4$.

Diamo una dimostrazione di

\begin{flushleft}
\textbf{Teorema}
$\alpha \vdash \beta \Leftrightarrow \vdash \Box \alpha \rightarrow \beta$ in $S4$

\textit{Dim.}
Lasciata al lettore, è una banale modifica della dimostrazione nel caso del calcolo proposizionale.
L'unica parte interessante è quella riguardante i teoremi ottenuti con la regola di necessitazione.
In quel caso dobbiamo far uso dell'assioma $S4$ ed è tutto banale.
\end{flushleft}



\section{Modalità e funzioni modali}
Rendiamo più formale il concetto di modalità.
\textbf{Definizione}
Una modalità è una successione di 0 o più di uno dei seguenti simboli: $\neg,\diamond,\Box$.

Viene da domandarci, ora, quante sono le modalità distinte che ci sono in ognuno dei sistemi che abbiamo considerato.
Si può dimostrare che nel Sistema $T$ ce ne sono infinite,
mentre in $S4$ e $S5$ ce ne sono solo un numero finito.

Parleremo più approfonditamente di tali questioni solo dopo aver introdotto una semantica
per il nostro linguaggio.

\begin{flushleft}
\textbf{Definizione 1}

Una funzione modale è una formula contenente almeno una funzione modale
\end{flushleft}

In $T$ ovviamente c'è un numero infinito di funzioni modali esprimibili.
In $S5$ ce n'è un numero finito, mentre sorprendentemente,
in $S4$ ce ne sono infinite, nonostante le modalità esprimibili siano in numero finito.

\section{Una semantica per i sistemi di logica modale}
Dopo aver definito il linguaggio comune a tutti i sistemi di logica modale introdotti,
lo abbiamo usato unicamente come base di un calcolo che usa le regole di deduzione per costruire
dimostrazioni. Come ben sappiamo un linguaggio può anche essere usato per parlare di qualcosa,
di oggetti esterni al sistema formale.
Fare ciò significa associare ad ogni formula ben formata del linguaggio un oggetto
che rappresenti il suo significato.
Quando seguiamo questo procedimento diciamo che abbiamo fornito una semantica per il linguaggio.

Definire semantiche per i sistemi formali ci aiuta sia a confrontare sistemi formali
con strutture maggiormente conosciute e ad assicurarci
che alcune costruzioni corrispondano alla nostra intuizione,
sia a determinare alcune proprietà del sistema formale. E nel nostro caso ci aiuterà anche
a comprendere meglio quali siano le nozioni di modali rappresentate in ciascuno dei sistemi
che abbiamo definito e come differiscono tra di loro.


Come abbiamo fatto per l'apparato deduttivo, prima di dare una semantica ai nostri sistemi,
partiamo ricordando come è fatta la semantica classica per $CP$.

Ricordiamo che un'interpretazione $I$ è una funzione dall'insieme $A$ delle variabili proposizionali
all'insieme $\{V, F\}$.

Data un'interpretazione $I$ definiamo per ricorsione la funzione di valutazione $V_I$ che associa ad ogni formula
del linguaggio di $CP$ un valore dell'insieme $\{V, F\}$ come segue:

Se $\alpha, \beta \in L$:
\begin{itemize}
\item $V_I(p) = I(p)$, $\forall p \in A$
\item $V_I(\alpha \vee \beta) = F \Leftrightarrow V_I(\alpha) = F$ e $V_I(\beta) = F$
\item $V_I(\neg \alpha) = F \Leftrightarrow V_I(\alpha) = V$
\end{itemize}

\begin{definition}
Sia $\alpha \in L$, diremo che $\alpha$ è valida e scriveremo $\vDash \alpha$ se
per ogni interpretazione $I$, $V_I(\alpha) = V$.
\end{definition}

L'idea alla base di questa semantica per $CP$ è di astrarre sul concetto di proposizione
e di tenerne solo il suo valore di verità, così si risolve il problema di definire
formalmente cos'è una proposizione e invece di associare ``Michele è altisonante''
alla variabile proposizionale $p$, le associamo solo $V$ o $F$ a seconda che nel mondo
in cui vogliamo interpretare Michele sia altisonante o meno.
Quindi un'interpretazione ci dà, in un certo senso, una lista di tutte le proposizioni
vere nella realtà di cui vogliamo parlare usando il linguaggio di $CP$.
Usiamo, poi, questi valori assegnati dall'interpretazione per determinare il valore di verità delle formule
in base alla semantica che abbiamo assegnato a ogni operatore (che sono date dalle regole
che abbiamo usato per definire la funzione $V_I$).

Ricordiamo che vale il seguente teorema, di fondamentale importanza,
che mette in relazione i due aspetti dei linguaggi, quello semantico e quello sintattico
$$\vdash \alpha \Leftrightarrow \vDash \alpha$$

Per i nostri sistemi di logica modale possiamo pensare di fare qualcosa di simile,
però non possiamo aspettarci che $V_I$ per formule del tipo $\Box \alpha$ si basi solo
sul valore di $V_I(\alpha)$.

La semantica che useremo è in gran parte dovuta a Saul A. Kripke (vale la pena notare
che pubblicò un teorema di completezza per i sistemi di logica modale all'età di 17 anni)
e si basa su un concetto già introdotto da Leibniz, quello dei mondi possibili.
Quindi non supponiamo più che esista un'unica realtà, ma che esistono tanti mondi possibili,
e in più, e questa è una delle idee chiave della semantica di Kripke, da alcuni mondi
si può accedere ad altri mondi, cioè sapere come sono fatte altre realtà alternative.
Il mondo in cui si può accedere agli altri mondi sarà la base per distinguire
tra i vari concetti di necessità e tra i vari sistemi formali che abbiamo definito prima.

\section{Semantica per il Sistema T}
Definiamo un T-modello come una tripla $(w, W, R, I)$, in cui:
\begin{itemize}
\item $W$ è un insieme non vuoto i cui elementi saranno chiamati mondi;
\item $w$ è un elementi di $W$;
\item $R$ è una relazione binaria riflessiva su $W$, detta relazione di accessibilità;
\item $I$ è una funzione dall'insieme $L \times W$ all'insieme $\{V, F\}$;
\end{itemize}

L'insieme $W$ è l'insieme di tutti i mondi.
$w$ è un mondo particolare tra tutti i mondi possibili, potremmo interpretarlo
come il mondo in cui vive chi deve verificare la validità di una proposizione.
$R$, invece è la relazione binaria
che ci permette di determinare a quali mondi si può accedere da un determinato mondo.
Quindi se $w_1, w_2 \in W e w_1 R w_2$, allora una persona in $w_1$ potrà accedere anche
al mondo $w_2$. Il fatto che $R$ sia riflessiva ci dice che l'unica garanzia che abbiamo
è che ogni persona può accedere al proprio mondo, e questa è una garanzia molto scarsa.
Infine, $I$ ci fornirà per ogni mondo la lista di proposizione vere
e quella delle proposizioni false, analogamente al suo corrispettivo nella semantica per $CP$.

Definiamo come prima una funzione di valutazione $V$ che dipenderà da un dato T-modello
$(w, W, R, I)$.

Se $\alpha, \beta \in L$:
\begin{itemize}
\item $V(p) = I(p, w)$, $\forall p \in A$
\item $V(\neg \alpha) = V \Leftrightarrow V(\alpha) = F$
\item $V(\alpha \vee \beta) = F \Leftrightarrow V(\alpha) = F$ e $V(\beta) = F$
\item $V(\Box \alpha) = V \Leftrightarrow \forall w_i \in W. w R w_i \Rightarrow V(\alpha, w_i) = V$
\end{itemize}

Le prime tre regole sono pressoché identiche a quelle specificate per la semantica di $CP$,
tranne per il fatto che nella prima regola usiamo la funzione $I$ valutandola per il mondo w
in cui ci troviamo.

La novità è la quarta regola, quella riguardante la semantica di $\Box$.
Questa regola afferma che nel mondo $w$, $\alpha$ è necessariamente vera se e solo se
è vera in tutti i mondi accessibili da $w$.

Diciamo che una formula $\alpha$ è T-valida se $V(\alpha) = T$ per ogni T-modello $(w, W, R, I)$.

\section{Mostrare teorema di adeguatezza}


\section{Un sistema interessante(DA AGGIUSTARE)}
Dedichiamo questo paragrafo un sistema di logica modale particolare, chiamato Sistema $B$.
Esso è intimamente collegato alla logica intuizionista, come si evince dal nome,
infatti la B è la iniziale di Brouwer, logico padre dell'intuizionismo.

\end{document}
