\documentclass[a4paper, 12pt]{article}
\usepackage{latexsym}
\begin{document}
La logica modale è un'estensione della logica classica.
Studieremo la validità di argomenti che comprendono le modalità.
Le modalità sono modi in cui si può affermare che una proposizione è vera in certe circostanze.
Ad esempio, ``è possibile che $\alpha$'' significa che $\alpha$ non è detto sia vera, ma nulla vieta che non lo sia.
Quindi estendiamo il numero di proposizioni su cui possiamo dare un giudizio circa la loro validità.
Di conseguenza partiremo da un sistema formale per il calcolo proposizionale, che chiameremo $CP$
e definiamo una sua estensione\footnote{Definire cos'è un'estensione, à la Tortora} $CM$ \footnote{Calcolo Modale? Un po' bruttino}

\section{Definizione del linguaggio di $CM$}
Usiamo come base quella di $CP$. Una sua definizione la si può trovare in [Tortora].
Aggiungiamo un nuovo operatore monadico\footnote{decidere se è più bello monadico o unario}
che indicheremo con $\Box$.

\section{Definizione dell'apparato deduttivo per $CM$}
Aggiungiamo anche una nuova regola di deduzione, chiamata necessitazione:

$\alpha \vdash \Box \alpha$

L'operatore che vogliamo aggiungere, ha una natura non verofunzionale. Vale a dirsi,
che la verità o falsità della proposizione $\Box \alpha$ non dipende solo
dal valore di verità di $\alpha$.

Per quanto riguarda gli assiomi, la situazione si complica. Infatti 

\end{document}
