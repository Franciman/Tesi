\documentclass[a4paper, 12pt]{article}
\usepackage[utf8]{inputenc}
\usepackage{latexsym}
\newtheorem{axiom}{Assioma}
\newtheorem{theorem}{Teorema}
\newtheorem{lemma}{Lemma}
\newtheorem{definition}{Definizione}

% Ambiente per le dimostrazioni
\newenvironment{proof}
    {\textit{Dim.}
    }
    {\begin{flushright}$\bullet$\end{flushright}
    }

% Ambiente per le dimostrazioni formali formali
\newenvironment{formal_proof}
    {\begin{proof}

    % \begin{tabular} {c c|c}
    }
    {%\end{tabular}

    \end{proof}
    }

\begin{document}

\section{Logica proposizionale}
In questo paragrafo introdurremo brevemente un sistema formale per la logica proposizionale classica.
Il sistema che introduciamo sarà usato come base per la costruzione di ulteriori sistemi formali.
Esso sarà molto minimale, infatti conterrà pochi connettivi di base
e tutti gli altri saranno definiti successivamente a partire da questi.

Chiameremo il sistema che stiamo definendo CP (calcolo proposizionale).

Il primo passaggio è quello di definire un alfabeto. Usiamo una piccola scorciatoia,
infatti supponiamo che ci sia dato a priori un insieme numerabile A i cui elementi saranno detti atomi.
Nella pratica questo insieme conterrà i simboli da usare per le variabili che appaiono nelle formule.
In virtù di questa interpretazione, chiameremo gli elementi di A \textit{variabili proposizionali}.

\begin{definition}
L'alfabeto $\Sigma$ è dato dall'insieme $A \cup \{\neg, \vee, (, ) \}$
\end{definition}

Passiamo ora alla definizione dell'insieme delle formule ben formate.

\begin{definition}
L'insieme delle formule ben formate (fbf) $L$ è il più piccolo sottoinsieme dell'insieme tutte le successioni
di elementi di $\Sigma$ determinato dalla seguente definizione induttiva:
\begin{itemize}
\item Se $p \in A$, allora $p \in L$;
\item Se $\alpha \in L$, allora $\neg (\alpha) \in L$;
\item Se $\alpha \in L$ e $\beta \in L$, allora $(\alpha) \vee (\beta) \in L$;
\item $L$ non contiene altri elementi.
\end{itemize}
\end{definition}

Possiamo ricavare, ora, i connettivi che mancano come segue:
\begin{definition}
Se $\alpha, \beta \in L$,
$$(\alpha) \rightarrow (\beta) := (\neg (\alpha)) \vee (\beta)$$
$$(\alpha) \wedge (\beta) := \neg ((\neg (\alpha)) \vee (\neg (\beta)))$$
\end{definition}

Per semplificare la scrittura e rendere più comprensibili le formule che scriveremo
nel seguito, spesso ometteremo le parentesi e per interpretare correttamente
le formule scritte si useranno le seguenti regole di precedenza tra operatori
e le regole di associatività.

\begin{definition}
Gli operatori binari $\wedge,\vee$ sono associativi a sinistra,
mentre $\rightarrow$ è associativo a destra.
Per determinare l'ordine di precedenza tra due operatori si fa uso della
seguente tabella, usando il seguente criterio:
``Gli operatori che si trovano più in alto hanno precedenza maggiore rispetto
agli operatori che si trovano più in basso''
\begin{center}
\begin{tabular} {c c}
    $\neg$ \\
    $\wedge$ \\
    $\vee$ \\
    $\rightarrow$ & $\leftrightarrow$
\end{tabular}
\end{center}
\end{definition}

Di conseguenza la formula $\alpha \rightarrow \beta \rightarrow \neg \gamma$
dovrà essere interpretata come:
$(\alpha) \rightarrow ((\beta) \rightarrow (\neg (\gamma)))$.

Passiamo infine alla determinazione di un sottoinsieme di formule ben formate
che fungeranno da assiomi e delle regole di deduzione
per poter così definire l'apparato deduttivo del calcolo proposizionale.

\begin{definition}
Gli assiomi del calcolo proposizionale sono ottenuti dai seguenti schemi
sostituendo in modo uniforme formule ben formate al posto di $\alpha, \beta, \gamma$:
\begin{itemize}
\item (HPD) $\alpha \rightarrow (\beta \rightarrow \alpha)$
\item (HPMP) $(\gamma \rightarrow (\alpha \rightarrow \beta)) \rightarrow (\gamma \rightarrow \alpha) \rightarrow (\gamma \rightarrow \beta)$
\item ($\vee$-I1) $\alpha \rightarrow (\alpha \vee \beta)$
\item ($\vee$-I2) $\beta \rightarrow (\alpha \vee \beta)$
\item ($\vee$-E) $(\alpha \rightarrow \gamma) \rightarrow (\beta \rightarrow \gamma) \rightarrow (\alpha \vee \beta \rightarrow \gamma)$
\item ($\wedge$-I) $\alpha \rightarrow \beta \rightarrow \alpha \wedge \beta$
\item ($\wedge$-E1) $\alpha \wedge \beta \rightarrow \alpha$
\item ($\wedge$-E2) $\alpha \wedge \beta \rightarrow \beta$
\item ($\neg$-I) $(\alpha \rightarrow \beta) \wedge (\alpha \rightarrow \neg \beta) \rightarrow \neg \alpha$
\item (TER) $\alpha \vee \neg \alpha$
\end{itemize}
\end{definition}

Ricordiamo le due seguenti definizioni:

Ricordiamo che se esiste una dimostrazione della formula $\alpha \in L$,
ovvero, se $\alpha$ è un teorema, scriviamo $\vdash \alpha$.
Descriviamo ora le due regole di deduzione del nostro sistema.

La prima regola è il classico \textbf{Modus Ponens}:
$$\vdash \alpha \rightarrow \beta, \vdash \alpha \Rightarrow \vdash \beta$$

La seconda regola non è strettamente necessaria, ma sarà particolarmente utile per riutilizzare
teoremi già dimostrati, essa è detta \textbf{Regola di sostituzione}:
Se $\vdash \alpha$, allora, sostituendo uniformenente (ovvero sempre allo stesso modo)
ogni occorrenza di una variabile proposizionale all'interno di $\alpha$
con una formula ben formata, si ottiene ancora un teorema.

Come ultima cosa dimostriamo il seguente teorema che riutilizzeremo spesso nel seguito

\begin{flushleft}
\textbf{Teorema(Sillogismo)}
$\vdash (\alpha \rightarrow \beta) \rightarrow
        (\beta \rightarrow \gamma) \rightarrow (\alpha \rightarrow \gamma)$ in $CP$

\textit{Dim.}
Ricordiamo che per $CP$ vale il teorema di deduzione, ovvero:
$$\Gamma,\alpha \vdash \beta \Leftrightarrow \Gamma \vdash \alpha \rightarrow \beta$$
Quindi semplifichiamo il lavoro dimostrando:
$$(\alpha \rightarrow \beta),(\beta \rightarrow \gamma),\alpha \vdash \gamma$$

(1) $\alpha \rightarrow \beta$ ; ipotesi \\
(2) $\alpha$ ; ipotesi \\
(3) $\beta$ ; Modus Ponens di (1) e (2) \\
(4) $\beta \rightarrow \gamma$ ; ipotesi \\
(5) $\gamma$ ; Modus Ponens di (3) e (4)

Ora applicando tre volte il teorema di deduzione otteniamo:
$$\vdash (\alpha \rightarrow \beta) \rightarrow (\beta \rightarrow \gamma) \rightarrow (\alpha \rightarrow \gamma)$$

\begin{flushright}
$\bullet$
\end{flushright}
\end{flushleft}


\section{Logica modale}
Ora che abbiamo definito un sistema formale per la logica proposizionale,
lo estenderemo in vari modi.

Il nostro obiettivo è avere un sistema formale nel quale possiamo definire e studiare
le proprietà di due nuovi operatori unari che vorremmo avessero un comportamento che rispecchi
i concetti di ``necessità'' e ``possibilità''.

Prima di cominciare facciamo delle osservazioni importanti.
Ricordiamo che un operatore verofunzionale è una funzione il cui risultato
è univocamente determinato dal valore di verità dei parametri da cui dipende.
Inoltre, si può dimostrare che nel calcolo proposizionale si possono esprimere
tutti gli operatori verofunzionali possibili in termini di $\neg e \vee$.
Notiamo, ora, che gli operatori che vogliamo aggiungere non possono essere verofunzionali,
infatti il valore di verità della frase ``è necessario che p'' non dipende \textit{solo}
dal valore di verità che si associ, infatti, il fatto che p sia vera, non ci permette
di concludere che p sia necessariamente vera. Quindi per avere una speranza di formalizzare
i concetti di necessità e possibilità siamo costretti ad estendere il sistema $CP$
in maniera sostanziale.

A tal proposito, non è esattamente chiaro come caratterizzare formalmente
le nozioni di ``necessità'' e ``possibilità'', e anzi vedremo che queste potranno essere
intese in modi diversi.
Questa ricchezza di significati si rifletterà in una varietà di sistemi formali più o meno potenti.
Noi ci focalizzeremo in particolare quattro sistemi formali:
\begin{itemize}
\item Sistema T
\item Sistema S4
\item Sistema S5
\end{itemize}

Questi sistemi differiscono tra di loro soltanto per gli assiomi scelti in ognuno di essi.
Quindi iniziamo definendo la parte comune di tutti i sistemi formali, ovvero il linguaggio
e le regole di deduzione.

L'alfabeto dei sistemi è ottenuto dall'alfabeto del sistema $CP$ aggiungendo il simbolo $\Box$,
che vorremo far corrispondere alla nozione di ``necessità''.

Per quanto riguarda le formule ben formate, la definizione induttiva che genera l'insieme $L$
delle formule ben formate è ottenuta dalle regole usate per definire $L$ in $CP$ più
la seguente regola aggiuntiva:
\begin{itemize}
\item Se $\alpha \in L$, allora $\Box (\alpha) \in L$
\end{itemize}

Aggiungiamo il seguente ulteriore operatore, che corrisponderà alla nozione di ``possibilità''.

\begin{definition}
Se $\alpha \in L$:
$$\Diamond (\alpha) := \neg \Box \neg (\alpha)$$
\end{definition}

Infine, per quanto riguarda le regole di deduzione, alle due di $CP$ aggiungiamo
una nuova regola, detta \textbf{Regola di necessitazione}:
$$\vdash \alpha \Rightarrow \vdash \Box \alpha$$

Una volta definita la parte comune a tutti i sistemi a cui siamo interessati,
passiamo ora ad esaminare per ogni sistema gli assiomi che lo caratterizzano.

\section{Sistema T}
Il sistema T ha come assiomi tutti gli assiomi di $CP$ più:
\begin{itemize}
\item (K) $\Box (\alpha \rightarrow \beta) \rightarrow (\Box \alpha \rightarrow \Box \beta)$
\item (T) $\Box \alpha \rightarrow \alpha$
\end{itemize}

\section{Sistema S4}
Il sistema S4 ha come assiomi tutti gli assiomi del Sistema T più il seguente assioma:
$$(S4) \Box \alpha \rightarrow \Box \Box \alpha$$

\section{Sistema S5}
Il sistema S5 ha come assiomi tutti gli assiomi del Sistema T più:
$$(S5) \Diamond \alpha \rightarrow \Box \Diamond \alpha$$


Ora che abbiamo definito il linguaggio e l'apparato deduttivo di tutti i sistemi
che ci interessano, facciamo delle osservazioni su di essi.
Per prima cosa, discende ovviamente dalla definizione che il Sistema T è contenuto
sia in S4 che in S5.
Ora, però mostreremo che anche S4 è contenuto in S5. A tal fine premettiamo alcuni risultati
preliminari.

Siccome il Sistema T è contenuto in ciascuno dei sistemi più potenti che abbiamo costruito,
se dimostriamo un teorema usando solo gli assiomi del Sistema T, questo teorema sarà ovviamente valido
anche nei sistemi più potenti.

\begin{flushleft}
\textbf{Teorema}
$\vdash \alpha \rightarrow \Diamond \alpha$ in $T$

\textit{Dim.}

(1) $\Box \neg \alpha \rightarrow \neg \alpha$ ; $T[\neg \alpha/\alpha]$ \\
(2) $\alpha \rightarrow \neg \Box \neg \alpha$ ; Contrapposta di (1) \\
(3) $\alpha \rightarrow \Diamond \alpha$ ; Definizione di $\Diamond$

\begin{flushright}
$\bullet$
\end{flushright}
\end{flushleft}

\begin{flushleft}
\textbf{Lemma 1}
$\vdash \diamond \Box x \rightarrow \Box x$ in $S5$

\textit{Dim.}

(1) $\Diamond \neg x \rightarrow \Box \Diamond \neg x$ ; $S5[\neg x/\alpha]$ \\
(2) $\neg \Box \diamond \neg x \rightarrow \neg \diamond \neg x$ ; contrapposta di (1) \\
(3) $\diamond \neg \diamond \neg x \rightarrow \neg \diamond \neg x$ ; definizione di $\diamond$ \\
(4) $\diamond \Box x \rightarrow \Box x$ ; definizione di $\diamond$ e $\Box$

\begin{flushright}
$\bullet$
\end{flushright}
\end{flushleft}

\begin{flushleft}
\textbf{Teorema 1}
$\vdash \Box x \rightarrow \Box \Box x$ in $S5$, ovvero il Sistema S5 contiene il Sistema S4.

\textit{Dim.}

(1) $\Box x \rightarrow \Diamond \Box x$ ; Teorema \\
(2) $(\Box x \rightarrow \Diamond \Box x) \rightarrow (\Diamond \Box x \rightarrow \Box\Diamond\Box x) \rightarrow (\Box x \rightarrow \Box \Diamond \Box x)$ ; Teorema \\
(3) $\Diamond x \rightarrow \Box \Diamond x$ ; Assioma S5 \\
(4) $\Box x \rightarrow \Box \Diamond \Box x$ ; Due volte Modus Ponens usando (1) (2) e (3) \\
(5) $\Diamond \Box x \rightarrow \Box x$ ; Lemma 1 \\
(4) $\Box (\Diamond \Box x \rightarrow \Box x)$ ; Necessitazione (5) \\
(5) $\Box(\Diamond\Box x \rightarrow \Box x) \rightarrow \Box \Diamond \Box x \rightarrow \Box \Box x$ ; Assioma K \\
(6) $\Box\Diamond\Box x \rightarrow \Box\Box x$ ; Modus Ponens (4) e (5) \\
(7) $(\Box x \rightarrow \Diamond \Box x) \rightarrow (\Box\Diamond\Box x \rightarrow \Box\Box x) \rightarrow (\Box x \rightarrow \Box\Box x)$ ; Teorema \\
(8) $\Box x \rightarrow \Box \Box x$ ; Due volte Modus Ponens usando (1) (5) e (6)

\begin{flushright}$\bullet$\end{flushright}
\end{flushleft}

\section{Un teorema di deduzione}
In questo paragrafo proviamo a derivare un teorema di deduzione per il Sistema $S4$.
Non possiamo aspettarci che il teorema di deduzione per il calcolo proposizionale $CP$ valga,
infatti la regola di deduzione di necessitazione ci porterebbe ad asserire che:
$$ x \vdash \Box x \Rightarrow \vdash x \rightarrow \Box x $$
E ciò vorrebbe dire che le modalità non aggiungono nulla al sistema del calcolo proposizionale.

Se guardiamo attentamente la regola di necessitazione e la scriviamo in questo modo:
$$x \vdash \Box x$$
possiamo notare che l'ipotesi $x$ è sufficiente per dimostrare $\Box x$ che afferma molto più
che semplicemente $x$.
Notiamo anche che la regola di necessitazione ci permette di provare
$$x \vdash \Box \Box x$$

Quindi, far semplicemente passare la x da sinistra a destra, significherebbe affermare che
da ipotesi più deboli si possono comunque dedurre le stesse conclusioni. Allora possiamo provare,
nello spostare la x da sinistra a destra di $\vdash$, ad aggiungere un $\Box$.
Questo può bastare in sistemi come $S4$ o più potenti, perché da $\Box x$ possiamo dedurre
$\Box \Box x$, $\Box \Box \Box x$, ecc...
Mentre nel sistema T questo non è vero e quindi dovremo risolvere diversamente.

Recuperiamo quindi un teorema di deduzione, facendo la seguente modifica:
$$\alpha \vdash \beta \Rightarrow \vdash \Box \alpha \rightarrow \beta$$
Ora il precedente problema è facilmente risolvibile, infatti, la deduzione $x \vdash \Box x$
diventa il teorema $\vdash \Box x \rightarrow \Box x$ banalmente valido.
Otteniamo però anche il seguente teorema, $\vdash \Box x \rightarrow \Box \Box x$ che è l'assioma $S4$
Ecco perché abbiamo supposto di trovarci nel sistema $S4$.

Diamo una dimostrazione di

\begin{flushleft}
\textbf{Teorema}
$\alpha \vdash \beta \Leftrightarrow \vdash \Box \alpha \rightarrow \beta$ in $S4$

\textit{Dim.}
Lasciata al lettore, è una banale modifica della dimostrazione nel caso del calcolo proposizionale.
L'unica parte interessante è quella riguardante i teoremi ottenuti con la regola di necessitazione.
In quel caso dobbiamo far uso dell'assioma $S4$ ed è tutto banale.
\end{flushleft}



\section{Modalità e funzioni modali}
Rendiamo più formale il concetto di modalità.
\textbf{Definizione}
Una modalità è una successione di 0 o più di uno dei seguenti simboli: $\neg,\diamond,\Box$.

Viene da domandarci, ora, quante modalità ci sono in ognuno dei sistemi che abbiamo considerato.
Si può dimostrare che nel Sistema $T$ ce ne sono infinite,
mentre in $S4$ e $S5$ ce ne sono solo un numero finito.

Parleremo più approfonditamente di tali questioni solo dopo aver introdotto una semantica
per il nostro linguaggio.

\begin{flushleft}
\textbf{Definizione 1}

Una funzione modale è una formula contenente almeno una funzione modale
\end{flushleft}

In $T$ ovviamente c'è un numero infinito di funzioni modali esprimibili.
In $S5$ ce n'è un numero finito, mentre sorprendentemente,
in $S4$ ce ne sono infinite, nonostante le modalità esprimibili siano in numero finito.

\section{Un sistema interessante}
Dedichiamo questo paragrafo un sistema di logica modale particolare, chiamato Sistema $B$.
Esso è intimamente collegato alla logica intuizionista, come si evince dal nome,
infatti la B è la iniziale di Brouwer, logico padre dell'intuizionismo.

\section{Una semantica per i sistemi modali}
Seguendo la tradizione, proviamo a dare una semantica per i sistemi formali sopra definiti.
La costruzione che presenteremo è dovuta a Kripke e Hintikka ed altri.

\end{document}
