\documentclass[a4paper, 12pt]{article}
\usepackage[utf8]{inputenc}
\usepackage{amssymb}
\usepackage{amsmath}
\newtheorem{axiom}{Assioma}
\newtheorem{theorem}{Teorema}
\newtheorem{lemma}{Lemma}
\newtheorem{definition}{Definizione}

\title{I sistemi di logica modale T, S4, S5: Sintassi e Semantica}
\date{}

\linespread{1.5}

% Ambiente per le dimostrazioni
\newenvironment{proof}
    {\textit{Dim.}
    }
    {\begin{flushright}$\bullet$\end{flushright}
    }

% Ambiente per le dimostrazioni formali formali
\newenvironment{formal_proof}
    {\begin{proof}

    % \begin{tabular} {c c|c}
    }
    {%\end{tabular}

    \end{proof}
    }

\begin{document}
\maketitle
\section{Introduzione}
In questa tesi ci occupiamo di usare i metodi della logica matematica
per studiare e descrivere i ragionamenti che fanno uso delle nozioni
di ``necessarietà'' e ``possibilità''. Nello specifico andremo a studiare
e confrontare tre sistemi insieme alle loro semantiche di Kripke
che formalizzano tali nozioni: T, S4 e S5;

\section{Nozioni Preliminari}
Premettiamo allo studio di sistemi di logica modale la definizione
di sintassi e semantica di un sistema formale per il calcolo proposizionale,
che da ora in poi denoteremo con $CP$, da usare come base per la definizione
dei sistemi formali successivi.
È possibile presentare un tale sistema in vari modi, tutti pressoché equivalenti tra di loro.
Noi seguiremo seguire la strada di [Hughes-Cresswell], definendo un linguaggio con un numero ristretto
di connettivi primitivi e introducendo successivamente i connettivi mancanti definendoli a partire
dai connettivi primitivi.
Questa scelta ci permette di usare tutti i connettivi che siamo soliti adoperare,
rendendo quindi la scrittura delle formule più chiara, ma allo stesso tempo ci permette anche di
considerare solo i connettivi primitivi nello studio delle proprietà del sistema,
semplificandoci il lavoro.

\subsection{Il Linguaggio}
Per definire il linguaggio di $CP$ avremo bisogno di un insieme di oggetti di base,
chiamati variabili proposizionali. Senza soffermarci ulteriormente sulla natura di questi oggetti,
possiamo tranquillamente assumere che esista un insieme $A$ non vuoto numberabile
disgiunto da $\{\neg, \lor, (, )\}$.

\begin{definition}
Chiamiamo variabile proposizionale ogni elemento di $A$.
\end{definition}

Scegliamo come connettivi primitivi $\neg$ e $\lor$,
a partire da essi sarà possibile definire tutti gli altri.
L'alfabeto di $CP$ è dunque dato dall'insieme $\Sigma = A \cup \{\neg, \lor, (, )\}$.

Chiamiamo $\Sigma^{*} = \bigcup_{n \in A} \Sigma^n$ l'insieme delle espressioni
e definiamo l'insieme delle formule ben formate (fbf) $L_{CP} \subseteq \Sigma^{*}$ con la seguente definizione
per induzione:
\begin{itemize}
\item $x \in A \Rightarrow x \in L_{CP}$;
\item $\alpha \in L_{CP} \Rightarrow \neg (\alpha) \in L_{CP}$;
\item $\alpha \in L_{CP}$ e $\beta \in L_{CP} \Rightarrow (\alpha) \lor (\beta) \in L_{CP}$;
\item Non ci sono altri elementi in $L_{CP}$.
\end{itemize}

Una volta definito il linguaggio, definiamo i connettivi che mancano a partire da $\neg$ e $\lor$
come segue:
\begin{definition}
Se $\alpha \in L_{CP}$ e $\beta \in L_{CP}$, allora:
\begin{itemize}
\item $(\alpha) \land (\beta) := \neg(\neg(\alpha) \lor \neg(\beta))$
\item $(\alpha) \rightarrow (\beta) := (\neg(\alpha)) \lor (\beta)$
\item $(\alpha) \leftrightarrow (\beta) := ((\alpha) \rightarrow (\beta)) \land ((\beta) \rightarrow (\alpha))$
\end{itemize}
\end{definition}

Al fine di rendere le formule che scriveremo più comprensibili, seguendo la prassi,
ometteremo spesso le parentesi usando le convenzioni descritte in [Tortora; pag 46].

\subsection{L'apparato deduttivo}
Per la definizione dell'apparato deduttivo di $CP$ seguiamo quanto fatto in [Tortora].

Vale la pena far notare che questa non è l'unica possibilità, ma ci sono tante altre scelte sia
per quanto riguarda gli assiomi che le regole di deduzione, che sono altrettanto valide
e che portano allo stesso risultato. La motivazione dietro la nostra scelta
risiede semplicemente nel fatto che questo approccio è quello da noi ritenuto più familiare.

\begin{definition}
Gli assiomi di $CP$ sono ottenuti dai seguenti schemi
sostituendo in modo uniforme formule ben formate al posto di $\alpha, \beta, \gamma$:
\begin{itemize}
\item (HPD) $\alpha \rightarrow (\beta \rightarrow \alpha)$
\item (HPMP) $(\gamma \rightarrow (\alpha \rightarrow \beta)) \rightarrow (\gamma \rightarrow \alpha) \rightarrow (\gamma \rightarrow \beta)$
\item ($\lor$-I1) $\alpha \rightarrow (\alpha \lor \beta)$
\item ($\lor$-I2) $\beta \rightarrow (\alpha \lor \beta)$
\item ($\lor$-E) $(\alpha \rightarrow \gamma) \rightarrow (\beta \rightarrow \gamma) \rightarrow (\alpha \lor \beta \rightarrow \gamma)$
\item ($\land$-I) $\alpha \rightarrow \beta \rightarrow \alpha \land \beta$
\item ($\land$-E1) $\alpha \land \beta \rightarrow \alpha$
\item ($\land$-E2) $\alpha \land \beta \rightarrow \beta$
\item ($\neg$-I) $(\alpha \rightarrow \beta) \land (\alpha \rightarrow \neg \beta) \rightarrow \neg \alpha$
\item (TER) $\alpha \lor \neg \alpha$
\end{itemize}
\end{definition}

L'unica regola di deduzione di $CP$ è il \textit{Modus Ponens}:
$$\frac{\alpha \rightarrow \beta \ \ \alpha}{\beta}$$

Notiamo che gli assiomi che abbiamo scelto per $CP$ non sono in numero finito,
ma sono infiniti. L'uso di schemi di assiomi porta a notevoli semplificazioni
nella pratica; infatti ci permette di dimostrare in una sola volta famiglie di teoremi,
senza dover ripetere volta per volta un procedimento che sarebbe essenzialmente sempre uguale.

Diamo ora le seguenti definizioni che introducono i concetti di dimostrazione e di
deduzione, strettamente collegati tra loro.

\begin{definition}
Una dimostrazione è una sequenza finita $D$ di fbf tale che
ogni termine $D_i$ della sequenza soddisfa almeno una delle seguenti condizioni
\begin{itemize}
\item $D_i$ è un assioma;
\item esistono $D_h, D_k$, con $h < i, k < i$, tali che $D_i$ è derivato per \textit{Modus Ponens}
da $D_h$ e $D_k$.
\end{itemize}
\end{definition}

Se $\alpha \in L_{CP}$ ed esiste una dimostrazione $D$ il cui ultimo termine è $\alpha$,
diciamo che esiste una dimostrazione di $\alpha$ e scriviamo $\vdash \alpha$.


\begin{definition}
Se $\Gamma \subseteq L_{CP}$, una deduzione a partire dalle ipotesi $\Gamma$ è una sequenza finita $D$
di fbf tale che ogni termine $D_i$ della sequenza soddisfa almeno una delle seguenti condizioni
\begin{itemize}
\item $D_i \in \Gamma$;
\item $D_i$ è un assioma;
\item esistono $D_h, D_k$, con $h < i, k < i$, tali che $D_i$ è derivato per \textit{Modus Ponens}
da $D_h$ e $D_k$.
\end{itemize}
\end{definition}

Se $\alpha \in L_{CP}$, $\Gamma \subseteq L_{CP}$ ed esiste una deduzione $D$ a partire dalle ipotesi $\Gamma$
il cui ultimo termine è $\alpha$, diciamo che esiste una deduzione di $\alpha$
a partire dalle ipotesi $\Gamma$ e scriviamo $\Gamma \vdash \alpha$.

Nel seguito, scriveremo $\Gamma, \alpha \vdash \beta$
per indicare $\Gamma \cup \{\alpha\} \vdash \beta$.

Vale la pena ricordare il seguente teorema:
\begin{flushleft}
\textbf{Teorema di deduzione}
$$\Gamma, \alpha \vdash \beta \Leftrightarrow \Gamma \vdash \alpha \rightarrow \beta$$
\textit{Dim.} [Tortora]
\begin{flushright}
$\bullet$
\end{flushright}
\end{flushleft}


Terminiamo questo paragrafo mostrando un esempio di dimostrazione di una famiglia
di teoremi, che ci tornerà utile anche nel seguito.

\begin{flushleft}
\textbf{Teorema(Sillogismo)}
Se $\alpha \in L_{CP}$ e $\beta \in L_{CP}$
$$\vdash (\alpha \rightarrow \beta) \rightarrow
        (\beta \rightarrow \gamma) \rightarrow (\alpha \rightarrow \gamma)$$

\textit{Dim.}
Semplifichiamo il lavoro dimostrando
$$(\alpha \rightarrow \beta),(\beta \rightarrow \gamma),\alpha \vdash \gamma$$
in luogo di
$$\vdash (\alpha \rightarrow \beta) \rightarrow
        (\beta \rightarrow \gamma) \rightarrow (\alpha \rightarrow \gamma)$$
e sfruttiamo poi il teorema di deduzione sopra menzionato.

Una possibile deduzione per $\gamma$ a partire dalle ipotesi
$\{(\alpha \rightarrow \beta),(\beta \rightarrow \gamma),\alpha\}$ è la seguente.

(1) $\alpha \rightarrow \beta$ ; ipotesi \\
(2) $\alpha$ ; ipotesi \\
(3) $\beta$ ; Modus Ponens di (1) e (2) \\
(4) $\beta \rightarrow \gamma$ ; ipotesi \\
(5) $\gamma$ ; Modus Ponens di (3) e (4)

Possiamo dunque affermare:
$$(\alpha \rightarrow \beta),(\beta \rightarrow \gamma),\alpha \vdash \gamma$$

Infine, applicando tre volte il teorema di deduzione otteniamo:
$$\vdash (\alpha \rightarrow \beta) \rightarrow (\beta \rightarrow \gamma) \rightarrow (\alpha \rightarrow \gamma)$$

\begin{flushright}
$\bullet$
\end{flushright}
\end{flushleft}

\subsection{Semantica di $CP$}
Ci proponiamo ora di fornire una semantica per $CP$.
La semantica che intendiamo fornire per $CP$ è quella standard, in particolare
seguiamo l'esposizione di [Tortora] per presentarla.
L'idea è quella di associare ad ogni formula
un valore di verità, interpretando le variabili proposizionali come proposizioni
che descrivono fatti della \textit{realtà}, ad esempio ``Michele è altisonante'',
il simbolo $\neg$ come il connettivo logico ``non'' dell'italiano
e il simbolo $\lor$ come disgiunzione inclusiva, ossia quel connettivo
logico che in italiano si rappresenta spesso come ``o..., o..., oppure entrambi''.
Per fare ciò dovremmo definire cosa si intende per proposizione,
ma possiamo omettere questo passaggio, a dire il vero un po' problematico.
Infatti, osserviamo che, associando ad ogni variabile proposizionale un valore di verità,
siamo in grado di associare un valore di verità ad ogni formula ben formata.
Infatti, il valore di verità di ``non $\alpha$'' dipende solo dal valore di verità
di $\alpha$ e allo stesso modo il valore di verità di
``o $\alpha$, o $\beta$, oppure entrambi``
dipende solo dal valore di verità di $\alpha$ e $\beta$.

Connettivi che hanno questa peculiarità vengono detti operatori vero-funzionali.

Passiamo ora alla formalizzazione di queste idee e diamo le seguenti definizioni:

\begin{definition}
Chiamiamo interpretazione una funzione $I: A \to \{0, 1\}$
\end{definition}

Data un'interpretazione $I$ definiamo per ricorsione la funzione di valutazione $V_I$ che associa ad ogni formula
del linguaggio di $CP$ un valore dell'insieme $\{0, 1\}$ come segue:

Se $\alpha, \beta \in L_{CP}$:
\begin{itemize}
\item $V_I(p) = I(p)$, $\forall p \in A$
\item $V_I(\alpha \lor \beta) = max\{V_I(\alpha), V_I(\beta)\}$
\item $V_I(\neg \alpha) = 1 - V_I(\alpha)$
\end{itemize}


Quindi un'interpretazione ci dà in un certo senso una lista di tutte le proposizioni
vere e di tutte le proposizioni false.
Usiamo poi i valori assegnati dall'interpretazione per determinare il valore di verità delle formule
in base alla semantica che abbiamo assegnato a ciascun connettivo (che sono codificate nelle regole
che abbiamo usato per definire la funzione $V_I$).

\begin{definition}
Sia $\alpha \in L_{CP}$, diremo che $\alpha$ è valida e scriveremo $\vDash \alpha$ se
$V_I(\alpha) = 1$, per ogni interpretazione $I$.
\end{definition}

È noto, e una discussione approfondita di questo argomento si può trovare in [Mendelson],
che è possibile definire qualsiasi tipo di operatore vero-funzionale usando
solo gli operatori di negazione e disgiunzione inclusiva.
Quindi, analogamente a quanto fatto per i connettivi $\land, \rightarrow, \leftrightarrow$,
possiamo pensare di aggiungere qualsiasi operatore vero-funzionale ci venga in mente
a $CP$ definendolo in termini di $\neg$ e $\lor$.
Quindi anche se dalla presentazione non sembrerebbe, in realtà in $CP$ sono presenti
tutti gli operatori vero-funzionali possibili.

\subsection{Alcune proprietà importanti di $CP$}

Terminiamo questo capitolo richiamando alcune proprietà di $CP$ fondamentali, omettendone
le dimostrazioni. Un trattamento più approfondito di questi argomenti si può trovare in
[Tortora] o [Mendelson].

Esiste una procedura di decisione che data una qualsiasi formula ben formata di $CP$
ci permette sempre di determinare in un numero finito di passi se questa è valida o no.

Inoltre vale il seguente teorema che collega l'aspetto sintattico all'aspetto semantico
di $CP$:

\begin{flushleft}
\textbf{Teorema di completezza}
$$\vdash \alpha \Leftrightarrow \vDash \alpha$$
\end{flushleft}

Un corollario immediato del teorema di completezza è che anche l'insieme dei teoremi
di $CP$ è un insieme decidibile, infatti secondo il teorema di completezza
l'insieme dei teoremi di $CP$ e l'insieme delle formule ben formate di $CP$ valide
coincidono, quindi usando la procedura di decisione per le formule valide
possiamo anche determinare le formule che sono teoremi.

\section{Logica modale}
Ora che abbiamo definito un sistema formale per la logica proposizionale,
lo estenderemo in vari modi.

Il nostro obiettivo è avere un sistema formale nel quale possiamo definire e studiare
le proprietà di due nuovi operatori unari che vorremmo avessero un comportamento che rispecchi
i concetti di ``necessità'' e ``possibilità''.

Prima di cominciare facciamo delle osservazioni importanti.
Ricordiamo che un operatore verofunzionale è una funzione il cui risultato
è univocamente determinato dal valore di verità dei parametri da cui dipende.
Inoltre, si può dimostrare che nel calcolo proposizionale si possono esprimere
tutti gli operatori verofunzionali possibili in termini di $\neg e \vee$.
Quindi, se gli operatori che vogliamo aggiungere fossero verofunzionali, potremmo
continuare il nostro studio usando $CP$ e non servirebbe altro.
Notiamo, però, che gli operatori che vogliamo aggiungere non possono essere verofunzionali,
infatti il valore di verità della frase ``è necessario che p'' non dipende \textit{solo}
dal valore di verità che si associ, infatti, il fatto che p sia vera, non ci permette
di concludere che p sia necessariamente vera. Quindi per avere una speranza di formalizzare
i concetti di necessità e possibilità siamo costretti ad estendere il sistema $CP$
in maniera sostanziale.

A tal proposito, non è esattamente chiaro come caratterizzare formalmente
le nozioni di ``necessità'' e ``possibilità'', e anzi vedremo che queste potranno essere
intese in modi diversi.
Questa ricchezza di significati si rifletterà in una varietà di sistemi formali più o meno potenti.
Noi ci focalizzeremo in particolare su quattro sistemi formali:
\begin{itemize}
\item Sistema T
\item Sistema S4
\item Sistema S5
\end{itemize}

Questi sistemi differiscono tra di loro soltanto per gli assiomi scelti in ognuno di essi.
Quindi iniziamo definendo la parte comune di tutti i sistemi formali, ovvero il linguaggio
e le regole di deduzione.

L'alfabeto dei sistemi è ottenuto dall'alfabeto del sistema $CP$ aggiungendo il simbolo $\Box$,
che vorremo far corrispondere alla nozione di ``necessità''.

Per quanto riguarda le formule ben formate, la definizione induttiva che genera l'insieme $L$
delle formule ben formate è ottenuta dalle regole usate per definire $L$ in $CP$ più
la seguente regola aggiuntiva:
\begin{itemize}
\item Se $\alpha \in L$, allora $\Box (\alpha) \in L$
\end{itemize}

Diamo la seguente definizione, il cui intento è esprimere l'altro operatore
che ci interessa, quello corrispondente alla nozione di ``possibilità'', in termini di $\Box$.

\begin{definition}
Se $\alpha \in L$:
$$\diamond (\alpha) := \neg \Box \neg (\alpha)$$
\end{definition}

Infine, per quanto riguarda le regole di deduzione, alle due di $CP$ aggiungiamo
una nuova regola, detta \textbf{Regola di necessitazione}:
$$\vdash \alpha \Rightarrow \vdash \Box \alpha$$

Una volta definita la parte comune a tutti i sistemi a cui siamo interessati,
passiamo ora ad esaminare per ogni sistema gli assiomi che lo caratterizzano.

\subsection{Sistema T}
Il sistema T ha come assiomi tutti gli assiomi di $CP$ più:
\begin{itemize}
\item (K) $\Box (\alpha \rightarrow \beta) \rightarrow (\Box \alpha \rightarrow \Box \beta)$
\item (T) $\Box \alpha \rightarrow \alpha$
\end{itemize}

\subsection{Sistema S4}
Il sistema S4 ha come assiomi tutti gli assiomi del Sistema T più il seguente assioma:
$$(S4) \Box \alpha \rightarrow \Box \Box \alpha$$

\subsection{Sistema S5}
Il sistema S5 ha come assiomi tutti gli assiomi del Sistema T più:
$$(S5) \diamond \alpha \rightarrow \Box \diamond \alpha$$


Ora che abbiamo definito il linguaggio e l'apparato deduttivo di tutti i sistemi
che ci interessano, facciamo delle osservazioni su di essi.
\begin{definition}
Diremo che un sistema formale è meno forte di un altro se tutte i teoremi del primo
sono teoremi anche del secondo.
\end{definition}

Ovviamente, per come abbiamo costruito i nostri sistemi formali, il Sistema T è meno forte
sia di S4 che di S5.
Ora, però mostreremo che anche S4 è meno forte di S5. A tal fine premettiamo alcuni risultati
preliminari.

\begin{flushleft}
\textbf{Teorema}
$\vdash \alpha \rightarrow \diamond \alpha$ in $T$

\textit{Dim.}

(1) $\Box \neg \alpha \rightarrow \neg \alpha$ ; $T[\neg \alpha/\alpha]$ \\
(2) $\alpha \rightarrow \neg \Box \neg \alpha$ ; Contrapposta di (1) \\
(3) $\alpha \rightarrow \diamond \alpha$ ; Definizione di $\diamond$

\begin{flushright}
$\bullet$
\end{flushright}
\end{flushleft}

\begin{flushleft}
\textbf{Lemma 1}
$\vdash \diamond \Box x \rightarrow \Box x$ in $S5$

\textit{Dim.}

(1) $\diamond \neg x \rightarrow \Box \diamond \neg x$ ; $S5[\neg x/\alpha]$ \\
(2) $\neg \Box \diamond \neg x \rightarrow \neg \diamond \neg x$ ; contrapposta di (1) \\
(3) $\diamond \neg \diamond \neg x \rightarrow \neg \diamond \neg x$ ; definizione di $\diamond$ \\
(4) $\diamond \Box x \rightarrow \Box x$ ; definizione di $\diamond$ e $\Box$

\begin{flushright}
$\bullet$
\end{flushright}
\end{flushleft}

\begin{flushleft}
\textbf{Teorema 1}
$\vdash \Box x \rightarrow \Box \Box x$ in $S5$, ovvero il Sistema S5 è più forte del Sistema S4.

\textit{Dim.}

(1) $\Box x \rightarrow \diamond \Box x$ ; Teorema \\
(2) $(\Box x \rightarrow \diamond \Box x) \rightarrow (\diamond \Box x \rightarrow \Box\diamond\Box x) \rightarrow (\Box x \rightarrow \Box \diamond \Box x)$ ; Teorema \\
(3) $\diamond x \rightarrow \Box \diamond x$ ; Assioma S5 \\
(4) $\Box x \rightarrow \Box \diamond \Box x$ ; Due volte Modus Ponens usando (1) (2) e (3) \\
(5) $\diamond \Box x \rightarrow \Box x$ ; Lemma 1 \\
(4) $\Box (\diamond \Box x \rightarrow \Box x)$ ; Necessitazione (5) \\
(5) $\Box(\diamond\Box x \rightarrow \Box x) \rightarrow \Box \diamond \Box x \rightarrow \Box \Box x$ ; Assioma K \\
(6) $\Box\diamond\Box x \rightarrow \Box\Box x$ ; Modus Ponens (4) e (5) \\
(7) $(\Box x \rightarrow \diamond \Box x) \rightarrow (\Box\diamond\Box x \rightarrow \Box\Box x) \rightarrow (\Box x \rightarrow \Box\Box x)$ ; Teorema \\
(8) $\Box x \rightarrow \Box \Box x$ ; Due volte Modus Ponens usando (1) (5) e (6)

\begin{flushright}$\bullet$\end{flushright}
\end{flushleft}

%\subsection{Un teorema di deduzione (DA AGGIUSTARE)}
%In questo paragrafo proviamo a derivare un teorema di deduzione per il Sistema $S4$.
%Non possiamo aspettarci che il teorema di deduzione per il calcolo proposizionale $CP$ valga,
%infatti la regola di deduzione di necessitazione ci porterebbe ad asserire che:
%$$ x \vdash \Box x \Rightarrow \vdash x \rightarrow \Box x $$
%E ciò vorrebbe dire che le modalità non aggiungono nulla al sistema del calcolo proposizionale.
%
%Se guardiamo attentamente la regola di necessitazione e la scriviamo in questo modo:
%$$x \vdash \Box x$$
%possiamo notare che l'ipotesi $x$ è sufficiente per dimostrare $\Box x$ che afferma molto più
%che semplicemente $x$.
%Notiamo anche che la regola di necessitazione ci permette di provare
%$$x \vdash \Box \Box x$$
%
%Quindi, far semplicemente passare la x da sinistra a destra, significherebbe affermare che
%da ipotesi più deboli si possono comunque dedurre le stesse conclusioni. Allora possiamo provare,
%nello spostare la x da sinistra a destra di $\vdash$, ad aggiungere un $\Box$.
%Questo può bastare in sistemi come $S4$ o più potenti, perché da $\Box x$ possiamo dedurre
%$\Box \Box x$, $\Box \Box \Box x$, ecc...
%Mentre nel sistema T questo non è vero e quindi dovremo risolvere diversamente.
%
%Recuperiamo quindi un teorema di deduzione, facendo la seguente modifica:
%$$\alpha \vdash \beta \Rightarrow \vdash \Box \alpha \rightarrow \beta$$
%Ora il precedente problema è facilmente risolvibile, infatti, la deduzione $x \vdash \Box x$
%diventa il teorema $\vdash \Box x \rightarrow \Box x$ banalmente valido.
%Otteniamo però anche il seguente teorema, $\vdash \Box x \rightarrow \Box \Box x$ che è l'assioma $S4$
%Ecco perché abbiamo supposto di trovarci nel sistema $S4$.
%
%Diamo una dimostrazione di
%
%\begin{flushleft}
%\textbf{Teorema}
%$\alpha \vdash \beta \Leftrightarrow \vdash \Box \alpha \rightarrow \beta$ in $S4$
%
%\textit{Dim.}
%Lasciata al lettore, è una banale modifica della dimostrazione nel caso del calcolo proposizionale.
%L'unica parte interessante è quella riguardante i teoremi ottenuti con la regola di necessitazione.
%In quel caso dobbiamo far uso dell'assioma $S4$ ed è tutto banale.
%\end{flushleft}
%
%
%
%\subsection{Modalità e funzioni modali}
%Rendiamo più formale il concetto di modalità.
%\textbf{Definizione}
%Una modalità è una successione di 0 o più di uno dei seguenti simboli: $\neg,\diamond,\Box$.
%
%Viene da domandarci, ora, quante sono le modalità distinte che ci sono in ognuno dei sistemi che abbiamo considerato.
%Si può dimostrare che nel Sistema $T$ ce ne sono infinite,
%mentre in $S4$ e $S5$ ce ne sono solo un numero finito.
%
%Parleremo più approfonditamente di tali questioni solo dopo aver introdotto una semantica
%per il nostro linguaggio.
%
%\begin{flushleft}
%\textbf{Definizione 1}
%
%Una funzione modale è una formula contenente almeno una funzione modale
%\end{flushleft}
%
%In $T$ ovviamente c'è un numero infinito di funzioni modali esprimibili.
%In $S5$ ce n'è un numero finito, mentre sorprendentemente,
%in $S4$ ce ne sono infinite, nonostante le modalità esprimibili siano in numero finito.
%
\subsection{Una semantica per i sistemi di logica modale}
Dopo aver definito il linguaggio comune a tutti i sistemi di logica modale introdotti,
lo abbiamo usato unicamente come base di un calcolo che usa le regole di deduzione per costruire
dimostrazioni. Come ben sappiamo un linguaggio può anche essere usato per parlare di qualcosa,
di oggetti esterni al sistema formale.
Fare ciò significa associare ad ogni formula ben formata del linguaggio un oggetto
che rappresenti il suo significato.
Quando seguiamo questo procedimento diciamo che abbiamo fornito una semantica per il linguaggio.

Definire semantiche per i sistemi formali ci aiuta sia a confrontare sistemi formali
con strutture maggiormente conosciute e ad assicurarci
che alcune costruzioni corrispondano alla nostra intuizione,
sia a determinare alcune proprietà del sistema formale. E nel nostro caso ci aiuterà anche
a comprendere meglio quali siano le nozioni di modali rappresentate in ciascuno dei sistemi
che abbiamo definito e come differiscono tra di loro.


Come abbiamo fatto per l'apparato deduttivo, prima di dare una semantica ai nostri sistemi,
partiamo ricordando come è fatta la semantica classica per $CP$.

Per i nostri sistemi di logica modale possiamo pensare di fare qualcosa di simile,
però non possiamo aspettarci che $V_I$ per formule del tipo $\Box \alpha$ si basi solo
sul valore di $V_I(\alpha)$.

La semantica che useremo è in gran parte dovuta a Saul A. Kripke (vale la pena notare
che pubblicò un teorema di completezza per i sistemi di logica modale all'età di 17 anni)
e si basa su un concetto già introdotto da Leibniz, quello dei mondi possibili.
Quindi non supponiamo più che esista un'unica realtà, ma che esistono tanti mondi possibili,
e in più, e questa è una delle idee chiave della semantica di Kripke, da alcuni mondi
si può accedere ad altri mondi, cioè sapere come sono fatte altre realtà alternative.
Il mondo in cui si può accedere agli altri mondi sarà la base per distinguire
tra i vari concetti di necessità e tra i vari sistemi formali che abbiamo definito prima.

\subsection{Semantica per il Sistema T}
Definiamo un T-modello come una tripla $(W, R, I)$, in cui:
\begin{itemize}
\item $W$ è un insieme non vuoto i cui elementi saranno chiamati mondi;
\item $R$ è una relazione binaria riflessiva su $W$, detta relazione di accessibilità;
\item $I$ è una funzione dall'insieme $L \times W$ all'insieme $\{0, 1\}$;
\end{itemize}

L'insieme $W$ è l'insieme di tutti i mondi.
$I$ ci fornisce per ogni mondo la lista di proposizione vere
e quella delle proposizioni false, analogamente al suo corrispettivo nella semantica per $CP$.
$R$, infine, è la \textit{relazione di accessibilità},
essa ci permette di determinare a quali mondi si può accedere da un determinato mondo.
Quindi se $w_1, w_2 \in W e w_1 R w_2$, allora una persona in $w_1$ potrà accedere
al mondo $w_2$ e sapere quali proposizioni sono vere in $w_2$.
Il fatto che $R$ sia riflessiva ci dice che l'unica garanzia che abbiamo
è che ogni persona può accedere al proprio mondo, e questa è una garanzia molto scarsa.

Definiamo come prima una funzione di valutazione $V : L \times W \to \{0, 1\}$ che dipenderà da un dato T-modello
$(w, W, R, I)$.

Se $\alpha, \beta \in L$:
\begin{itemize}
\item $V(p, w) = I(p, w)$, $\forall p \in A$
\item $V(\neg \alpha, w) = 1 - V(\alpha, w)$
\item $V(\alpha \vee \beta, w) = max\{V(\alpha, w), V(\beta)\}$
\item $V(\Box \alpha, w) = min\{ V(\alpha, w') : w R w' \}$
\end{itemize}

Le prime tre regole sono pressoché identiche a quelle specificate per la semantica di $CP$,
tranne per il fatto che nella prima regola usiamo la funzione $I$ valutandola per il mondo w
a cui siamo interessati.

La novità è la quarta regola, quella riguardante la semantica di $\Box$.
Questa regola afferma che nel mondo $w$, $\alpha$ è necessariamente vera se e solo se
è $\alpha$ vera in tutti i mondi accessibili da $w$.

Diciamo che una formula $\alpha$ è T-valida e scriviamo $\vDash \alpha$
se per ogni T-modello $(w, W, R, I)$ $\forall w \in W. V(\alpha, w) = 1$.

Ci occupiamo adesso di mostrare che tutti i teoremi del sistema T sono formule valide
nella semantica che abbiamo fornito.

\begin{flushleft}
\textbf{Teorema di adeguatezza per T}
$\vdash \alpha \Rightarrow \vDash \alpha$ in $T$

\textit{Dim.}

Premettiamo alla dimostrazione vera e propria le seguenti osservazioni:

Se $(W, R, I)$ è un T-modello, $w \in W$ e $\alpha, \beta \in L$, allora:
$V(\alpha \rightarrow \beta, w) = V(\neg \alpha \lor \beta, w) = 0$ se e solo se
$V(\neg \alpha, w) = 0$ e $V(\beta, w) = 0$, ovvero, se e solo se $V(\alpha, w) = 1$ e $V(\beta, w) = 0$.

Se poi $V(\alpha \rightarrow \beta, w) = 1$ e $V(\alpha, w) = 1$, allora, se fosse
$V(\beta, w) = 0$, per quanto detto prima avremmo $V(\alpha \rightarrow \beta, w) = 1$, ma ciò
è assurdo, quindi deve essere $V(\beta, w) = 1$.

Occupiamoci ora di dimostrare il teorema di adeguatezza, la strategia che seguiremo è la seguente:
mostriamo che tutti gli assiomi sono T-validi e che le regole di deduzione conservano la T-validità.
Di conseguenza per le condizioni che abbiamo dato sulle formule ben formate che formano una dimostrazione,
deduciamo che tutte le formule ben formate che appaiono in una dimostrazione sono T-valide
ed in particolare l'ultima formula della sequenza finita è T-valida.
Questo sarà sufficiente per giungere alla tesi.

Iniziamo col mostrare che tutti gli assiomi sono T-validi:

Mostriamo che $\vDash \Box \alpha \rightarrow \alpha$:

Se $(W, R, I)$ è un T-modello, per quanto visto in precedenza abbiamo che:
$V(\Box \alpha \rightarrow \alpha, w) = 0 \Leftrightarrow V(\Box \alpha, w) = 1$ e $V(\alpha, w) = 1$.
Siccome $R$ è una relazione riflessiva, dalla definizione di $V$ segue che:
$V(\Box \alpha, w) <= V(\alpha, w)$ e quindi se $V(\Box \alpha, w) = 1$, si ha anche che
$V(\alpha, w) = 1$ e quindi otteniamo che
$\forall w \in W. V(\Box \alpha \rightarrow \alpha, w) = 1$ e per l'arbitrarietà
del T-modello otteniamo l'asserto.

Mostriamo che $\vDash \Box (\alpha \rightarrow \beta) \rightarrow (\Box \alpha \rightarrow \Box \beta)$:

Se $(W, R, I)$ è un T-modello, come prima abbiamo che:
$V(\Box (\alpha \rightarrow \beta) \rightarrow (\Box \alpha \rightarrow \Box \beta), w) = 0
\Leftrightarrow V(\Box (\alpha \rightarrow \beta), w) = 1$ e $V(\Box \alpha \rightarrow \Box \beta, w) = 0$.

Se $V(\Box(\alpha \rightarrow \beta), w) = 1$, allora: $\forall w' \in W. wRw' \Rightarrow V(\alpha \rightarrow \beta, w') = 1$
Se fosse $V(\Box \alpha, w) = 1$ e $V(\Box \beta, w) = 0$, allora si avrebbe la seguente
situazione: $\forall w' \in W. wRw' \Rightarrow V(\alpha, w') = 1$. Quindi avremmo che:
$\forall w' \in W. wRw' \Rightarrow V(\alpha \rightarrow \beta, w') = 1$ e $V(\alpha, w') = 1$,
dunque per l'osservazione precedente si avrebbe: $\forall w' \in W. wRw' \Rightarrow V(\beta, w') = 1$
e perciò $V(\Box \beta, w) = 1$, il che è assurdo.

Quindi se $V(\Box(\alpha \rightarrow \beta), w) = 1$, deve essere $V(\Box \alpha \rightarrow \Box \beta, w) = 1$,
e quindi per l'arbitrarietà di $w$ e del T-modello otteniamo l'asserto.

Ci resta da verificare la T-validità degli assiomi derivanti dagli schemi di assiomi ereditati da $CP$.
Gli assiomi di $CP$ sono ottenuti sostituendo in modo uniforme le metavariabili con
formule del linguaggio $L_{CP}$.
Per il teorema di completezza per $CP$, tutti gli assiomi sono formule valide per $CP$,
questo significa che fissata un'interpretazione $I$ per $CP$,
non importa quale sia il valore di verità delle formule sostituite nello schema,
l'assioma risultante sarà sempre vero.

Quindi, poiché, una volta forniti i valori di verità per $\alpha$ e $\beta$,
le regole per determinare i valori di verità delle formule $\neg \alpha$ e $\alpha \lor \beta$
sono le stesse sia per $CP$ che per il sistema T
($\neg$ e $\lor$ sono interpretati allo stesso modo sia in T che in $CP$) e poiché
gli schemi di assiomi ereditati da $CP$ contengono solo $\neg$ e $\lor$ come connettivi
(gli altri connettivi di $CP$ sono definiti in termini di questi due),
possiamo dedurre che tutte le possibili sostituzioni uniformi di formule di $L$
negli schemi di assiomi derivati da $CP$ forniscono formule ancora valide.

Ora dimostriamo che il \textit{Modus Ponens} conserva la validità:

Se $\alpha \rightarrow \beta$ e $\alpha$ sono T-valide, allora fissato un T-modello $(W, R, I)$,
$\forall w \in W. V(\alpha \rightarrow \beta, w) = 1$ e $V(\alpha, w) = 1$.
Allora per l'osservazione che abbiamo fatto prima otteniamo che:
$\forall w \in W. V(\beta, w) = 1$. Per l'arbitrarietà del T-modello otteniamo l'asserto.

Infine dimostriamo che la regola di necessitazione conserva la validità:
Fissato un T-modello $(W, R, I)$ e un mondo $w \in W$,
se $\alpha$ è T-valida, allora $\forall w' \in W.V(\alpha, w') = 1$,
quindi in particolare si ha: $\forall w' \in W. wRw' \Rightarrow V(\alpha, w') = 1$.
Quindi $V(\Box \alpha, w) = 1$ e per l'arbitrarietà di $w$ e del T-modello otteniamo l'asserto.


\begin{flushright}
$\bullet$
\end{flushright}
\end{flushleft}

\subsection{Semantica per i sistemi S4 e S5}
Con leggere modifiche alla definizione di T-modello possiamo costruire delle semantiche
anche per i sistemi S4 e S5.

\begin{definition}
Un S4-modello è un T-modello $(W, R, I)$ in cui $R$ oltre ad essere riflessiva è anche transitiva.
\end{definition}

\begin{definition}
Un S5-modello è un T-modello $(W, R, I)$ in cui $R$ è la relazione di equivalenza totale
(e in particolare è una relazione di equivalenza),
vale a dire che si ha: $\forall w_1, w_2 \in W. w_1 R w_2$.
\end{definition}

\begin{flushleft}
\textbf{Teorema di adeguatezza per S4}
$\vdash \alpha \Rightarrow \vDash \alpha$ in $S4$

\textit{Dim.}
Siccome un S4-modello è anche un T-modello e poiché ogni teorema del Sistema T
è anche un teorema del Sistema S4, possiamo sfruttare il teorema di adeguatezza per T
e dimostrare qui solo che l'assioma $S4$ è una formula valida:

Se $(W, R, I)$ è un S4-modello e $w \in W$,
se $V(\Box \alpha, w) = 1$, allora $\forall w_1 \in W. wRw_1 \Rightarrow V(\alpha, w_1) = 1$,
quindi poiché $R$ è riflessiva e transitiva
se $w_1, w_2 \in W$ e $wRw_1, w_1Rw_2$, risulta $wRw_2$, quindi per ipotesi,
$\forall w_2 \in W. w_1Rw_2 \Rightarrow V(\alpha, w_2) = 1$, quindi $V(\Box \alpha, w_1) = 1$
e perciò $\forall w_1 \in W. wRw_1 \Rightarrow V(\Box \alpha, w_1) = 1$,
dunque $V(\Box \Box \alpha, w) = 1$.
Per l'arbitrarietà di $w$ e dell'S4-modello abbiamo la tesi.


\begin{flushright}
$\bullet$
\end{flushright}
\end{flushleft}


\begin{flushleft}
\textbf{Teorema di adeguatezza per S4}
$\vdash \alpha \Rightarrow \vDash \alpha$ in $S5$

\textit{Dim.}
Siccome un S5-modello è anche un T-modello e poiché ogni teorema del Sistema T
è anche un teorema del Sistema S5, possiamo sfruttare il teorema di adeguatezza per T
e dimostrare qui solo che l'assioma $S5$ è una formula valida:

Se $(W, R, I)$ è un S5-modello e $w \in W$,

se $V(\diamond \alpha, w) = 1$, allora $\exists w_1 \in W. wRw_1$ e $V(\alpha, w_1) = 1$,

Se $w_2 \in W$ e $wRw_2$, allora siccome $R$ è simmetrica, $w_2Rw$ e poiché
è transitiva $w_2Rw_1$, ma $V(\alpha, w_1) = 1$, quindi
$\exists w_1 \in W. w_2Rw_1$ e $V(\alpha, w_1) = 1$, cioè $V(\diamond \alpha, w_2) = 1$.

Ciò ci permette di affermare che $\forall w_2 \in W. wRw_2 \Rightarrow V(\diamond \alpha, w_2) = 1$
e quindi $V(\Box \diamond \alpha, w) = 1$.
Per l'arbitrarietà di $w$ e dell'S5-modello otteniamo la tesi.

\begin{flushright}
$\bullet$
\end{flushright}
\end{flushleft}

\subsection{Proprietà dei sistemi T, S4, S5}
I teoremi di adeguatezza che abbiamo dimostrato ci dicono che ogni teorema
è anche una formula valida per i vari sistemi.
Questo significa che il concetto di semantico di validità che abbiamo poco fa definita
si concilia bene con quello di dimostrabilità. Ciò è una buona assicurazione circa
la bontà delle semantiche.

Il teorema di adeguatezza ci permette di dimostrare varie importanti proprietà tramite
strumenti semantici e che sarebbero difficili da dimostrare usando solo l'apparato deduttivo.

\begin{flushleft}
\textbf{Teorema}
S4 e T sono due sistemi distinti, ovvero l'assioma S4 non è una tesi del sistema T.

\textit{Dim.}

In virtù del teorema di adeguatezza se troviamo un T-modello nel quale l'assioma S4
sia falso, potremo dedurre che S4 non può essere un teorema del sistema T.

Consideriamo 3 mondi $W = \{w_1, w_2, w_3\}$, e definiamo la relazione di accessibilità
in questo modo:
\begin{itemize}
    \item $wRw, \forall w \in W$;
    \item $w_1Rw_2$;
    \item $w_2Rw_3$;
    \item Non ci sono altre relazioni tra mondi.
\end{itemize}

Siccome $A$ è non vuoto, esiste $p \in A$ e definiamo la seguente interpretazione $I:$
\begin{itemize}
    \item Se $x \in A \setminus \{p\}, \forall w \in W. I(x, w) = 0$;
    \item $I(p, w_1) = I(p, w_2) = 1$;
    \item $I(p, w_3) = 0$.
\end{itemize}

È facile vedere che $(W, R, I)$ è un T-modello, infatti $W$ è non vuoto e $R$ è una relazione binaria riflessiva.

Inoltre $V(p, w_1) = V(p, w_2) = 1$ e $V(p, w_3) = 0$, quindi per come è definita
la relazione di accessibilità, $V(\Box p, w_1) = 1$, mentre
$V(\Box p, w_2) = 0$. Quindi $\Box p$ non è vera in tutti i mondi accessibili da $w_1$
e quindi $V(\Box \Box p, w_1) = 0$ e perciò: $V(\Box p \rightarrow \Box\Box p, w_1) = 0$.
Quindi $\Box p \rightarrow \Box\Box p$ non è una formula valida per il sistema T e quindi
non è dimostrabile.
Nel sistema S4 invece è un'istanza dell'assioma (S4) e quindi è banalmente dimostrabile.
Possiamo perciò concludere che i sistemi S4 e T non hanno gli stessi teoremi e quindi non
sono lo stesso sistema formale.


\end{flushleft}

\begin{flushleft}
\textbf{Corollario}
S5 e T sono due sistemi distinti

\textit{Dim.}

Abbiamo dimostrato in precedenza che l'assioma (S4) è una tesi di S5 e quindi
per il teorema precedente, S5 ha più tesi di T.
\end{flushleft}

\begin{flushleft}
\textbf{Teorema}
S5 e S4 sono due sistemi distinti, ovvero l'assioma S5 non è una tesi del sistema S4.

\textit{Dim.}

In virtù del teorema di adeguatezza se troviamo un S4-modello nel quale l'assioma S5
sia falso, potremo dedurre che S5 non può essere un teorema del sistema S4.

Consideriamo 2 mondi $W = \{w_1, w_2\}$, e definiamo la relazione di accessibilità
in questo modo:
\begin{itemize}
    \item $wRw, \forall w \in W$;
    \item $w_1Rw_2$;
    \item Non ci sono altre relazioni tra mondi.
\end{itemize}

Siccome $A$ è non vuoto, esiste $p \in A$ e definiamo la seguente interpretazione $I:$
\begin{itemize}
    \item Se $x \in A \setminus \{p\}, \forall w \in W. I(x, w) = 0$;
    \item $I(p, w_1) = 1$;
    \item $I(p, w_2) = 0$.
\end{itemize}

È facile vedere che $(W, R, I)$ è un S4-modello, infatti $W$ è non vuoto e
$R$ è una relazione binaria riflessiva e transitiva.

Inoltre $V(p, w_1) = 1$ e $V(p, w_2) = 0$, quindi per come è definita
la relazione di accessibilità, $V(\diamond p, w_1) = 1$, mentre
$V(\diamond p, w_2) = 0$. Quindi $\diamond p$ non è vera in tutti i mondi accessibili da $w_1$
e quindi $V(\Box \diamond p, w_1) = 0$ e perciò: $V(\diamond p \rightarrow \Box\diamond p, w_1) = 0$.
Quindi $\diamond p \rightarrow \Box\diamond p$ non è una formula valida per il sistema S4 e quindi
non è dimostrabile.
Nel sistema S5 invece è un'istanza dell'assioma (S5) e quindi è banalmente dimostrabile.
Possiamo perciò concludere che i sistemi S5 e S4 non hanno gli stessi teoremi e quindi non
sono lo stesso sistema formale.


\end{flushleft}

In definitiva abbiamo dimostrato che i sistemi T, S4 e S5 sono tre sistemi distinti
e questo significa che presentano tre nozioni diverse di ``necessità'' e ``possibilità''.

Forniamo, infine quest'ultima dimostrazione:

\begin{flushleft}
\textbf{Teorema}
Esiste $\alpha \in L$, tale che $\alpha \rightarrow \Box \alpha$ non è una tesi di T (e quindi nemmeno di S4 e S5)

\textit{Dim.}

Costruiamo un T-modello in cui $\alpha \rightarrow \Box \alpha$ è falsa.

Consideriamo 2 mondi $W = \{w_1, w_2\}$, e definiamo la relazione di accessibilità
in questo modo:
\begin{itemize}
    \item $wRw, \forall w \in W$;
    \item $w_1Rw_2$;
    \item Non ci sono altre relazioni tra mondi.
\end{itemize}

Siccome $A$ è non vuoto, esiste $p \in A$ e definiamo la seguente interpretazione $I:$
\begin{itemize}
    \item Se $x \in A \setminus \{p\}, \forall w \in W. I(x, w) = 0$;
    \item $I(p, w_1) = 1$;
    \item $I(p, w_2) = 0$.
\end{itemize}


È facile vedere che $(W, R, I)$ è un T-modello, infatti $W$ è non vuoto e
$R$ è una relazione binaria riflessiva.

Inoltre $V(p, w_1) = 1$ e $V(p, w_2) = 0$, quindi per come è definita
la relazione di accessibilità, $V(\Box p, w_1) = 1$, perciò: $V(p \rightarrow \Box p, w_1) = 0$.
Quindi $\diamond p \rightarrow \Box\diamond p$ non è una formula valida per il sistema T e quindi
non è dimostrabile.
Abbiamo ottenuto così la tesi.

\end{flushleft}

Nei sistemi di logica modale in cui per ogni
$\alpha \in L$ vale $\alpha \leftrightarrow \Box \alpha$ si dice che avviene il collasso
delle modalità. Questi sistemi sono solo una versione barocca, con simboli inutili,
di un sistema formale per il calcolo proposizionale:
gli operatori modali sono inutili, infatti la formula
precedente ci dice che affermare $\Box \alpha$ è esattamente lo stesso che affermare $\alpha$
e quindi non aggiungono niente di nuovo.

Il teorema che abbiamo appena dimostrato ci assicura che nei sistemi definiti non avviene
il collasso delle modalità e che quindi gli operatori modali introdotti non sono solo inutili
orpelli che non aggiungono niente a $CP$.

\subsection{Procedura di decisione per T}
Ci occupiamo ora di definira una procedura di decisione per l'insieme delle formule T-valide.

Per verificare la T-validità di una formula $\alpha$ dovremmo verificare che essa sia vera in tutti
i mondi di tutti i T-modelli, mentre ciò che ci proponiamo di fare è dedurre quali sono i valori di verità
che le sottoformule di $\alpha$ devono assumere nei vari mondi affinché esista un T-modello
in cui esista un mondo in cui $\alpha$ sia falsa.
Se così facendo riusciamo ad assegnare per ogni mondo un valore di verità ad ogni variabile proposizionale
senza cadere in contraddizione, ossia senza dover assegnare ad una sottoformula di $\alpha$ sia $1$ che $0$,
allora ciò significa che non esiste alcun T-modello che falsifichi $\alpha$, visto che se un tale T-modello
esistesse, necessariamente ci sarebbe una valutazione che associa
alla stessa formula sia $1$ che $0$ ma ciò è impossibile per un T-modello.
Al contrario se non incontriamo alcuna inconsistenza, saremo in grado
di determinare un T-modello in cui esiste un mondo in cui $\alpha$ è falsa.

Presentiamo la procedura di decisione mostrando come applicarla a vari esempi che ne
esemplificano bene tutte le regole:

Per ogni mondo disegniamo un rettangolo che conterrà delle formule.
Successivamente per indicare che un mondo $w_2$ è accessibile da un'altro mondo $w_1$,
colleghiamo i due rettangoli con una freccia che va dal rettangolo del mondo $w_1$
a quello del rettangolo del mondo $w_2$.

Nel rettangolo del mondo $w_1$, quando scriviamo un valore di verità $v$ sotto ad una
variabile proposizionale $p$ intendiamo indicare che nel modello che stiamo costruendo
vale $I(p, w_1) = v$.
Mentre se scriviamo un valore di verità $v$ sotto ad un connettivo, intendiamo indicare
che per la formula $\alpha$ che ha come connettivo più esterno quel connettivo,
vale $V(\alpha, w_1) = v$ nel modello che stiamo costruendo.

Quindi se scriviamo
$\underset{1}{q} \underset{1}{\rightarrow} \underset{1}{q}$ nel rettangolo di $w_1$
intendiamo indicare che $I(q, w_1) = 1$ e che $V(q \rightarrow q, w_1) = 1$.

Iniziamo considerando la formula
$$\Box(p \rightarrow \Box(q \rightarrow r)) \rightarrow \diamond(q \rightarrow (\Box p \rightarrow \diamond r))$$
Vogliamo che questa formula sia falsa nel primo mondo che introduciamo $w_1$.
Quindi nel rettangolo per $w_1$ scriviamo dapprima:
$$\Box(p \rightarrow \Box(q \rightarrow r)) \underset{0}{\rightarrow} \diamond(q \rightarrow (\Box p \rightarrow \diamond r))$$
Perciò per le regole di valutazione che abbiamo dato:
$$\underset{1}{\Box}(p \rightarrow \Box(q \rightarrow r)) \underset{0}{\rightarrow} \underset{0}\diamond(q \rightarrow (\Box p \rightarrow \diamond r))$$
Il fatto che $\diamond(q \rightarrow (\Box p \rightarrow \diamond r))$ sia falsa in $w_1$ ci porta a dire che:
$$\underset{1}{\Box}(p \rightarrow \Box(q \rightarrow r)) \underset{0}{\rightarrow} \underset{0}\diamond(q \underset{0}{\rightarrow} (\Box p \rightarrow \diamond r))$$
E quindi:
$$\underset{1}{\Box}(p \rightarrow \Box(q \rightarrow r)) \underset{0}{\rightarrow} \underset{0}\diamond(\underset{1}{q} \underset{0}{\rightarrow} (\Box p \underset{0}{\rightarrow} \diamond r))$$
Continuando nell'argomento di $\diamond$ deduciamo che:
$$\underset{1}{\Box}(p \rightarrow \Box(q \rightarrow r)) \underset{0}{\rightarrow} \underset{0}\diamond(\underset{1}{q} \underset{0}{\rightarrow} (\underset{1}{\Box} \underset{1}{p} \underset{0}{\rightarrow} \underset{0}{\diamond} \underset{0}{r}))$$
Abbiamo determinato un valore di verità per tutte le variabili proposizionali nella formula, non ci resta che vedere se c'è qualche inconsistenza nella valutazione.
Se assegnamo alle variabili proposizionali dell'antecedente dell'implicazione più esterna i valori
che abbiamo trovato otteniamo:
$$\underset{1}{\Box}(\underset{1}{p} \rightarrow \Box(\underset{1}{q} \rightarrow \underset{0}{r})) \underset{0}{\rightarrow} \underset{0}\diamond(\underset{1}{q} \underset{0}{\rightarrow} (\underset{1}{\Box} \underset{1}{p} \underset{0}{\rightarrow} \underset{0}{\diamond} \underset{0}{r}))$$
Perciò:
$$\underset{1}{\Box}(\underset{1}{p} \underset{0}{\rightarrow} \underset{0}{\Box}(\underset{1}{q} \underset{0}{\rightarrow} \underset{0}{r})) \underset{0}{\rightarrow} \underset{0}\diamond(\underset{1}{q} \underset{0}{\rightarrow} (\underset{1}{\Box} \underset{1}{p} \underset{0}{\rightarrow} \underset{0}{\diamond} \underset{0}{r}))$$

Ma abbiamo un'inconsistenza, perché abbiamo:
$$\underset{1}{\Box}(\underset{1}{p} \underset{0}{\rightarrow}\underset{0}{\Box}(\underset{1}{q} \underset{0}{\rightarrow} \underset{0}{r}))$$
cioè in un primo momento abbiamo dedotto che in $w_1$ questa formula deve essere vera in $w_1$,
ma per le assegnazioni che abbiamo trovato per le variabili $p, q, r$, essa risulta essere falsa.

Abbiamo perciò ottenuto che la seguente formula è T-valida:
$$\Box(p \rightarrow \Box(q \rightarrow r)) \rightarrow \diamond(q \rightarrow (\Box p \rightarrow \diamond r))$$

In questo caso siamo riusciti a decidere circa la validità della formula fornitaci
senza dover introdurre nuovi mondi, ma ciò avviene solo di rado.

E allora descriviamo la procedura di decisione come segue:

Si inizia disegnado un rettangolo  con etichetta $w_1$ dentro cui scriviamo la formula $\alpha$
di cui si vuole controllare la validità, seguendo il procedimento delineato sopra
si iniziano a desumere dei valori di verità da assegnare alle sottoformule ben formate di $\alpha$
in modo che $\alpha$ risulti falsa.

Ogni rettangolo è accessibile a se stesso.

Se in un rettangolo con etichetta $w_i$ si assegna il valore $1$ ($0$) ad una sottoformula
del tipo $\Box \beta$ ($\diamond \beta$), allora in tutti i rettangoli accessibili da $w_i$ si dovrà
associare a $\beta$ il valore di verità $1$ ($0$) e per ricordarcelo scriviamo questa condizione
vicino ad ogni rettangolo in cui essa vale.

Se in un rettangolo con etichetta $w_i$ c'è una formula $\beta$
in cui ci sono delle sottoformule per cui univoco il valore di verità da associare loro,
come può succedere quando $\gamma \lor \delta$ deve essere vera: In questo caso abbiamo
tre possibili assegnamenti:
$\gamma$ vera e $\delta$ vera, $\gamma$ vera e $\delta$ falsa, $\gamma$ falsa, $\delta$ vera.
Allora procediamo creando più versioni alternative dello stesso rettangolo indicate con $w_i(n)$,
e facciamo in modo che in ognuno di essi alle sottoformule con valori ambigui siano assegnati
tutti le possibili combinazioni di valori di verità.
Poi per ogni rettangolo alternativo $w_i(n)$ facciamo una copia del diagramma in costruzione
a cui però al posto del rettangolo $w_i$ sostituiamo il
rettangolo alternativo $w_i(n)$. Otteniamo perciò diversi diagrammi
e per ciascuno di essi continuiamo con la procedura che stiamo definendo.

Se in un rettangolo con etichetta $w_i$ si assegna il valore $0$ ($1$) ad una sottoformula
del tipo $\Box \beta$ ($\diamond \beta$), allora bisogna aggiungere un nuovo
rettangolo con una nuova etichetta $w_j$ che sia accessibile dal rettangolo con etichetta $w_i$
e in questo nuovo rettangolo scriviamo la formula $\beta$ ($\neg \beta$).
Su questo nuovo rettangolo si esegue di nuovo tutto il procedimento che stiamo descrivendo.

Una volta che si si sono applicate tutte queste regole e non se ne possono applicare altre,
diciamo che abbiamo ottenuto un sistema completo di T-diagrammi per la formula $\alpha$.

Se in un rettangolo viene assegnata ad una sottoformula sia il valore $1$ che il valore $0$,
diciamo che il rettangolo è esplicitamente inconsistente.

Se tutti i rettangoli alternativi ad un rettangolo sono esplicitamente inconsistenti,
allora il rettangolo è esplicitamente inconsistente.

Il predecessore diretto di un rettangolo $w_i$ è il rettangolo che ha causato
la creazione di $w_i$.

Se un rettangolo è predecessore diretto di un rettangolo esplicitamente inconsistente,
allora esso è esplicitamente inconsistente.

Per le osservazioni precedenti otteniamo che una formula è T-valida
se e solo se il diagramma con etichetta $w_1$ è esplicitamente inconsistente.

Forniamo alcuni esempi di determinazioni di sistemi completi di T-diagrammi:

\textbf{FORNIRLI}


Adesso dobbiamo dimostrare che la procedura che abbiamo definito è effettivamente
una procedura di decisione, vogliamo cioè dimostrare che per ogni formula di $L$
ci permette di determinare in modo univoco e in un numero finito di passi
se questa è T-valida o meno.

Il fatto che il procedimento funzioni e che se il procedimento termina la risposta è univoca
sono ovvi, dobbiamo solo mostrare che per costruire un sistema completo di T-diagrammi
sono necessari solo un numero finito di passi:

Iniziamo osservando che in un rettangolo $w_i$
la determinazione dei valori di verità delle sottoformule della formula $\alpha$,
nel caso in cui non ci siano ambiguità comporta solo un numero finito di passaggi.
Se c'è qualche ambiguità bisogna usare la regola per introdurre i rettangoli alternativi.
Il numero di possibili combinazioni di valori di verità è finito e quindi
i rettangoli alternativi introdotti sono in numero finito.

Notiamo ora che se è necessario introdurre un nuovo mondo $w_j$ accessibile da $w_i$,
allora in questo mondo le formule avranno un numero strettamente inferiore di operatori
modali rispetto al numero di operatori modali nelle formule di $w_i$, questo perché
le formule scritte in $w_j$ si ottengono rimuovendo un operatore da una sottoformula
della formula scritta in $w_i$.
Dal momento che in $w_1$ c'è solo un numero finito di operatori
modali, allora il numero di mondi da dover introdurre sarà necessariamente finito.

Abbiamo perciò che il numero di rettangoli in un sistema completo di T-diagrammi è finito
e perciò per ciascuno di essi le operazioni da compiere sono finite. In definitiva
la procedura che abbiamo definito termina in un numero finito di passi.

\subsection{Procedura di decisione per S4}
Cerchiamo di adattare la procedura appena definita per decidere le formule S4-valide.
Siccome un S4-modello ha una relazione riflessiva e transitiva, allora bisogna stare
attenti che quando si introducano mondi nuovi le relazioni di accessibilità siano
correttamente registrate in modo che se $w_3$ è accessibile da $w_2$ e $w_2$ è accessibile
da $w_1$, allora $w_3$ risulti accessibile da $w_1$.
Questa situazione ci porta delle complicazione, per illustrare bene il problema
mostriamo cosa succede con la seguente formula:

$$MLp \rightarrow LMp$$
\textbf{fare il grafico}

Il problema che riscontriamo è che essendo la relazione di accessibilità transitiva,
quando in $w_i$ si ha che $\Box \alpha$ è vera, dobbiamo registrare in ogni altro mondo
accessibile da $w_i$ che $\alpha$ è vera, e in particolare se c'è un mondo $w_j$
accessibile da $w_i$ e un mondo $w_k$ accessibile da $w_j$, allora $w_k$ è accessibile
da $w_i$ e quindi sia in $w_j$ che in $w_k$ dobbiamo registrare che $\alpha$ deve essere vera,
ma ciò significa che ora nei mondi accessibili da un mondo $w_m$ non è più detto
che il numero di operatori modali presenti nelle formule sia diminuito e quindi
può capitare come nell'esempio precedente di avere un numero infinito di mondi.

Introduciamo la seguente definizione:
\begin{definition}
diciamo che i mondi $w_1, ..., w_n$ formano una catena se:
$w_{i+1}$ è accessibile da $w_i$ per ogni $i = 1, ..., n-1$
\end{definition}

Se in un rettangolo $w_{i+1}$ della catena ci sono delle condizioni introdotte
a partire dal mondo $w_i$, allora quelle condizioni devono valere
in ogni altro elemento successivo della catena.

Per ovviare al problema che abbiamo riscontrato, osserviamo che
in ogni rettangolo sono presenti solo
sottoformule ben formate della formula $\alpha$ presente in $w_1$,
siccome tutte le sottoformule possibili di $\alpha$ sono in numero finito,
allora ad un certo punto, quando dovremo aggiungere un nuovo rettangolo $w_i$
al termine di una catena, ci sarà sicuramente un rettangolo più in alto nella catena
che contenga la formula e tutte le condizioni che vogliamo aggiungere $w_i$.
Inoltre notiamo che anche se questo rettangolo contiene più condizioni di $w_i$,
ciò non è un problema, perché essendo $w_i$ accessibile da questo rettangolo,
quelle condizioni valgono anche in $w_i$, facendo parte della stessa catena.
capiterà sicuramente che esista un altro rettangolo già introdotto
con la stessa formula.

I T-modelli in cui la relazione tra i rettangoli è transitiva e si usa la regola
appena definita vengono detti S4-modelli.

Con questa modifica possiamo dimostrare che il numero di rettangoli in un sistema
completo di S4-diagrammi è finito:

Una catena di rettangoli è finita, per quanto detto prima.
Ogni catena inizia da $w_1$, siccome si introducono rettangoli
solo in corrispondenza di operatori modali, e ogni rettangolo contiene solo un numero finito
di operatori modali, allora il numero di catene necessarie è finito
così come il numero di rettangoli.

\subsection{Procedura di decisione per S5}
Un S5-diagramma è un S4-diagramma in cui, però, ogni rettangolo è accessibile
da ogni altro rettangolo. Con un argomento molto simile a quello usato
per dimostrare che la costruzione di un sistema completo di S4-diagrammi termina dopo
un numero finito di passi, si può dimostrare lo stesso risultato per gli S5-diagrammi.
Quindi anche per S5 c'è una procedura di decisione per le formule S5-valide.


\subsection{Teorema di completezza per T}
Data una formula $\alpha$, una volta costruito il suo sistema completo di T-diagrammi,
associamo ad ogni rettango $w_i$ del sistema una formula $w_i'$ definita in questo modo:

nel rettangolo $w_i$ c'è una formula $\beta$ che deve essere falsa ed eventualmente delle condizioni,
cioè un numero finito di formule $\gamma_1, ...\gamma_n$ che devono essere vere, ereditate
dal suo predecessore diretto, allora definiamo $w_i'$ come la formula
$\beta \lor \neg \gamma_1 \lor ... \lor \neg \gamma_n$.

\begin{flushleft}
\textbf{Lemma}
Dato un sistema completo di T-diagrammi e un rettangolo $w_i$,
se $\beta$ è una sottoformula $w_i'$ e a $\beta$ è assegnato il valore di verità falso,

allora $\vdash \beta \rightarrow w_i'$ in T

Se invece a $\beta$ è associato il valore di verità vero,


allora $\vdash \neg \beta \rightarrow w_i'$ in T

\textit{Dim.}

Supponiamo che a $\beta$ sia associato il valore di verità falso, l'altro caso è analogo,
allora se $\delta_1, ..., \delta_n$ sono tutte le sottoformule ben formate di $w_i'$ è
facile vedere che
$$(\Box \delta_1 \rightarrow \delta_1) \rightarrow ... \rightarrow (\Box \delta_n \rightarrow \delta_n) \rightarrow (\beta \rightarrow w_i')$$
è ottenuta da una famiglia di tautologie di $CP$, infatti l'unico modo per rendere
falsa questa formula è assegnare a $\delta_k$ $1$ ogni volta che associamo a $\delta_k$ $1$,
per $k = 1 ... n$, assegnare $1$ a $\beta$ e $0$ a $w_i'$.
Ma sappiamo, per quanto dedotto dal T-diagramma che se $w_i'$ ha valore di verità $1$,
allora $\beta$ deve avere valore di verità $0$.

Quindi otteniamo l'asserto.

\end{flushleft}


\begin{flushleft}
\textbf{Lemma}

Dato un sistema completo di T-diagrammi e un rettangolo $w_i$,
se in $w_i$ c'è un'inconsistenza, allora $\vdash w_i'$ in T.

\textit{Dim.}

Siccome $w_i$ è esplicitamente inconsistente, esiste una sottoformula $\beta$ di $w_i'$
tale che ad essa è associata sia il valore di verità $0$ che il valore di verità $1$,
allora per il lemma precedente si ha:
$$\vdash \beta \rightarrow w_i'$$
e
$$\vdash \neg \beta \rightarrow w_i'$$

È facile provare che
$(\neg \beta \rightarrow \alpha) \rightarrow (\beta \rightarrow \alpha) \rightarrow \alpha$
è una tautologia in $CP$ e quindi è un teorema in $CP$ e dunque è un teorema nel sistema T.

Allora applicando due volte il modus ponens e sfruttando anche i teoremi che abbiamo mostrato poco sopra,
otteniamo che $\vdash w_i'$.

\end{flushleft}


\begin{flushleft}
\textbf{Lemma}

Dato un sistema completo di T-diagrammi e un rettangolo $w_i$, se $\vdash w_i'$
e $w_j$ è il suo predecessore diretto,
allora $\vdash w_j'$

\textit{Dim.}

Se $w_i$ e $w_j$ sono lo stesso rettangolo, non c'è niente da dimostrare.
Mentre se $w_i$ e $w_j$ sono due rettangoli diversi, siccome $w_i$ è costruito
a partire da $w_j$ secondo le regole della procedura, allora in $w_i$ c'è
una formula $\gamma$ a cui è assegnato il valore $0$ e tale che in $w_j$ c'è
$\Box \gamma$ a cui è assegnato il valore $0$ e poi ci sono delle formule
$\beta_1, ..., \beta_n$ a cui è assegnato il valore $1$ e tali che in $w_j$
a $\Box \beta_1, ..., \Box \beta_n$ è assegnato il valore $1$ in $w_i$.
Per questo motivo $w_j'$ è la formula: $\neg \beta_1 \lor ... \neg \beta_n \lor \gamma$.

Quindi per ipotesi abbiamo:
$$\vdash \neg \beta_1 \lor ... \neg \beta_n \lor \gamma$$
per la regola di necessitazione, otteniamo:
$$\vdash \Box(\neg \beta_1 \lor ... \neg \beta_n \lor \gamma)$$
Applicando la definizione di $\rightarrow$:
$$\vdash \Box(\beta_1 \rightarrow ... \beta_n \rightarrow \gamma)$$
Per applicazioni ripetute dell'assioma $K$:
$$\vdash (\Box \beta_1 \rightarrow ... \Box \beta_n \rightarrow \Box \gamma)$$
E quindi per definizione di $\rightarrow$:
$$\vdash (\neg \Box \beta_1 \lor ... \neg \Box \beta_n \lor \Box \gamma)$$

Siccome in $w_j'$ abbiamo assegnato a $\Box \beta_m$ il valore di verità $1$, per $m = 1...n$,
per il lemma mostrato prima abbiamo che:
$$\vdash \neg \Box \beta_1 \rightarrow w_j'$$
$$...$$
$$\vdash \neg \Box \beta_n \rightarrow w_j'$$

Allora, usando un ragionamento valido in $CP$, otteniamo:
$$\vdash (\neg \Box \beta_1 \lor ... \neg \Box \beta_n \lor \Box \gamma) \rightarrow w_j'$$

Dunque per Modus Ponens si ricava:
$$\vdash w_j'$$
cioè la tesi.

\end{flushleft}

Consideriamo ora il caso in cui un rettangolo contenga delle sottoformule
a cui non è possibile dare un valore di verità univoco e per cui bisogna
usare la regola dei rettangoli alternativi. Se nelle formule rimuoviamo
tutte le occorrenze di connettivi non primitivi, sostituendoli con la loro definizione
in termini di $\neg$ e $\lor$, allora otteniamo che l'unico connettivo
per cui si può presentare una situazione di ambiguità e $\lor$ e per le sue sottoformule
sono possibili tre assegnamenti diversi. Inoltre ricordiamo che un rettangolo a cui è stata applicata
la regola dei rettangoli alternativi è esplicitamente inconsistente se tutti i
rettangoli alternativi sono inconsistenti. Quindi dato un rettangolo
a cui è applicata la regola dei rettangoli alternativi non perdiamo di generalità
se assumiamo che i rettangoli alternativi siano tre. Allora dimostriamo il seguente lemma:

\begin{flushleft}
\textbf{Lemma}
$$\vdash w_i(1)', \vdash w_i(2)', \vdash w_i(3)' \rightarrow \vdash w_i'$$

\textit{Dim.}

Siccome in $w_i$ c'è un'ambiguità sappiamo che c'è una sottoformula in $w_i'$
del tipo $\beta \lor \gamma$ a cui è stato assegnato $1$, quindi vale:
$$\vdash \neg (\beta \lor \gamma) \rightarrow w_i'$$

In $w_i(1)$ a $\beta$ è associato $1$, a $\gamma$ $0$, quindi $w_i(1)'$ è $w_i' \lor \neg \beta \lor \gamma$.

In $w_i(2)$ a $\beta$ è associato $0$, a $\gamma$ $1$, quindi $w_i(2)'$ è $w_i' \lor \beta \lor \neg \gamma$.

In $w_i(3)$ a $\beta$ è associato $1$, a $\gamma$ $1$, quindi $w_i(3)'$ è $w_i' \lor \neg \beta \lor \neg \gamma$.

Perciò, per ipotesi, abbiamo rispettivamente:

$$\vdash w_i' \lor \neg \beta \lor \gamma$$
$$\vdash w_i' \lor \beta \lor \neg \gamma$$
$$\vdash w_i' \lor \neg \beta \lor \neg \gamma$$

Ora usiamo la tautologia di $CP$:
$$\vdash (\neg (\delta \lor \epsilon) \rightarrow \rho) \rightarrow
         (\rho \lor \neg \delta \lor \neg \epsilon) \rightarrow (\rho \lor \delta \lor \neg \epsilon)
           \rightarrow (\rho \lor \neg \delta \lor \epsilon) \rightarrow \rho$$

Sostituendo al posto di $\rho$ la formula $w_i'$, al posto di $\delta$ la formula $\beta$
e al posto di $\epsilon$ la formula $\gamma$.

Dunque abbiamo, sfruttando anche la definizione di $w_i(1)'$, $w_i(3)'$, $w_i(3)'$:
$$\vdash (\neg (\beta \lor \epsilon) \rightarrow w_i') \rightarrow
         (w_i(1)' \rightarrow w_i(2)' \rightarrow w_i(3)' \rightarrow w_i')$$

Allora applicando quattro volte il Modus Ponens, otteniamo $\vdash w_i'$
\end{flushleft}

Questi lemmi che abbiamo dimostrato ci permettono di concludere facilmente
che se in un sistema completo di T-diagrammi per la formula c'è un rettangolo esplicitamente inconsistente,
allora $\vdash w_1'$.

Quindi se $\vDash \alpha$, allora il sistema completo di T-diagrammi per $\alpha$ è inconsistente
e quindi $\vdash w_1'$, ma $w_1' = \alpha$, quindi $\vdash \alpha$.

\subsection{Teoremi di completezza per S4 e S5}
È possibile mostrare dei teoremi di completezza anche per S4 e S5 e una loro dimostrazione
è reperibile su [Hughes-Cresswell]

\section{Conclusioni}
Abbiamo inizialmente introdotto tre sistemi di logica modale: T, S4 ed S5,
dicendo che non era chiaro se ci fosse un sistema da preferire agli altri.
Dopo aver introdotto delle semantiche di Kripke per essi, abbiamo mostrato
come queste semantiche rispecchino perfettamente l'apparato deduttivo dei sistemi formali.
Ora possiamo usare i concetti introdotti da queste semantiche per fare alcune riflessioni
circa la natura dei tre sistemi introdotti. Infatti è emerso dallo studio che abbiamo condotto
che i tre sistemi rappresentano connotazioni diverse del concetto di ``necessità''
che abbiamo in mente.
Ad esempio nel sistema S5, abbiamo visto grazie alla semantica, che è possibile
determinare che una proposizione è necessaria solo quando questa è vera in tutti i mondi
possibili. Mentre in S4 e in T, ciò non è detto, infatti in questi sistemi una proposizione
è necessaria in un mondo quando è vera in tutti i mondi accessibili da quel mondo.
Questa differenza di definizioni mette in mostra una proprietà importante della necessità,
infatti non è detto che una persona sia in grado di concepire tutti i mondi possibili,
ma la nostra capacità di concepire stati di cose diversi è sicuramente condizionata dal mondo
in cui viviamo. Proprio su quest'osservazione possiamo indagare sulla la distinzione tra T e S4,
infatti in S4 quando si concepisce un mondo alternativo è possibile concepire quel mondo
come se ci vivessimo dentro, e quindi è per noi possibile concepire tutti i mondi che sarebbero
concepibili se vivessimo in quel mondo. Al contrario in T non è così, concepire un mondo
significa solo pensare ad un mondo alternativo, senza che questo influisca sulla nostra capacità
di concepire mondi alterantivi.

In definitiva ciò che viene messo in evidenza dalle semantiche di Kripke per questi sistemi è
che il concetto di necessità è in un certo senso collegato al concetto di conoscenza, più
è possibile conoscere mondi alternativi, più sarà forti saranno le proposizioni che possiamo
affermare essere vere.

Questo fatto ci spinge a pensare alla pluralità di sistemi modali possibili
come a una fonte di ricchezza di sfumature del concetto di ``necessità'' e perciò non ha molto
senso chiedersi quale sia il sistema formale migliore, mentre ha più senso usare tali sistemi
come strumento per un'analisi più approfondita per distinguere le varie sfumature del concetto
di ``necessità'' che si usano nei ragionamenti.
Alcuni esempi di questo approccio sono forniti da Lemmon [Lemmon] e terminiamo riportandone qui alcuni:
Se vogliamo interpretare $\Box \alpha$ come ``è analiticamente vero che $\alpha$'', allora il sistema
formale S5 è quello corretto. Mentre se vogliamo interpretare $\Box \alpha$ come
è ``informalmente dimostrabile in matematica che $\alpha$'', allora il sistema corretto è S4.



\end{document}
